\documentclass[ILAS2025-program.tex]{subfiles}

\begin{document}

\parindent=0pt

\section{Abstracts of Plenary Talks}
 \hypertarget{down0009}{}\begin{ilasabstract}
   \talktitle{nan}
    
    \textbf{Haim Avron}, \info{11:00\textrm{--}12:00 @ SYS Hall (June 27, Friday)} \hfill \hyperlink{up0009}{$\Uparrow$}
    
    \mtskip
    nan\end{ilasabstract}
     \hypertarget{down0005}{}\begin{ilasabstract}
   \talktitle{Matrix structures in queueing and network models: An overview}
    
    \textbf{Dario Andrea Bini}, \info{09:00\textrm{--}10:00 @ SYS Hall (June 25, Wednesday)} \hfill \hyperlink{up0005}{$\Uparrow$}
    
    \mtskip
    We provide an overview of some results concerning classes of matrices involved in queueing and network models. We show specific examples where the analysis of matrix structures, besides providing a better understanding of the original problem, is a fundamental step in designing effective ad hoc solution algorithms.

Two specific issues are considered:  the analysis of random walks on a regular grid in the quarter plane, and the assessment of the centrality of the edges in a graph. Both issues derive from the analysis of relevant real-world problems and are modeled by Markov chains describing random walks on a graph. 

In the first topic, the graph is a regular grid, and the specific features of the problem lead to semi-infinite transition probability matrices with a two-level tridiagonal and almost Toeplitz structure. We show that these matrices live in a suitable infinite-dimensional structured matrix algebra which is also a Banach space with respect to the infinity norm. This fact will be crucial to designing effective fast algorithms for computing the steady-state vector of the associated Markov chain.

In the second topic, the graph is typically a road map of a city or region. Thus, the Toeplitz structure of the associated transition probability matrix is lost. However, the sparsity and band structure of this matrix will allow us to easily compute a centrality measure of the edges defined in terms of Kemeny's constant of the associated Markov chain. The effectiveness of the model and the solution algorithms is tested on the road maps of Pisa and Tuscany.
\end{ilasabstract}
     \hypertarget{down0006}{}\begin{ilasabstract}
   \talktitle{Clustering in graphs with high clustering coefficients}
    
    \textbf{Fan Chung}, \info{08:00\textrm{--}09:00 @ SYS Hall (June 26, Thursday)} \hfill \hyperlink{up0006}{$\Uparrow$}
    
    \mtskip
    Many real world networks possess the so-called small world phenomenon where every node is relatively close to every other node and have a large clustering coefficient, i.e., friends of friends are likely friends. The task of learning an adequate similarity measure on various feature spaces often involves  graphs with high clustering coefficients.
We investigate the clustering effect in sparse  clustering graphs by examining the structural and spectral  properties as well as the enumeration  of patterns. In addition, we consider random graph models for clustering graphs that can be use to analyze the behavior of complex networks.\end{ilasabstract}
     \hypertarget{down0003}{}\begin{ilasabstract}
   \talktitle{The canonical form for congruence: some history and applications}
    
    \textbf{Fernando De Terán}, \info{09:00\textrm{--}10:00 @ SYS Hall (June 24, Tuesday)} \hfill \hyperlink{up0003}{$\Uparrow$}
    
    \mtskip
    A square matrix $A$ can be either seen as a linear map or as a bilinear form. When considered as a linear map, it is natural to introduce the relation of ``similarity'', $P^{-1}AP$ (with $P$ invertible), which is a change of basis that allows us to represent the linear map in a simpler form, in particular in the well-known ``Jordan canonical form''. When considering the matrix $A$ as a bilinear form, the natural relation instead is the one of ``congruence'', $P^\top AP$ (with $P$ invertible), which is the suitable change of basis for bilinear forms. Is there a canonical form for such a relation? The answer is yes, and actually some different canonical forms for congruence have been introduced over the years since the 1930's. In this talk I will introduce the most recent one (by Horn and Sergeichuk, 2006), review the history of this and the other canonical forms, and show some applications in the context of my past and current research, including:
\begin{itemize}
    \item The solution of the equation $AX+X^\top A=0$ and its connection to $\top$-palindromic pencils.
    \item The consistency of the equation $X^\top AX=B$ when $B$ is either symmetric or skew-symmetric.
\end{itemize}\end{ilasabstract}
     \hypertarget{down0001}{}\begin{ilasabstract}
   \talktitle{Various inequalities for quasi-arithmetic mean and quasi-geometric type means of matrices}
    
    \textbf{Fumio Hiai}, \info{09:30\textrm{--}10:30 @ SYS Hall (June 23, Monday)} \hfill \hyperlink{up0001}{$\Uparrow$}
    
    \mtskip
    Our targets in this talk are the quasi extensions of the weighted arithmetic mean and of the weighted geometric type means, including the weighted geometric mean, the weighted spectral geometric mean, the R\'enyi mean, and the Log-Euclidean mean, for positive (semi)definite matrices. For example,
\begin{align*}
&\mathcal{A}_{\alpha,p}(A,B):=(\alpha A^p+(1-\alpha)B^p)^{1/p},
\ \mbox{the quasi-weighted arithmetic mean}, \\
&G_{\alpha,p}(A,B):=(A^p\#_\alpha B^p)^{1/p},
\ \mbox{the quasi-weighted geometric mean}, \\
&R_{\alpha,p}(A,B):=\bigl(B^{{\frac{1-\alpha}{2}}p}A^{\alpha p}B^{{\frac{1-\alpha}{2}}p}\bigr)^{1/p},
\ \mbox{the R\'enyi mean}, \\
&LE_\alpha(A,B):=\exp(\alpha\log A+(1-\alpha)\log B),
\ \mbox{the weighted Log-Euclidean mean},
\end{align*}
where $\alpha\in(0,1)$ (or all $\alpha>0$) is the weight parameter and $p>0$ is the parameter of the quasi extension.
For the quasi versions of these weighted matrix means we consider inequalities in different types of orders such as the Loewner order $\le$, the chaotic $\le_{\mathrm{ch}}$ (i.e., $\log X\le \log Y$), the near order $\le_{\mathrm{ne}}$ (i.e., $X\#Y^{-1}\le I$), the entrywise eigenvalue order $\le_\lambda$, the log-majorization $\prec_{\log}$, the weak majorization $\prec_w$, and the order under trace (i.e., $\mathrm{Tr}\,X\le\mathrm{Tr}\,Y$), whose order relations weaken in this writing order.
Our objective is to pursue under which condition of the parameters $\alpha,p,q$ the inequality $\mathcal{M}_{\alpha,p}(A,B)\triangleleft\mathcal{N}_{\alpha,q}(A,B)$ holds for all positive (semi)definite matrices $A,B$, for any pair $(\mathcal{M}_{\alpha,p},\mathcal{N}_{\alpha,q})$ from the above quasi-weighted means and for any order $\triangleleft$ mentioned above.
For instance, it is shown that $\mathcal{A}_{\alpha,p}(A,B)\le\mathcal{A}_{\alpha,q}(A,B)$ holds for all $A,B\ge0$ if and only if $p=q$ or $1\le p<q$ or $1/2\le p<1\le q$, whereas, for any other weaker order $\triangleleft$, $\mathcal{A}_{\alpha,p}(A,B)\triangleleft\mathcal{A}_{\alpha,q}(A,B)$ holds for all $A,B\ge0$ if and only if $p\le q$. When $(\mathcal{M}_{\alpha,p},\mathcal{N}_{\alpha,q})$ is a pair from the quasi-weighted geometric type means, we are mostly interested in the condition of $\alpha,p,q$ under which the log-majorization $\mathcal{M}_{\alpha,p}(A,B)\prec_{\log}\mathcal{N}_{\alpha,q}(A,B)$ holds and whether $AB=BA$ follows from the equality case $\mathrm{Tr}\,\mathcal{M}_{\alpha,p}(A,B)=\mathrm{Tr}\,\mathcal{N}_{\alpha,q}(A,B)$ in this situation.\end{ilasabstract}
     \hypertarget{down0004}{}\begin{ilasabstract}
   \talktitle{Adaptive randomized pivoting}
    
    \textbf{Daniel Kressner}, \info{08:00\textrm{--}09:00 @ SYS Hall (June 25, Wednesday)} \hfill \hyperlink{up0004}{$\Uparrow$}
    
    \mtskip
    Finding good subsets of row and column indices, often called pivots, is a ubiquitous task in applied and numerical linear algebra. One of its most famous appearances is arguably in Gaussian elimination for solving linear systems, where a good choice of pivots is crucial for numerical stability. This talk will focus on pivoting in the context of low-dimensional approximation, including column subset selection, discrete empirical interpolation, and various interpolative decompositions, such as the CUR and Cholesky/Nystrom approximations. In all of these cases, a greedy choice of pivots usually works well but there are well-known counterexamples where such a choice leads to poor results. We present a new randomized pivot selection strategy that avoids such unfavorable worst-case performance by using adaptivity in two senses: It adapts to information on the range / co-range of a matrix, and the sampling distribution is updated after each pivot selection. Adaptive randomized pivoting enjoys error guarantees that match, in expectation, the best known existence results. At the same time, it is   simpler and usually cheaper than volume-based techniques, which achieve similar guarantees through volume sampling or iterative local volume maximization. We will illustrate several applications of adaptive randomized pivoting and discuss derandomized variants. This talk is based on joint work with Alice Cortinovis, University of Pisa.\end{ilasabstract}
     \hypertarget{down0008}{}\begin{ilasabstract}
   \talktitle{nan}
    
    \textbf{Ren-Cang Li}, \info{10:00\textrm{--}11:00 @ SYS Hall (June 27, Friday)} \hfill \hyperlink{up0008}{$\Uparrow$}
    
    \mtskip
    nan\end{ilasabstract}
     \hypertarget{down0007}{}\begin{ilasabstract}
   \talktitle{nan}
    
    \textbf{Karen Meagher}, \info{09:00\textrm{--}10:00 @ SYS Hall (June 26, Thursday)} \hfill \hyperlink{up0007}{$\Uparrow$}
    
    \mtskip
    nan\end{ilasabstract}
     \hypertarget{down0002}{}\begin{ilasabstract}
   \talktitle{nan}
    
    \textbf{Polona Oblak}, \info{08:00\textrm{--}09:00 @ SYS Hall (June 24, Tuesday)} \hfill \hyperlink{up0002}{$\Uparrow$}
    
    \mtskip
    nan\end{ilasabstract}
     \hypertarget{down0000}{}\begin{ilasabstract}
   \talktitle{A ramble through mathematics relevant for quantum theory: \\  A personal perspective
}
    
    \textbf{Karol Życzkowski}, \info{08:30\textrm{--}09:30 @ SYS Hall (June 23, Monday)} \hfill \hyperlink{up0000}{$\Uparrow$}
    
    \mtskip
    Which branches of mathematics are the most important for a physicist working in quantum theory? 
How to foster an interdisciplinary collaboration of a theoretical physicists and a mathematician?
Which problems motivated by quantum theory can be inspiring for the mathematical community?  
Basing on personal experience concerning research related to linear algebra, operator theory, matrix analysis, random matrices, geometry of convex sets, group theory and combinatorics, I will argue that such a collaboration is possible and can be fruitful.
\end{ilasabstract}
    \newpage

\section{Abstracts of Mini-symposium Talks}
 \hypertarget{down0179}{}\begin{ilasabstract}
   \talktitle{Recent developments of switching methods for the construction of cospectral graphs}
    
    \textbf{Aida Abiad}, \info{14:30\textrm{--}15:00 @ SC1005 (June 24, Tuesday)} \hfill \hyperlink{up0179}{$\Uparrow$}
    
    (in {\color{mstitle}MS17: Graphs and matrices in honor of Leslie Hogben's retirement})
        
        \mtskip
    Switching is an operation on a graph that does not change the spectrum of the adjacency matrix, thus producing cospectral graphs. An important activity in the field of spectral graph theory is the characterization of graphs by their spectrum. Hence, switching provides a tool for disproving the existence of such a characterization. 

In this talk we will overview recent progress on switching methods for the construction of cospectral graphs. Work by Wang and Xu (2010) suggests that most cospectral graphs with cospectral complements can be constructed using regular orthogonal matrices of level 2, which has relevance for Haemers' conjecture. In this direction, we will present two new switching methods based on regular orthogonal matrices of level 2. We will also show a general framework for counting the number of graphs that have a non-isomorphic cospectral graph through any of the existing switching methods for the adjacency matrix, expanding on the work by Haemers and Spence (2004).  

This is joint work with Nils van de Berg and Robin Simoens.
\end{ilasabstract}
     \hypertarget{down0059}{}\begin{ilasabstract}
   \talktitle{A low-complexity LSTM network to realize multibeam beamforming
}
    
    \textbf{Hansaka Aluvihare}, \info{14:30\textrm{--}15:00 @ SC1003 (June 23, Monday)} \hfill \hyperlink{up0059}{$\Uparrow$}
    
    (in {\color{mstitle}MS26: Utilizing structure to achieve low-complexity algorithms for data science, engineering, and physics})
        
        \mtskip
    Even large amounts of data can be efficiently realized by imposing structures into neural networks through structured weight matrices. These structured weight matrices could be trained to realize phase shifts for specific beams in multi-beam beamforming, along with input and output vectors made up of time-domain signals. In our previous research, we showed that wideband multi-beam beamformers using true-time-delays (TTDs) can be represented by delay Vandermonde matrices (DVM). We use a frequency-domain transformation to explicitly express the TTD-based time delay data in terms of the elements of the DVM-structured weight matrices. Building upon these weight matrices, we introduce a novel low-complexity neural network LSTM architecture for realizing wideband multi-beam beamformers. The proposed structured LSTM network reduces the computational complexity for realizing wideband multi-beam beamformers from $\mathcal{O}(N^2L)$ to $\mathcal{O}(N^sL)$, where $1 < s < 2$ and $L$ is the number of layers.
\end{ilasabstract}
     \hypertarget{down0170}{}\begin{ilasabstract}
   \talktitle{Doubly stochastic matrices and graphs}
    
    \textbf{Enide Andrade}, \info{14:00\textrm{--}14:30 @ SC1001 (June 24, Tuesday)} \hfill \hyperlink{up0170}{$\Uparrow$}
    
    (in {\color{mstitle}MS21: Linear algebra techniques in graph theory})
        
        \mtskip
    In this talk we present a study about inverses of modified Laplacian matrices; the modification is by adding the
identity matrix which gives a positive definite matrix. We investigate the relationship between
the underlying graph and the properties of this inverse. 

%Some related questions are also
%. The questions and objects are of interest to the matrix theory community and may also
%be of interest in applications involving Laplacians.



\end{ilasabstract}
     \hypertarget{down0338}{}\begin{ilasabstract}
   \talktitle{Laplacian eigenvalues of weighted threshold graphs}
    
    \textbf{Milica Anđelić}, \info{15:00\textrm{--}15:30 @ SC2001 (June 26, Thursday)} \hfill \hyperlink{up0338}{$\Uparrow$}
    
    (in {\color{mstitle}MS21: Linear algebra techniques in graph theory})
        
        \mtskip
    We provide a closed  formula to compute the Laplacian spectrum of weighted threshold graphs. We also  show that their Laplacian eigenvalues are the parts of the conjugate partition of the associated   weighted Ferrers diagrams. \\

({\it This is a joint work with Zoran Stani\'c, Faculty of Mathematics, University of Belgrade, Serbia.})\end{ilasabstract}
     \hypertarget{down0210}{}\begin{ilasabstract}
   \talktitle{Identifying and estimating dynamical covariance matrices with hierarchical rank structure}
    
    \textbf{Robin Armstrong}, \info{16:30\textrm{--}17:00 @ SC0014 (June 24, Tuesday)} \hfill \hyperlink{up0210}{$\Uparrow$}
    
    (in {\color{mstitle}MS6: Model reduction})
        
        \mtskip
    Many algorithms in data assimilation and model reduction rely on sample-based estimates for covariance matrices associated with the trajectory of a high-dimensional dynamical system. The number of available samples is often far less than the dimension of the underlying state space, making it necessary to impose a regularizing structural assumption such as spatial localization. This talk will examine the use of hierarchical rank structure as a regularizing assumption for high-dimensional covariance estimation. Whereas spatial localization assumes that long-range correlations are near-zero, hierarchical rank structure corresponds to the situation where long-range correlations vary more smoothly than short-range ones. We will first examine, from theoretical and empirical circumstances, the conditions under which this assumption is appropriate. We will then present algorithms and experiments which show how to estimate a high-dimensional covariance matrix from limited samples by imposing hierarchical rank structure. Covariance matrices associated with turbulent fluid dynamics, numerical weather prediction models, and Lorenz-type systems will serve as illustrative examples throughout.
\end{ilasabstract}
     \hypertarget{down0158}{}\begin{ilasabstract}
   \talktitle{nan}
    
    \textbf{Athul Augustine}, \info{14:00\textrm{--}14:30 @ SC0009 (June 24, Tuesday)} \hfill \hyperlink{up0158}{$\Uparrow$}
    
    (in {\color{mstitle}MS29: Matrix functions and related topics})
        
        \mtskip
    nan\end{ilasabstract}
     \hypertarget{down0344}{}\begin{ilasabstract}
   \talktitle{nan}
    
    \textbf{Oleg Balabanov}, \info{14:00\textrm{--}14:30 @ SC3001 (June 26, Thursday)} \hfill \hyperlink{up0344}{$\Uparrow$}
    
    (in {\color{mstitle}MS16: Approximations and errors in Krylov-based solvers})
        
        \mtskip
    nan\end{ilasabstract}
     \hypertarget{down0381}{}\begin{ilasabstract}
   \talktitle{nan}
    
    \textbf{Anirban Banerjee}, \info{17:00\textrm{--}17:30 @ SC2001 (June 26, Thursday)} \hfill \hyperlink{up0381}{$\Uparrow$}
    
    (in {\color{mstitle}MS21: Linear algebra techniques in graph theory})
        
        \mtskip
    nan\end{ilasabstract}
     \hypertarget{down0336}{}\begin{ilasabstract}
   \talktitle{nan}
    
    \textbf{Sasmita Barik}, \info{14:00\textrm{--}14:30 @ SC2001 (June 26, Thursday)} \hfill \hyperlink{up0336}{$\Uparrow$}
    
    (in {\color{mstitle}MS21: Linear algebra techniques in graph theory})
        
        \mtskip
    nan\end{ilasabstract}
     \hypertarget{down0211}{}\begin{ilasabstract}
   \talktitle{Gaussian process regression for the identification of model dynamics}
    
    \textbf{Christopher Beattie}, \info{17:00\textrm{--}17:30 @ SC0014 (June 24, Tuesday)} \hfill \hyperlink{up0211}{$\Uparrow$}
    
    (in {\color{mstitle}MS6: Model reduction})
        
        \mtskip
    Standard approximation strategies for Gaussian Process Regression may be linked with optimal spline approximation within the framework of a reproducing kernel Hilbert space determined by a given covariance function associated with a Bayesian prior.   We extend this framework to obtain optimal rational approximants (modeling rational transfer functions) utilizing intrinsic Gaussian processes and Bayesian priors that allow for data-driven modeling of dynamical systems with the management of nonstationary temporal dependence, drift, and observation errors.  I will review elements of a basic framework for Gaussian Process Regression and a version of ``rational kriging'' recently introduced by V. R. Joseph,  connecting this to optimal approximation in reproducing kernel Hilbert spaces with rational kernels, paralleling traditional Gaussian Process Regression approaches.
\end{ilasabstract}
     \hypertarget{down0060}{}\begin{ilasabstract}
   \talktitle{nan}
    
    \textbf{Natalia Bebiano}, \info{15:00\textrm{--}15:30 @ SC1003 (June 23, Monday)} \hfill \hyperlink{up0060}{$\Uparrow$}
    
    (in {\color{mstitle}MS26: Utilizing structure to achieve low-complexity algorithms for data science, engineering, and physics})
        
        \mtskip
    nan\end{ilasabstract}
     \hypertarget{down0365}{}\begin{ilasabstract}
   \talktitle{Computation of an exact and approximate GCRD of several polynomial matrices using generalized Sylvester matrices}
    
    \textbf{Anjali Beniwal}, \info{17:00\textrm{--}17:30 @ SC0014 (June 26, Thursday)} \hfill \hyperlink{up0365}{$\Uparrow$}
    
    (in {\color{mstitle}MS5: Advances in matrix equations: Theory, computations, and applications})
        
        \mtskip
    The computation of an exact and approximate Greatest Common Right Divisor (GCRD) of polynomial matrices is a fundamental problem in control theory and signal processing. While extensive research has been conducted on the Greatest Common Divisors (GCDs) of scalar polynomials, the study of GCRDs in polynomial matrices remains relatively limited. It is important to develop fast and reliable ways to compute both exact and approximate GCRD, especially when dealing with approximate cases where the data is corrupted by noise.\\
We address the problem of computing both exact and approximate GCRDs for a given set of univariate polynomial matrices $B_1(s), \dots, B_t(s)$. We establish a key theoretical result—proving that the rank deficiency of a particular generalized Sylvester matrix associated with $P(s)$, where $P(s)$ is obtained by stacking $B_1(s), \dots, B_t(s)$ one below the other, corresponds to the degree of the determinant of their GCRD. This equivalence enables us to develop efficient algorithms for computing an exact GCRD using \textit{Effectively Eliminating QR} (EEQR) decomposition and $SVD$ of that particular generalized Sylvester matrix. $SVD$ can handle noise-corrupted data, making it particularly effective for computing an approximate GCRD. This approach leverages the numerical rank of that particular generalized Sylvester matrix to estimate the degree of an approximate GCRD. Both algorithms are simple to develop, easy to understand, and convenient to implement, ensuring scalability.\\
We validate our results through various numerical examples, demonstrating their effectiveness. These findings have significant implications for applications requiring polynomial matrix factorizations and simplifications in the presence of noise and uncertainty.\end{ilasabstract}
     \hypertarget{down0380}{}\begin{ilasabstract}
   \talktitle{nan}
    
    \textbf{Mushtaq Ahmad Bhat}, \info{16:30\textrm{--}17:00 @ SC2001 (June 26, Thursday)} \hfill \hyperlink{up0380}{$\Uparrow$}
    
    (in {\color{mstitle}MS21: Linear algebra techniques in graph theory})
        
        \mtskip
    nan\end{ilasabstract}
     \hypertarget{down0207}{}\begin{ilasabstract}
   \talktitle{nan}
    
    \textbf{Saptak Bhattacharya}, \info{17:00\textrm{--}17:30 @ SC0012 (June 24, Tuesday)} \hfill \hyperlink{up0207}{$\Uparrow$}
    
    (in {\color{mstitle}MS7: Linear algebra and quantum information science})
        
        \mtskip
    nan\end{ilasabstract}
     \hypertarget{down0201}{}\begin{ilasabstract}
   \talktitle{nan}
    
    \textbf{Davide Bianchi}, \info{16:00\textrm{--}16:30 @ SC0009 (June 24, Tuesday)} \hfill \hyperlink{up0201}{$\Uparrow$}
    
    (in {\color{mstitle}MS4: Linear algebra methods for inverse problems and data assimilation})
        
        \mtskip
    nan\end{ilasabstract}
     \hypertarget{down0025}{}\begin{ilasabstract}
   \talktitle{Decay bounds for inverses of banded matrices via quasiseparable structure}
    
    \textbf{Paola Boito}, \info{11:00\textrm{--}11:30 @ SC1003 (June 23, Monday)} \hfill \hyperlink{up0025}{$\Uparrow$}
    
    (in {\color{mstitle}MS26: Utilizing structure to achieve low-complexity algorithms for data science, engineering, and physics})
        
        \mtskip
    \begin{bibunit}
        A well-known result in matrix theory states that, under suitable hypotheses, the inverse of a banded matrix $A$ exhibits an exponential off-diagonal decay behavior. In other words, there exist constants $K>0$ and $0<\xi<1$, independent of matrix size, such that 
$$
[A^{-1}]_{ij}\leq K \xi^{|i-j|}.
$$
Several versions of such bounds are available in the literature. Most of them rely on polynomial approximation of the function $x\rightarrow 1/x$ on a convex subset of $\mathbb{C}$ containing the spectrum of $A$; see for instance the seminal work by Demko, Moss and Smith \cite{DMS84}.

Here we take a different approach, which exploits the quasiseparable structure of $A$ and $A^{-1}$. Based on recently proposed inversion algorithms for banded matrices \cite{BE23}, we develop new decay bounds for inverses of one-sided and two-sided banded matrices, under a hypothesis of strong diagonal dominance. Our bounds are easily computable, do not require spectral information on $A$ and can be advantageous for symmetric indefinite or nonsymmetric matrices.

This is joint work with Yuli Eidelman (Tel-Aviv University). 


\begin{thebibliography}{99}
\bibitem{BE23}
P.~Boito and Y.~Eidelman, Computation of quasiseparable representations of Green matrices. Linear Algebra and its Applications, in press (available online 6 May 2024).
\bibitem{DMS84}
S.~Demko, W.~F.~Moss, and P.~W.~Smith,
Decay rates for inverses of band matrices. Mathematics of Computation 43.168 (1984), 491-499.
\end{thebibliography}


        \end{bibunit}
        \end{ilasabstract}
     \hypertarget{down0215}{}\begin{ilasabstract}
   \talktitle{The distance to bounded realness revisited}
    
    \textbf{Shreemayee Bora}, \info{17:00\textrm{--}17:30 @ SC1001 (June 24, Tuesday)} \hfill \hyperlink{up0215}{$\Uparrow$}
    
    (in {\color{mstitle}MS14: Pencils, polynomial, and rational matrices})
        
        \mtskip
    The spectrum of a Hamiltonian matrix is symmetric with respect to the imaginary axis. Hence its eigenvalues occur in pairs $(\lambda, -\bar{\lambda})$ if the matrix is complex and in quadruples, $(\lambda, \bar{\lambda}, -\bar{\lambda},-\lambda)$ if the matrix is real. This pairing breaks down when the eigenvalues are purely imaginary and this can lead to numerical challenges in computational methods used in optimal control.  The distance from a given Hamiltonian matrix $H$ to a nearest Hamiltonian matrix $H + \Delta H$ such that any further arbitrarily small Hamiltonian perturbation to $H + \Delta H$ generically removes all its purely imaginary eigenvalues is called the \emph{distance to bounded realness}. Algorithms for finding an upper bound of this distance have been obtained in the literature. In this talk we will present upper and lower bounds on the distance to \emph{bounded realness} which are often seen to be tight in numerical experiments. In fact in many cases the bounds are seen to be equal. We identify conditions under which the equality holds. In particular, we show that our algorithm computes the distance if the Hamiltonian matrix $H$ has only purely imaginary eigenvalues with the ones of positive type being separated from those of negative type. The key to obtaining the results is to convert the Hamiltonian matrix eigenvalue problem into that of an eigenvalue problem associated with a closely related Hermitian matrix pencil.

This is joint work with Kannan R. of the Department of Mathematics, IIT Guwahati. 
\end{ilasabstract}
     \hypertarget{down0222}{}\begin{ilasabstract}
   \talktitle{On the tree cover number and the positive semidefinite maximum nullity of a graph}
    
    \textbf{Chassidy Bozeman}, \info{16:30\textrm{--}17:00 @ SC1005 (June 24, Tuesday)} \hfill \hyperlink{up0222}{$\Uparrow$}
    
    (in {\color{mstitle}MS17: Graphs and matrices in honor of Leslie Hogben's retirement})
        
        \mtskip
    Let $G=(V,E)$ be  a simple graph. A tree cover of $G$ is a collection of  vertex-disjoint simple trees occurring as induced subgraphs of $G$ that together cover all the vertices of $G$. The tree cover number of $G$, denoted $T(G)$, is the minimum cardinality of a tree cover. We give a characterization of connected outerplanar graphs whose tree cover number equals the upper bound of $\lceil \frac{n}{2} \rceil$.  We also present results on tree cover number of graph with girth at least 5. In 2011, [Barioli et al., Minimum semidefinite rank of outerplanar graphs and the tree cover number, {\em Elec. J. Lin. Alg.,} 2011] introduced the tree cover number as a tool for studying the maximum nullity of a family of matrices associated with a graph:  Let $\mathcal{S}_+(G)$ denote the set of real positive semidefinite matrices $A=(a_{ij})$ such that for $i\neq j$, $a_{ij}\neq 0$ if $\{i,j\}\in E$ and $a_{ij}=0$ if $\{i,j\}\notin E$. The positive semidefinite maximum  nullity of $G$, denoted $M_+(G),$ is $\max\{\text{null}(A)|A\in \mathcal{S}_+(G)\}.$ It was conjectured in 2011 that $T(G) \le \mathrm{M}_+(G)$ holds for all graphs, and shown that equality holds when $G$ is outerplanar.  Therefore our bounds on $T(G)$ give bounds on $M_+(G)$ for outerplanar graphs. We show that the conjecture $T(G)\leq M_+(G)$ is true for certain other graph families.
\end{ilasabstract}
     \hypertarget{down0032}{}\begin{ilasabstract}
   \talktitle{nan}
    
    \textbf{Jane Breen}, \info{11:30\textrm{--}12:00 @ SC2001 (June 23, Monday)} \hfill \hyperlink{up0032}{$\Uparrow$}
    
    (in {\color{mstitle}MS2: Combinatorial matrix theory})
        
        \mtskip
    nan\end{ilasabstract}
     \hypertarget{down0332}{}\begin{ilasabstract}
   \talktitle{nan}
    
    \textbf{Jane Breen}, \info{14:00\textrm{--}14:30 @ SC1005 (June 26, Thursday)} \hfill \hyperlink{up0332}{$\Uparrow$}
    
    (in {\color{mstitle}MS27: Linear algebra education})
        
        \mtskip
    nan\end{ilasabstract}
     \hypertarget{down0203}{}\begin{ilasabstract}
   \talktitle{Matrix-free stochastic calculation of operator norms without using adjoint---on the way to compute the adjoint mismatch}
    
    \textbf{Jonas Bresch}, \info{17:00\textrm{--}17:30 @ SC0009 (June 24, Tuesday)} \hfill \hyperlink{up0203}{$\Uparrow$}
    
    (in {\color{mstitle}MS4: Linear algebra methods for inverse problems and data assimilation})
        
        \mtskip
    Linear inverse problems are of great interest in the last years.
In this talk, we investigate linear inverse problems where the adjoint of the forward operator is not know exactly.
Therefore, we focus on the problem of computing the norm of an operator (between finite dimensional Hilbert spaces),
more precisely $\|A\|$ respectively $\|A - V\|$,
where only evaluations of the linear map $x \mapsto A x$, 
respectively and $y \mapsto V^*y$ are available 
with restrictive storage assumptions for the proposed algorithm.
We propose stochastic methods of random search type for the maximization of the Rayleigh quotient
respectively Rayleigh-like quotient
and employ exact line search in the random search directions.
Moreover, 
we can show that the proposed algorithms converge to the global maximum 
(the operator norm) almost surely 
and illustrate the performance of the method with numerical experiments.
Furthermore, 
for the latter problem we can prove a convergence rate.
\end{ilasabstract}
     \hypertarget{down0031}{}\begin{ilasabstract}
   \talktitle{(Reverse-)Grassmannian permutation matrices}
    
    \textbf{Richard A. Brualdi}, \info{11:00\textrm{--}11:30 @ SC2001 (June 23, Monday)} \hfill \hyperlink{up0031}{$\Uparrow$}
    
    (in {\color{mstitle}MS2: Combinatorial matrix theory})
        
        \mtskip
    A {\it Grassmannian permutation} $i_1i_2\cdots i_n$  is a permutation of $\{1,2,\ldots,n\}$  with at most one descent $i_{k+1}<i_k$; a {\it reverse-Grassmannian} has at most one ascent $i_{k+1}>i_k$. We consider reverse-Grassmannians ({revG}'s) and their corresponding permutation matrices.The number of such revG's is $2^n-n$ which includes the only permutation $n(n-1)\cdots 1$ without any ascents (the reverse-diagonal matrix or Hankel diagonal matrix $L_n$). An $n\times n$ $(0,1)$-matrix $A$  with total support  (every 1 of $A$ belongs to a permutationn matrix $P\le A$) is a {\it revG-blocker} provided that there does not exist a revG permutation matrix $P\le A$. Such a revG-blocker must contain at least $n$ 0's.
We determine a Frobenius-K\"onig-type theorem for revG-blockers with exactly $n$ 0's. Just as the convex polytope $\Omega_n$ of $n\times n$ doubly stochastic matrices gives continuous analogues of permutation matrices, the convex hull $\Omega_n(\mbox{revG})$ of the $n\times n$ revG's gives continuous analogues of revGs. $\Omega_n(\mbox{revG})$ has the same dimension as $\Omega_n$ since there is a basis of the $n\times n$ permutation matrices consisting of revGs. (This talk is based on ongoing work with Lei Cao.)\end{ilasabstract}
     \hypertarget{down0360}{}\begin{ilasabstract}
   \talktitle{On entanglement-breaking quantum channels}
    
    \textbf{Ngoc Muoi Bui}, \info{16:30\textrm{--}17:00 @ SC0012 (June 26, Thursday)} \hfill \hyperlink{up0360}{$\Uparrow$}
    
    (in {\color{mstitle}MS7: Linear algebra and quantum information science})
        
        \mtskip
    Entanglement breaking (EB) channel is a completely positive and trace-preserving linear operator that disrupts the entanglement between the input system with any system. Examples of EB channels include depolarizing channels, quantum-classical channels, etc. We will discuss some characterizations of the class of EB channels for finite- and infinite-dimensional quantum systems. In particular, we show some sufficient conditions for channels to be EB.\end{ilasabstract}
     \hypertarget{down0311}{}\begin{ilasabstract}
   \talktitle{The linear algebra of space-time regularization for time-dependent distributed inverse problems}
    
    \textbf{Daniela Calvetti}, \info{13:30\textrm{--}14:00 @ SC0009 (June 26, Thursday)} \hfill \hyperlink{up0311}{$\Uparrow$}
    
    (in {\color{mstitle}MS4: Linear algebra methods for inverse problems and data assimilation})
        
        \mtskip
    In Bayesian dynamic ill-posed inverse problem, the a priori belief about the solution may have different characteristics in the spatial and temporal directions.  This is the case, for example, for a  promoting sparsity in space  and  smoothness in the temporal direction. In this talk we will consider dynamic inverse problems where the prior should promote group sparsity  in the spatial direction and some degree of time continuity in time, and we will show how linear algebraic techniques can be used to design robust and computationally efficient algorithms. 
\end{ilasabstract}
     \hypertarget{down0320}{}\begin{ilasabstract}
   \talktitle{nan}
    
    \textbf{Roberto Canogar}, \info{14:00\textrm{--}14:30 @ SC0014 (June 26, Thursday)} \hfill \hyperlink{up0320}{$\Uparrow$}
    
    (in {\color{mstitle}MS5: Advances in matrix equations: Theory, computations, and applications})
        
        \mtskip
    nan\end{ilasabstract}
     \hypertarget{down0369}{}\begin{ilasabstract}
   \talktitle{Pattern avoiding and pattern forcing $(0,1)$-matrices for some permutation patterns
}
    
    \textbf{Lei Cao}, \info{17:00\textrm{--}17:30 @ SC1001 (June 26, Thursday)} \hfill \hyperlink{up0369}{$\Uparrow$}
    
    (in {\color{mstitle}MS25: Enumerative/algebraic combinatorics and matrices})
        
        \mtskip
    Let $A = [a_{ij}]$ be an $n \times n$ $(0,1)$-matrix, $P = [p_{ij}]$ an $n \times n$ permutation matrix, and $Q$ a $k \times k$ permutation matrix with $k \leq n$. We write $P \leq A$ if $p_{ij} \leq a_{ij}$ for all $i, j = 1, 2, \ldots, n$.

\begin{itemize}
    \item $A$ is \emph{$Q$-avoiding} if and only if there does not exist a $k \times k$ submatrix $A_k$ of $A$ such that $Q \leq A_k$ entrywise.

    \item $A$ is \emph{$Q$-permutation avoiding} if and only if there does not exist an $n \times n$ permutation matrix $P \leq A$ such that $Q$ is a $k \times k$ submatrix of $P$.

    \item $A$ is \emph{$Q$-forcing} if and only if every $n \times n$ permutation matrix $P \leq A$ contains $Q$ as a $k \times k$ submatrix.
\end{itemize}



In this presentation, I will discuss recent results on $Q$-avoiding, $Q$-permutation avoiding, and $Q$-forcing $(0,1)$-matrices for certain special permutation patterns $Q$. Specifically, we investigate the minimum number of zeros in $(0,1)$-matrices that avoid or force particular permutation patterns. Additionally, we examine the polytope of 123-avoiding doubly stochastic matrices—that is, the convex hull of all 123-avoiding permutation matrices.
\end{ilasabstract}
     \hypertarget{down0393}{}\begin{ilasabstract}
   \talktitle{nan}
    
    \textbf{Benjamin Carrel}, \info{17:00\textrm{--}17:30 @ SC4011 (June 26, Thursday)} \hfill \hyperlink{up0393}{$\Uparrow$}
    
    (in {\color{mstitle}MS23: Advances in Krylov subspace methods and their applications})
        
        \mtskip
    nan\end{ilasabstract}
     \hypertarget{down0198}{}\begin{ilasabstract}
   \talktitle{Refined inertias of nonnegative patterns with positive off-diagonal entries}
    
    \textbf{Minnie Catral}, \info{16:30\textrm{--}17:00 @ SC0008 (June 24, Tuesday)} \hfill \hyperlink{up0198}{$\Uparrow$}
    
    (in {\color{mstitle}MS24: Nonnegative and related families of matrices})
        
        \mtskip
    \newcommand{\ri}{\operatorname{ri}}
The  {\em refined inertia} of  an $n\times n$ matrix $A$  is the 4-tuple $\ri(A)=(n_+,n_-,n_z,2n_p)$ where $n_+,n_-$ equal the number of eigenvalues of $A$ with positive, negative (respectively) real parts, $n_z$ is the number of eigenvalues of $A$  equal to zero and $2n_p$ is the number of nonzero pure imaginary eigenvalues of $A$ (note that $n_+ + n_-+ n_z+2n_p = n$).  
%For an $n\times n$ sign pattern $\mathcal{A}$, the {\em inertia} of $\A$ is the set $\inertia(\mathcal{A})=\left\{ \inertia(A) \mid A\in\Q(\A)\right\}$ and the {\em refined inertia} of $\A$ is the set $\ri(\mathcal{A})=\left\{ \ri(A) \mid A\in\Q(\A)\right\}$.
We investigate refined inertias of nonnegative patterns with  all off-diagonal entries positive,  $k \in\{0, \dots, n\}$ diagonal entries positive and the remaining $n-k$ diagonal entries $0$. The case $k=n$ correspond to the positive pattern and the case $k=0$ correspond to the hollow positive pattern. For the positive sign pattern, every refined inertia  $(n_+, n_-, n_z, 2n_p)$ with $n_+ \geq 1$  can be realized; for the hollow positive  pattern, every refined inertia with $n_+ \geq 1$ and $n_- \geq 2$ can be realized. For the intermediate nonnegative patterns, that is, for  $k \in\{1, \dots, n-1\}$, we show that for $k \geq 2$,  there is no restriction on $n_-$ for the refined inertia set, but  $n_- \geq 1$ for $k=1$. 

This talk is based on joint work with Adam Berliner, Dale Olesky and Pauline van den Driessche.





\end{ilasabstract}
     \hypertarget{down0231}{}\begin{ilasabstract}
   \talktitle{nan}
    
    \textbf{Eunice Y. S. Chan}, \info{17:00\textrm{--}17:30 @ SC2006 (June 24, Tuesday)} \hfill \hyperlink{up0231}{$\Uparrow$}
    
    (in {\color{mstitle}MS30: Bohemian matrices: Theory, applications, and explorations})
        
        \mtskip
    nan\end{ilasabstract}
     \hypertarget{down0145}{}\begin{ilasabstract}
   \talktitle{Integral inequalities of Kantorovich and Fiedler types for Hadamard products of matrices}
    
    \textbf{Pattrawut Chansangiam}, \info{11:00\textrm{--}11:30 @ SC2006 (June 24, Tuesday)} \hfill \hyperlink{up0145}{$\Uparrow$}
    
    (in {\color{mstitle}MS3: Matrix inequalities with applications})
        
        \mtskip
    The scalar Kantorovich inequality is a reverse weighted arithmetic-harmonic mean
inequality. In matrix case, this inequality is also a reverse version of Fiedler’s inequality. In
this talk, we discuss several Kantorovich and Fiedler types integral inequalities involving
Hadamard products of continuous fields of Hilbert space operators. Kantorovich type inequality
in which the product is replaced by an operator mean is also investigated. Such inequalities
include discrete inequalities as special cases. Moreover, we obtain the monotonicity of certain
maps involving Hadamard products of operators. As special cases, we get some operator versions of Fiedler matrix inequality.
\end{ilasabstract}
     \hypertarget{down0165}{}\begin{ilasabstract}
   \talktitle{Nonlinear model reduction using machine learning on Grassmann Manifol}
    
    \textbf{Saifon Chaturantabut}, \info{13:30\textrm{--}14:00 @ SC0014 (June 24, Tuesday)} \hfill \hyperlink{up0165}{$\Uparrow$}
    
    (in {\color{mstitle}MS6: Model reduction})
        
        \mtskip
    This work explores a parametric model order reduction approach for nonlinear dynamical systems, integrating machine learning methods on a Grassmann manifold. Distances between parameters are first calculated using a metric defined on the Grassmann manifold of solution subspaces. Then, the K-medoids clustering algorithm is used with these Grassmann distances to categorize the parameters into classes and form a dictionary of local bases. Machine learning techniques based on support vector machines (SVM) and artificial neural networks (ANN) are finally employed to create classifiers that can automatically determine the most appropriate local basis for constructing reduced-order models. Numerical experiments are conducted on nonlinear differential equations, including the parametrized Burger's equation, the Sine-Gordon equation, and a nonlinear flow.
\end{ilasabstract}
     \hypertarget{down0192}{}\begin{ilasabstract}
   \talktitle{nan}
    
    \textbf{Maolin Che}, \info{15:00\textrm{--}15:30 @ SC3001 (June 24, Tuesday)} \hfill \hyperlink{up0192}{$\Uparrow$}
    
    (in {\color{mstitle}MS18: New methods in numerical multilinear algebra})
        
        \mtskip
    nan\end{ilasabstract}
     \hypertarget{down0073}{}\begin{ilasabstract}
   \talktitle{On parameter tuning for spectral clustering: two simple, fast, and effective criteria}
    
    \textbf{Guangliang Chen}, \info{14:00\textrm{--}14:30 @ SC4011 (June 23, Monday)} \hfill \hyperlink{up0073}{$\Uparrow$}
    
    (in {\color{mstitle}MS20: Manifold learning and statistical applications})
        
        \mtskip
    Spectral clustering is a modern, powerful clustering approach with many successful applications. However, it has faced two major challenges -- high computational complexity and parameter tuning. Since its introduction, much effort has been devoted to improving the scalability of spectral clustering, while little research has been conducted on parameter tuning. In this talk, we address the parameter tuning challenge of spectral clustering in a general context. Specifically, we propose two new criteria for tuning the scale parameter used in similarity functions such as Gaussian and cosine. Experiments demonstrate the effectiveness of these tuning techniques. \end{ilasabstract}
     \hypertarget{down0139}{}\begin{ilasabstract}
   \talktitle{Curvature on graphs via distance matrix}
    
    \textbf{Wei-Chia Chen}, \info{11:00\textrm{--}11:30 @ SC1005 (June 24, Tuesday)} \hfill \hyperlink{up0139}{$\Uparrow$}
    
    (in {\color{mstitle}MS15: Graphs and their eigenvalues: Celebrating the work of Fan Chung Graham})
        
        \mtskip
    Recently, Steinerberger introduced a notion of curvature on graphs based on solutions to the linear system $Dx=\mathbf{1},$ where $D$ is the graph distance matrix and $\mathbf{1}$ is the all-one vector. 
They also observed in the Mathematica database that graphs so that $Dx=\mathbf{1}$ has no solutions seem to be rare. 
In this talk, we will focus on how nonnegative solutions to $Dx=\mathbf{1}$ behave when certain graph operations are performed, such as adding an edge between two graphs. We also provide a way to create an infinite family of graphs so that $Dx=\mathbf{1}$ has no solutions. This is joint work with Mao-Pei Tsui.
\end{ilasabstract}
     \hypertarget{down0152}{}\begin{ilasabstract}
   \talktitle{Convergence of Hessian estimator from random samples on a manifold with boundary}
    
    \textbf{Chih-Wei Chen}, \info{11:30\textrm{--}12:00 @ SC4011 (June 24, Tuesday)} \hfill \hyperlink{up0152}{$\Uparrow$}
    
    (in {\color{mstitle}MS20: Manifold learning and statistical applications})
        
        \mtskip
    A common method for estimating the Hessian operator from random samples on a submanifold involves locally fitting a quadratic polynomial. Although widely used, it is unclear if this estimator introduces bias, especially in manifolds with boundaries and nonuniform sampling. We show that this estimator asymptotically converges to the Hessian operator, with nonuniform sampling and curvature effects proving negligible, even near boundaries. Our analysis framework simplifies the intensive computations required for direct analysis. This is join work with Hau-Tieng Wu. 
\end{ilasabstract}
     \hypertarget{down0306}{}\begin{ilasabstract}
   \talktitle{nan}
    
    \textbf{Pengwen Chen}, \info{11:30\textrm{--}12:00 @ SC4011 (June 26, Thursday)} \hfill \hyperlink{up0306}{$\Uparrow$}
    
    (in {\color{mstitle}MS1: Embracing new opportunities in numerical linear algebra})
        
        \mtskip
    nan\end{ilasabstract}
     \hypertarget{down0228}{}\begin{ilasabstract}
   \talktitle{On the maximal spectral radius of digraphs with a prescribed number of arcs
}
    
    \textbf{Yen-Jen Cheng}, \info{17:30\textrm{--}18:00 @ SC2001 (June 24, Tuesday)} \hfill \hyperlink{up0228}{$\Uparrow$}
    
    (in {\color{mstitle}MS25: Enumerative/algebraic combinatorics and matrices})
        
        \mtskip
    The spectral radius of a matrix is the largest magnitude of its eigenvalues. The spectral radius of a graph is the spectral radius of its adjacency matrix. It captures important structural information about the graph and plays a key role in spectral graph theory. In this talk, I will introduce a new approach to find upper bounds of the spectral radius of a nonnegative matrix, and use it to
identify the unique digraph whose spectral radius is
maximum among all digraphs with a prescribed number
of arcs. This result resolves a problem 
independently posed by R. Brualdi and A. Hoffman, as well as F.
Friedland, back in 1985.
\end{ilasabstract}
     \hypertarget{down0268}{}\begin{ilasabstract}
   \talktitle{nan}
    
    \textbf{Chun-Yueh Chiang}, \info{10:30\textrm{--}11:00 @ SC3001 (June 25, Wednesday)} \hfill \hyperlink{up0268}{$\Uparrow$}
    
    (in {\color{mstitle}MS11: Structured matrix computations and its applications})
        
        \mtskip
    nan\end{ilasabstract}
     \hypertarget{down0024}{}\begin{ilasabstract}
   \talktitle{Geometric mean of T-positive definite tensors}
    
    \textbf{Hayoung Choi}, \info{12:00\textrm{--}12:30 @ SC1001 (June 23, Monday)} \hfill \hyperlink{up0024}{$\Uparrow$}
    
    (in {\color{mstitle}MS10: Matrix means and related topics})
        
        \mtskip
    In this talk, we generalize the geometric mean of two positive definite matrices to that of third-order tensors using the notion of T-product. Specifically, we define the geometric mean of two T-positive definite tensors and verify several properties that ``mean'' should satisfy including the idempotence and the commutative property, and so on. Moreover, it is shown that the geometric mean is a unique T-positive definite solution of an algebraic Riccati tensor equation and can be expressed as solutions of algebraic Riccati matrix equations. 
\end{ilasabstract}
     \hypertarget{down0034}{}\begin{ilasabstract}
   \talktitle{Density-equalizing map with applications}
    
    \textbf{Gary Choi}, \info{11:00\textrm{--}11:30 @ SC2006 (June 23, Monday)} \hfill \hyperlink{up0034}{$\Uparrow$}
    
    (in {\color{mstitle}MS22: Linear algebra applications in computational geometry})
        
        \mtskip
    We present surface and volumetric mapping methods based on a natural principle of density diffusion. Specifically, we start with a prescribed density distribution in a surface or volumetric domain and then create shape deformations with different regions enlarged or shrunk based on the density gradient. Using the proposed methods, we can easily achieve different mapping effects with controllable area change. Applications to shape registration, morphing, remeshing, medical shape analysis, and data visualization will be presented. 
\end{ilasabstract}
     \hypertarget{down0205}{}\begin{ilasabstract}
   \talktitle{nan}
    
    \textbf{Man-Duen Choi}, \info{16:00\textrm{--}16:30 @ SC0012 (June 24, Tuesday)} \hfill \hyperlink{up0205}{$\Uparrow$}
    
    (in {\color{mstitle}MS7: Linear algebra and quantum information science})
        
        \mtskip
    nan\end{ilasabstract}
     \hypertarget{down0335}{}\begin{ilasabstract}
   \talktitle{Two invariants of distance matrices of trees, in a unified framework}
    
    \textbf{Projesh Nath Choudhury}, \info{13:30\textrm{--}14:00 @ SC2001 (June 26, Thursday)} \hfill \hyperlink{up0335}{$\Uparrow$}
    
    (in {\color{mstitle}MS21: Linear algebra techniques in graph theory})
        
        \mtskip
    In 1971, Graham and Pollak showed that if $D_T$ is the distance matrix of a tree $T$ on $n$ nodes, then $\det(D_T)$ depends only on $n$, not $T$. This independence from the tree structure has been verified for many different variants of weighted bi-directed trees. In my talk (over an arbitrary commutative ring):
\begin{enumerate}
 \item I will present a general setting which strictly subsumes every known variant, and where we show that $\det(D_T)$ - as well as another graph invariant, the cofactor-sum - depends only on the edge-data, not the tree-structure.
 \item More generally - even in the original unweighted setting - we strengthen the state-of-the-art, by computing the minors of $D_T$ where one removes rows and columns indexed by equal-sized sets of pendant nodes. (In fact, we go beyond pendant nodes.)
 \item We explain why our result is the ``most general possible'', in that allowing greater freedom in the parameters leads to dependence on the tree-structure.
\end{enumerate}
We will discuss related results for arbitrary strongly connected graphs, including a third, novel invariant. If time permits, a formula for  $D_T^{-1}$ will be presented for trees $T$, whose special case answers an open problem of Bapat-Lal-Pati (Linear Alg. Appl. 2006), and which extends to our general setting a result of Graham-Lovasz (Advances in Math. 1978). (Joint with Apoorva Khare.)\end{ilasabstract}
     \hypertarget{down0304}{}\begin{ilasabstract}
   \talktitle{nan}
    
    \textbf{Eric King-wah Chu}, \info{10:30\textrm{--}11:00 @ SC4011 (June 26, Thursday)} \hfill \hyperlink{up0304}{$\Uparrow$}
    
    (in {\color{mstitle}MS1: Embracing new opportunities in numerical linear algebra})
        
        \mtskip
    nan\end{ilasabstract}
     \hypertarget{down0042}{}\begin{ilasabstract}
   \talktitle{A framework for exploring the conformational landscape of cryo-EM using energy-aware pathfinding algorithm}
    
    \textbf{Szu-Chi Chung}, \info{12:00\textrm{--}12:30 @ SC4011 (June 23, Monday)} \hfill \hyperlink{up0042}{$\Uparrow$}
    
    (in {\color{mstitle}MS20: Manifold learning and statistical applications})
        
        \mtskip
    Cryo-electron microscopy (cryo-EM) is a powerful technique for investigating macromolecular structures and holds great promise for uncovering kinetically preferred transition sequences between conformational states. While such transitions are often explored using two-dimensional energy landscapes, the intrinsic complexity of biomolecular conformations frequently renders low-dimensional representations insufficient. Recent advances in reconstruction models have enabled the characterization of structural heterogeneity from cryo-EM images through high-dimensional latent spaces. However, constructing conformational landscapes in these spaces and identifying preferred transition pathways remain major challenges. 

In this study, we propose a novel framework for identifying preferred trajectories within high-dimensional conformational landscapes. Our method formulates the problem as a graph-based search for minimum energy paths, where edge weights are computed from local energy estimates derived from density in high-dimensional space. We demonstrate the effectiveness of this approach by accurately identifying transition states in both synthetic and experimental datasets exhibiting continuous conformational changes. To facilitate future research and reproducibility, we provide a modular implementation of our framework at https://github.com/tengyulin/energy\_aware\_pathfinding/.
\end{ilasabstract}
     \hypertarget{down0202}{}\begin{ilasabstract}
   \talktitle{nan}
    
    \textbf{Matthias Chung}, \info{16:30\textrm{--}17:00 @ SC0009 (June 24, Tuesday)} \hfill \hyperlink{up0202}{$\Uparrow$}
    
    (in {\color{mstitle}MS4: Linear algebra methods for inverse problems and data assimilation})
        
        \mtskip
    nan\end{ilasabstract}
     \hypertarget{down0217}{}\begin{ilasabstract}
   \talktitle{Signal processing of spherical data. From real life to mathematical challenges}
    
    \textbf{Antonio Cicone}, \info{16:00\textrm{--}16:30 @ SC1003 (June 24, Tuesday)} \hfill \hyperlink{up0217}{$\Uparrow$}
    
    (in {\color{mstitle}MS19: Explicit and hidden asymptotic structures, GLT Analysis, and applications})
        
        \mtskip
    In signal processing the time-frequency analysis of nonlinear and non-stationary processes, as well as the determination of the unknown number of active sub-signals of a blind-source composite signal are, in general, challenging inverse problem tasks. If we consider data sampled on a sphere, things get even more complicated. This is the reason why just a few techniques have been developed so far to study this kind of data. However, many real-life data are of this nature, like in Geophysics and Physics.

The idea is to extend the Iterative Filtering (IF) algorithm to work on the sphere. IF is a non-stationary signal decomposition method proposed a decade ago [1] to address the problem of extracting time-varying oscillatory components from a non-stationary multicomponent signal. This method proved to be really valuable in many applications, see [2] and references therein, and it was accelerated in what is known as Fast Iterative Filtering (FIF) [3] by leveraging the Toeplitz matrix theory.

In this talk, we introduce the generalization of IF to handle spherical data and show how we can address the question about its convergence [4]. We conclude with some examples of application to geophysical data.

This is joint work with {\em Giovanni Barbarino}, {\em Roberto Cavassi}, {\em Wing S. Li}, {\em Edward J. Timko}, {\em Haomin Zhou}.

[1] L. Lin, Y. Wang, and H. Zhou. ``Iterative filtering as an alternative algorithm for empirical mode decomposition''. Adv. in Adap. Data An., 2009, 1.04, 543-560.

[2] G. Barbarino, A. Cicone. ``Conjectures on spectral properties of ALIF algorithm''. Linear Algebra and its Applications, 2022, 647, 127--152.

[3] A. Cicone, H. Zhou. ``Numerical Analysis for Iterative Filtering with New Efficient Implementations Based on FFT''. Num. Math., 2021, 147 (1), 1--28.

[4] G. Barbarino, R. Cavassi, A. Cicone. Extension and convergence analysis of Iterative Filtering to spherical data. Linear Algebra and its Applications, 2024.

\end{ilasabstract}
     \hypertarget{down0029}{}\begin{ilasabstract}
   \talktitle{Principal eigenvectors and principal ratios in hypergraph Tur\'{a}n problems}
    
    \textbf{Joshua Cooper}, \info{11:30\textrm{--}12:00 @ SC1005 (June 23, Monday)} \hfill \hyperlink{up0029}{$\Uparrow$}
    
    (in {\color{mstitle}MS15: Graphs and their eigenvalues: Celebrating the work of Fan Chung Graham})
        
        \mtskip
    For a general class of hypergraph Tur\'{a}n problems with uniformity $r$, we investigate the principal eigenvector for the $p$-spectral radius of its extremal graphs, showing in a strong sense that these eigenvectors have equal weight on each vertex (equivalently, showing that the principal ratio is close to $1$). Our result is sharp for the conjectural extremizers of the Tur\'{a}n tetrahedron problem, and for some other problems in which the extremizers are well understood; it is unclear whether it always sharp. We establish a result which may have also have independent interest, proving a lower bound on the spectral radius depending on the degrees of the graph. The case $1 < p < r$ of our results leads to some subtleties connected to Nikiforov's notion of $k$-tightness. We raise a conjecture about these issues, and provide some preliminary evidence for our conjecture.
\end{ilasabstract}
     \hypertarget{down0379}{}\begin{ilasabstract}
   \talktitle{Determinants of Steiner distance hypermatrices}
    
    \textbf{Joshua Cooper}, \info{16:00\textrm{--}16:30 @ SC2001 (June 26, Thursday)} \hfill \hyperlink{up0379}{$\Uparrow$}
    
    (in {\color{mstitle}MS21: Linear algebra techniques in graph theory})
        
        \mtskip
    Generalizing work from the 1970s on the determinants of distance hypermatrices of trees, we consider the hyperdeterminants of order-$k$ Steiner distance hypermatrices of trees $T$ on $n$ vertices.  We show that this hypermatrix has the same diagonalization as a $k$-form for any $T$ on $n$ vertices, generalizing of a result of Graham-Lov\'{a}sz, implying a tensor version of ``conditional negative definiteness'', providing new proofs of previous results of the authors and Tauscheck, and resolving the conjecture that these hyperdeterminants depend only on $k$ and $n$ -- as Graham-Pollak showed for $k=2$.
\end{ilasabstract}
     \hypertarget{down0204}{}\begin{ilasabstract}
   \talktitle{nan}
    
    \textbf{Tiangang Cui}, \info{17:30\textrm{--}18:00 @ SC0009 (June 24, Tuesday)} \hfill \hyperlink{up0204}{$\Uparrow$}
    
    (in {\color{mstitle}MS4: Linear algebra methods for inverse problems and data assimilation})
        
        \mtskip
    nan\end{ilasabstract}
     \hypertarget{down0245}{}\begin{ilasabstract}
   \talktitle{Large similarities between the numerical range of an operator and
of certain generalized inverses}
    
    \textbf{Dragana Cvetkovic Ilic}, \info{11:00\textrm{--}11:30 @ SC0009 (June 25, Wednesday)} \hfill \hyperlink{up0245}{$\Uparrow$}
    
    (in {\color{mstitle}MS33: Norms of matrices, numerical range, applications of functional analysis to matrix theory})
        
        \mtskip
    In this talk we will consider the different properties of the numerical range for a bounded linear operator from the point of view of its similarities  to the numerical range  of certain generalized inverses such as the Moore-Penrose inverse, Drazin inverse, DMP-inverse, MDP-inverse and CMP-inverse.  We will discuss  the question whether the numerical ranges of an operator and its certain generalized inverse simultaneously contain the origin as well as whether this is true in the case of the closure, boundary, extreme points and sharp points of the corresponding numerical ranges.  To illustrate our results we will exhibit some numerical examples.
\end{ilasabstract}
     \hypertarget{down0200}{}\begin{ilasabstract}
   \talktitle{Near-derangements and polytopes}
    
    \textbf{Geir Dahl}, \info{17:30\textrm{--}18:00 @ SC0008 (June 24, Tuesday)} \hfill \hyperlink{up0200}{$\Uparrow$}
    
    (in {\color{mstitle}MS24: Nonnegative and related families of matrices})
        
        \mtskip
    A derangement is a permutation with no fixed point. This talk deals with  {\em near-derangements}, defined as permutations with at most one fixed point. We present some properties of  the corresponding set of permutation matrices $\mathcal{P}^{(\le 1)}_n$. 
  Also, we briefly discuss  the polytope determined by $\mathcal{P}^{(\le 1)}_n$ and the related polytope $\mathcal{P}^*$ of  all $n\times n$  doubly stochastic matrices with trace at most 1.
  In particular, all its  extreme points of $\mathcal{P}^*$ are determined. 
This is joint work with Richard A. Brualdi.
\end{ilasabstract}
     \hypertarget{down0121}{}\begin{ilasabstract}
   \talktitle{nan}
    
    \textbf{Sujit Sakharam Damase}, \info{11:00\textrm{--}11:30 @ SC0008 (June 24, Tuesday)} \hfill \hyperlink{up0121}{$\Uparrow$}
    
    (in {\color{mstitle}MS9: Total positivity})
        
        \mtskip
    nan\end{ilasabstract}
     \hypertarget{down0259}{}\begin{ilasabstract}
   \talktitle{A linear algebra online course experience at UC3M: development and teaching}
    
    \textbf{Fernando De Terán}, \info{10:30\textrm{--}11:00 @ SC1005 (June 25, Wednesday)} \hfill \hyperlink{up0259}{$\Uparrow$}
    
    (in {\color{mstitle}MS27: Linear algebra education})
        
        \mtskip
    I will review my experience with a SPOC (Small Private Online Course) with flipped classroom that we developed to teach basic Linear Algebra at UC3M. We used this course for 6 academic years in a first-year Linear Algebra course for engineers.\end{ilasabstract}
     \hypertarget{down0319}{}\begin{ilasabstract}
   \talktitle{The uniqueness of solution of systems of generalized Sylvester and $\star$-Sylvester equations}
    
    \textbf{Fernando De Terán}, \info{13:30\textrm{--}14:00 @ SC0014 (June 26, Thursday)} \hfill \hyperlink{up0319}{$\Uparrow$}
    
    (in {\color{mstitle}MS5: Advances in matrix equations: Theory, computations, and applications})
        
        \mtskip
    The {\em generalized Sylvester equation}
$$
AXB-CXD=E
$$
has been a subject of interest since, at least, the early 20th century. More recently, the {\em$\star$-generalized Sylvester equation}
$$
AXB-CX^\star D=E,
$$
with $\star$ being either the transpose or the conjugate transpose, has attracted some attention within the linear algebra community.

In this talk, we provide necessary and sufficient conditions for the uniqueness of solution of homogeneous systems of generalized Sylvester and $\star$-Sylvester equations, namely
$$
A_iX_{\alpha_i}^{s_i}B_i-C_iX_{\beta_i}^{t_i}D_i=0,\qquad i=1,\hdots,r,
$$
with $s_i,t_i\in\{1,\star\}$. 

We focus on the case where the system has the same number of equations and unknowns (namely, $r$), and where all coefficient matrices (and unknowns) are square and with the same size.\end{ilasabstract}
     \hypertarget{down0177}{}\begin{ilasabstract}
   \talktitle{I love the triangle number!}
    
    \textbf{Louis Deaett}, \info{13:30\textrm{--}14:00 @ SC1005 (June 24, Tuesday)} \hfill \hyperlink{up0177}{$\Uparrow$}
    
    (in {\color{mstitle}MS17: Graphs and matrices in honor of Leslie Hogben's retirement})
        
        \mtskip
    A \emph{zero-nonzero pattern} is a matrix with entries from the set $\{0,*\}$.  The \emph{triangle number} of the pattern is the largest $k$ such that some $k\times k$ submatrix is (up to permutation) triangular with only $*$ entries on its diagonal.


A \emph{realization} of the pattern over some field is a matrix that can be obtained by replacing each $*$ entry of the pattern with a nonzero value from that field.
The smallest rank of such a realization is the \emph{minimum rank} of the pattern (over that field).  The triangle number provides a natural and simple lower bound on this value.

This talk explores some different combinatorial angles on the triangle number.  For example, we review how the triangle number is connected with the zero forcing number, both for simple graphs and directed graphs.  In fact, the bound it provides on the minimum rank of a pattern is closely related to the bound given by the zero forcing number on the maximum nullity of a graph; we see how, in certain special cases, the relationship is especially strong.  This also motivates the question of Nordhaus-Gaddum type bounds for the triangle number, and we present some recent results on this front.  We also examine the triangle number through the lens of lattice theory and matroid theory, where it provides a lower bound on a related class of minimum rank problems for matroids, one which can provide insights into the original minimum rank problem for matrix patterns.
\end{ilasabstract}
     \hypertarget{down0043}{}\begin{ilasabstract}
   \talktitle{Computations with high relative accuracy for the collocation matrices of q-Jacobi polynomials}
    
    \textbf{Jorge Delgado}, \info{14:00\textrm{--}14:30 @ SC0008 (June 23, Monday)} \hfill \hyperlink{up0043}{$\Uparrow$}
    
    (in {\color{mstitle}MS9: Total positivity})
        
        \mtskip
    In this talk the bidiagonal decomposition of the Collocation Matrices of q-Jacobi Polynomials will be presented.
In addition, it will be shown that this bidiagonal decomposition can be constructed with high relative accuracy (HRA)
in many cases.
Then, for these cases, the bidiagonal decomposition will be used to solve with HRA the following linear algebra
problems: computation of the inverse, the eigenvalues and the singular values of those collocation matices, and
the solution of some related linear systems of equations.

This is a joint work with Jorge Delgado, Héctor Orera and Juan Manuel Peña.\end{ilasabstract}
     \hypertarget{down0357}{}\begin{ilasabstract}
   \talktitle{nan}
    
    \textbf{Philip Dinenis}, \info{17:00\textrm{--}17:30 @ SC0009 (June 26, Thursday)} \hfill \hyperlink{up0357}{$\Uparrow$}
    
    (in {\color{mstitle}MS4: Linear algebra methods for inverse problems and data assimilation})
        
        \mtskip
    nan\end{ilasabstract}
     \hypertarget{down0254}{}\begin{ilasabstract}
   \talktitle{Operator means and quantum divergences}
    
    \textbf{Trung Hoa Dinh}, \info{11:00\textrm{--}11:30 @ SC1001 (June 25, Wednesday)} \hfill \hyperlink{up0254}{$\Uparrow$}
    
    (in {\color{mstitle}MS10: Matrix means and related topics})
        
        \mtskip
    In this talk, we will explore the use of operator means to construct new quantum divergences. We will also examine the least squares problem in the context of various quantum divergences. Additionally, we will discuss some applications and highlight open questions in this area.
\end{ilasabstract}
     \hypertarget{down0216}{}\begin{ilasabstract}
   \talktitle{Minimal indices through perturbation behavior}
    
    \textbf{Andrii Dmytryshyn}, \info{17:30\textrm{--}18:00 @ SC1001 (June 24, Tuesday)} \hfill \hyperlink{up0216}{$\Uparrow$}
    
    (in {\color{mstitle}MS14: Pencils, polynomial, and rational matrices})
        
        \mtskip
    Computing the complete eigenstructure of matrix pencils is a challenging problem. Small perturbations can change both the eigenvalues with their multiplicities, as well as the minimal indices of a given pencil. Recently, however, perturbation theory was used to compute eigenvalues of singular matrix pencils. In this talk, we investigate how the behavior of a general matrix pencil under small perturbations can help determine its minimal indices.
\end{ilasabstract}
     \hypertarget{down0330}{}\begin{ilasabstract}
   \talktitle{Polynomial and rational matrices with the invariant rational functions and the four sequences of minimal indices prescribed}
    
    \textbf{Froilan M. Dopico}, \info{15:00\textrm{--}15:30 @ SC1003 (June 26, Thursday)} \hfill \hyperlink{up0330}{$\Uparrow$}
    
    (in {\color{mstitle}MS14: Pencils, polynomial, and rational matrices})
        
        \mtskip
    The complete eigenstructure, or structural data, of a rational matrix
$R(s)$ is comprised by its invariant rational functions, both finite
and at infinity, which determine its finite and infinite pole and zero
structures, and by the minimal indices of its left and right null spaces.
These quantities arise in many applications and have been thoroughly
studied in numerous references. However, $R(s)$ has other two fundamental
subspaces which, in contrast, have received much less attention
in the literature. They are its column and row spaces, which also have
their associated minimal indices. This work solves the problems of
finding necessary and sufficient conditions for the existence of rational
matrices in two scenarios: (a) when the invariant rational functions and the minimal indices of the column and row spaces are prescribed,
and (b) when the complete eigenstructure together with the minimal
indices of the column and row spaces are prescribed. The particular,
but extremely important, cases of these problems for polynomial matrices
are solved first and are the main tool for solving the general
problems. The results in this work complete and extend non-trivially
the necessary and sufficient conditions recently developed in the literature
for the existence of polynomial and rational matrices when only
the complete eigenstructure is prescribed.

This is joint work with Itziar Baraga\~na and Silvia Marcaida (Universidad del Pa\'is Vasco UPV/EHU, Spain) and Alicia Roca (Universitat Polit\`ecnica de Val\`encia, Spain).\end{ilasabstract}
     \hypertarget{down0131}{}\begin{ilasabstract}
   \talktitle{Koopman mode decomposition for nonlinear model reduction}
    
    \textbf{Zlatko Drmac}, \info{11:30\textrm{--}12:00 @ SC0014 (June 24, Tuesday)} \hfill \hyperlink{up0131}{$\Uparrow$}
    
    (in {\color{mstitle}MS6: Model reduction})
        
        \mtskip
    The Dynamic Mode Decomposition (DMD) is a popular numerical method
for data driven analysis of nonlinear dynamical systems, with a wide spectrum 
of applications. It can be used for model order reduction, analysis of
latent structures in the dynamics, and e.g. for forecasting and control.  The
theoretical bedrock upon which the more general Extended DMD framework 
is built is the Koopman composition operator.
%
DMD can be described as a data driven Rayleigh-Ritz extraction of spectral information
of a Koopman operator associated with the underlying dynamical system. 
The nonlinear data snapshots are represented using the eigenvectors of the
operator resulting in a modal decomposition KMD (Koopman Mode Decomposition).
This becomes a model order reduction tool that represents the nonlinear dynamics
using selected eigenpairs. It can be used to reveal coherent states and for forecasting. \\
In some cases, a numerically computed compression of the Koopman operator exhibits 
high non-normality, and, as a result, the eigenvectors are highly ill-conditioned and the
KMD becomes numerically unstable.
%
We address this problem and introduce a new theoretical and computational framework for
data driven Koopman mode analysis of nonlinear dynamics. The 
problem of ill-conditioned eigenvectors  is solved using a Koopman-Schur 
decomposition that is entirely 
based on unitary transformations. The analysis in terms of the eigenvectors as 
modes of a Koopman operator  is replaced with a modal decomposition 
in terms of a flag of invariant subspaces that correspond to selected eigenvalues. 
  \\
%
The main computational tool from the numerical linear algebra is the partial 
ordered Schur decomposition that provides convenient orthonormal bases for these 
subspaces. \\
%
This is a joint work with Igor Mezi\'{c}, University of California at Santa Barbara.
%

Reference:\\
Zlatko Drma\v{c}, Igor Mezi\'{c} A data driven Koopman-Schur decomposition for 
computational analysis of nonlinear dynamics, SIAM J. Sci. Comp. (in review)\\
https://doi.org/10.48550/arXiv.2312.15837



\end{ilasabstract}
     \hypertarget{down0238}{}\begin{ilasabstract}
   \talktitle{Krylov: better, faster, parallel}
    
    \textbf{Fabio Durastante}, \info{16:30\textrm{--}17:00 @ SC4011 (June 24, Tuesday)} \hfill \hyperlink{up0238}{$\Uparrow$}
    
    (in {\color{mstitle}MS23: Advances in Krylov subspace methods and their applications})
        
        \mtskip
    Krylov subspace methods are indispensable for addressing a wide array of challenges in sparse and large-scale linear algebra. They are pivotal in solving large, sparse linear systems \( A \mathbf{x} = \mathbf{b} \), computing select eigenvalues and eigenvectors of expansive sparse matrices \( A \mathbf{v} = \lambda \mathbf{v} \), evaluating matrix-function vector products \( \mathbf{y} = f(A) \mathbf{x} \), and resolving linear matrix equations with low-rank right-hand sides \( A X + X B^T = UV^T \).
In contemporary computational science, ``large-scale'' denotes problems involving sparse matrices with millions to billions of unknowns. Addressing such complexities necessitates leveraging supercomputers and implementing distributed-memory versions of Krylov-based algorithms. The Parallel Sparse Computation Toolkit (PSCToolkit) is a comprehensive software framework designed to tackle these challenges. It offers modular components for managing distributed sparse matrices and executing sparse matrix computations across various hybrid architectures, from small servers to high-end supercomputers equipped with multicore CPUs and NVIDIA GPUs. PSCToolkit's design emphasizes node-level efficiency, flexibility, and usability, supporting integration with both Fortran and C/C++ applications. In this presentation, I will delve into the implementation nuances of Krylov subspace methods within distributed computing environments, focusing on the capabilities and structure of PSCToolkit. Furthermore, I will showcase scalability results and performance benchmarks achieved on supercomputers that integrate multicore processors and GPU accelerators, demonstrating PSCToolkit's efficacy in harnessing the computational power of modern heterogeneous systems.
\end{ilasabstract}
     \hypertarget{down0052}{}\begin{ilasabstract}
   \talktitle{Max-type quasidistances probability simplices}
    
    \textbf{Michal Eckstein}, \info{14:00\textrm{--}14:30 @ SC0014 (June 23, Monday)} \hfill \hyperlink{up0052}{$\Uparrow$}
    
    (in {\color{mstitle}MS8: Tensor and quantum information science})
        
        \mtskip
    A quasidistance is a function on a set, which is non-negative, non-degenerate and satisfies the triangle inequality. Despite the lack of symmetry, quasidistances lead to a rich theory involving geometric structures, such as geodesics. 

%It is thus a generalisation of a distance function, which need not be symmetric. 

We introduce a new family of max-type quasimetrics on probability simplices $\Delta_N$ defined by
\[
D_f(P,Q) = \max_i \big( f(q_i) - f(p_i) \big),
\]
where \(f\colon [0,1]\to [0,1]\) is a continuous, strictly increasing function with \(f(0)=0\) and \(f(1)=1\). Under mild regularity assumptions on \(f\), the quasimetric space $(\Delta_N,D_f)$ has a Finslerian structure and admits geodesics (both forward and backward) between any two points. Moreover, the function \(D_f\) is monotone under bistochastic maps.
\end{ilasabstract}
     \hypertarget{down0258}{}\begin{ilasabstract}
   \talktitle{nan}
    
    \textbf{Sven-Erik Ekström}, \info{11:30\textrm{--}12:00 @ SC1003 (June 25, Wednesday)} \hfill \hyperlink{up0258}{$\Uparrow$}
    
    (in {\color{mstitle}MS19: Explicit and hidden asymptotic structures, GLT Analysis, and applications})
        
        \mtskip
    nan\end{ilasabstract}
     \hypertarget{down0197}{}\begin{ilasabstract}
   \talktitle{Connecting the Hermite-Biehler theorem to the nonnegative inverse eigenvalue problem}
    
    \textbf{Richard Ellard}, \info{16:00\textrm{--}16:30 @ SC0008 (June 24, Tuesday)} \hfill \hyperlink{up0197}{$\Uparrow$}
    
    (in {\color{mstitle}MS24: Nonnegative and related families of matrices})
        
        \mtskip
    The famous Hermite-Biehler Theorem (Hermite, 1856; Biehler, 1879) states that a real polynomial $f(x)=p(x^2)+xq(x^2)$ is Hurwitz stable (all of the roots of $f$ lie in the open left half-plane) if and only if the leading coefficients of $p$ and $q$ have the same sign and all the roots of $p(-x^2)$ and $xq(-x^2)$ are real and interlace. More generally, the number and relative positions of the nonnegative roots of $p(-x)$ and $q(-x)$ determine the number of roots of $f$ which lie in the left (or right) half-plane. The \emph{Nonnegative Inverse Eigenvalue Problem} (NIEP) asks for a characterisation of those lists of complex numbers which are \emph{realisable} as the spectrum of some (entrywise) nonnegative matrix. An important special case arises when the Perron eigenvalue is the only root of the characteristic polynomial $f$ in the right half-plane, and, in this special case, a complete characterisation was given by Laffey and \v{S}migoc (2006) which employed a rather long and technical argument. By examining the relationship between the roots of $f$ and those of $p$ and $q$ from a simple algorithmic perspective, we give a new---and perhaps more elegant---proof of the Laffey-\v{S}migoc characterisation which provides a deeper insight into the result.
\end{ilasabstract}
     \hypertarget{down0355}{}\begin{ilasabstract}
   \talktitle{nan}
    
    \textbf{Srinivas Eswar}, \info{16:00\textrm{--}16:30 @ SC0009 (June 26, Thursday)} \hfill \hyperlink{up0355}{$\Uparrow$}
    
    (in {\color{mstitle}MS4: Linear algebra methods for inverse problems and data assimilation})
        
        \mtskip
    nan\end{ilasabstract}
     \hypertarget{down0391}{}\begin{ilasabstract}
   \talktitle{Krylov subspaces and Sobolev functions}
    
    \textbf{Amin Faghih}, \info{16:00\textrm{--}16:30 @ SC4011 (June 26, Thursday)} \hfill \hyperlink{up0391}{$\Uparrow$}
    
    (in {\color{mstitle}MS23: Advances in Krylov subspace methods and their applications})
        
        \mtskip
    In this talk, we will discuss how to generate Sobolev rational functions that are orthonormal with respect to a discrete Sobolev inner product through a long recurrence relation. The recurrence coefficients of this relation can be stored in a Hessenberg pencil. Two important numerical methods are those based on Krylov subspace methods and those based on restoring matrix structures. The methods are based on the connection between Sobolev orthonormal rational functions and the orthonormal bases for rational Krylov subspaces generated by a Jordan-like matrix.\end{ilasabstract}
     \hypertarget{down0010}{}\begin{ilasabstract}
   \talktitle{Preservers of totally positive and totally nonnegative matrices}
    
    \textbf{Shaun Fallat}, \info{11:00\textrm{--}11:30 @ SC0008 (June 23, Monday)} \hfill \hyperlink{up0010}{$\Uparrow$}
    
    (in {\color{mstitle}MS9: Total positivity})
        
        \mtskip
    A matrix is called totally nonnegative (positive) if all its minors are nonnegative (positive). In this talk we consider functions that preserve that class of totally nonnegative (positive) matrices. This subject has rightfully received significant attention over the years, including previous studies on characterizing surjective linear preservers and more recent interesting inquiries into various types of entry-wise preservers for this class of matrices. Building upon the basic fact that the class of totally nonnegative (positive) matrices forms a semigroup we highlight some existing work and investigate and report on some recent progress concerning multiplicative maps that preserve this semigroup of positive matrices.
\end{ilasabstract}
     \hypertarget{down0376}{}\begin{ilasabstract}
   \talktitle{Leslie's enduring influence on the IEP-G and zero forcing}
    
    \textbf{Shaun Fallat}, \info{16:30\textrm{--}17:00 @ SC1005 (June 26, Thursday)} \hfill \hyperlink{up0376}{$\Uparrow$}
    
    (in {\color{mstitle}MS17: Graphs and matrices in honor of Leslie Hogben's retirement})
        
        \mtskip
    When I think about Leslie's contributions to the IEP-G and Zero forcing, it is easy to conjure her impact via: Handbook(s) of LA, AIM workshops, AIM Squares, AIM ARC, BIRS FRGs, Sessions at ILAS, AMS, and the ``Boca'' conferences, not to mention, the very recent IEP-G/Zero forcing book with Shader and Lin…But Leslie has done SO much more to steer many recent advances in this area, including her numerous research team members from students, to postdocs, to collaborators. It is safe to say, the Leslie has personally opened the door to the IEP-G and Zero forcing to more people than anyone else (and it is not even close).  Her devotion to this topic and her steadfast support and encouragement to engage with so many researchers is astonishing…I hope to highlight several instances of Leslie's leadership and guidance that has shaped the direction of both the IEP-G and topics related to Zero forcing. When you look at the past 25 years her stamp is nearly everywhere on the IEP-G and Zero forcing. Personally, she was instrumental in supporting my research and I am truly blessed to have published 15 papers with my friend Dr. Leslie Hogben!\end{ilasabstract}
     \hypertarget{down0269}{}\begin{ilasabstract}
   \talktitle{Inheritance properties of conjugate discrete-time algebraic Riccati equations}
    
    \textbf{Hung-Yuan Fan}, \info{11:00\textrm{--}11:30 @ SC3001 (June 25, Wednesday)} \hfill \hyperlink{up0269}{$\Uparrow$}
    
    (in {\color{mstitle}MS11: Structured matrix computations and its applications})
        
        \mtskip
    In this talk we consider a class of conjugate discrete-time Riccati equations, arising originally from the linear quadratic regulator problem for discrete-time antilinear systems. Under mild and reasonable assumptions, the existence of the maximal solution to the conjugate discrete-time Riccati equation, in which the control weighting matrix is nonsingular and its constant term is Hermitian, will be inherited to a transformed discrete-time algebraic Riccati equation. Based on this inheritance property, an accelerated fixed-point iteration is proposed for finding the maximal solution via the transformed Riccati equation. Numerical examples are shown to illustrate the correctness of our theoretical results and the feasibility of the proposed algorithm.\end{ilasabstract}
     \hypertarget{down0167}{}\begin{ilasabstract}
   \talktitle{nan}
    
    \textbf{Ionut Farcas}, \info{14:30\textrm{--}15:00 @ SC0014 (June 24, Tuesday)} \hfill \hyperlink{up0167}{$\Uparrow$}
    
    (in {\color{mstitle}MS6: Model reduction})
        
        \mtskip
    nan\end{ilasabstract}
     \hypertarget{down0066}{}\begin{ilasabstract}
   \talktitle{Mixed-precision algorithms for the Sylvester matrix equation}
    
    \textbf{Massimiliano Fasi}, \info{15:00\textrm{--}15:30 @ SC2001 (June 23, Monday)} \hfill \hyperlink{up0066}{$\Uparrow$}
    
    (in {\color{mstitle}MS5: Advances in matrix equations: Theory, computations, and applications})
        
        \mtskip
    Modern supercomputers achieve their remarkable speeds by leveraging machine-learning hardware accelerators.
These accelerators deliver extraordinary throughput by trading off some degree of accuracy, and, to fully utilise their potential, we must rely on low-precision floating-point formats such as binary16, bfloat16, or binary8.
These reduced-precision formats can have a throughput up to two orders of magnitude higher than binary64, but they lack the precision needed for traditional scientific simulations, which require higher accuracy to yield meaningful results.
To integrate GPUs effectively into scientific computing, we must reimagine high-precision computations by strategically applying lower precision whenever feasible.

We consider techniques to solve the general Sylvester equation $AX + XB = C$ in mixed precision.
By revisiting a stationary iteration for linear systems, we derive a new iterative refinement scheme for the quasi-triangular Sylvester equation.
We then leverage this iterative scheme to solve the general Sylvester equation in mixed precision.
The new algorithms compute the Schur decomposition of the matrix coefficients in low precision, use the low-precision Schur factors to obtain an approximate solution to the quasi-triangular equation, and iteratively refine it to obtain a working-precision solution to the quasi-triangular equation.
However, being only orthonormal to low precision, the unitary Schur factors of A and B cannot be used to recover the solution to the original equation.
We propose two effective approaches to address this issue: one is based on re-orthonormalization in the working precision, and the other on explicit inversion of the almost-unitary factors.

This is joint work with Andrii Dmytryshyn, Nicholas J.~Higham, and Xiaobo Liu.
 \end{ilasabstract}
     \hypertarget{down0375}{}\begin{ilasabstract}
   \talktitle{Reconfiguring a community}
    
    \textbf{Mary Flagg}, \info{16:00\textrm{--}16:30 @ SC1005 (June 26, Thursday)} \hfill \hyperlink{up0375}{$\Uparrow$}
    
    (in {\color{mstitle}MS17: Graphs and matrices in honor of Leslie Hogben's retirement})
        
        \mtskip
    Reconfiguration is a process of transforming one feasible solution to a problem into another feasible solution in small incremental steps while maintaining the property of being a solution at each step. In the case of graph theory, solutions to problems are often subsets of the vertex set with a specific property. We refer to these as vertex set parameters. In this talk I will share a universal approach to vertex set parameter reconfiguration developed with Leslie Hogben and Bryan Curtis that is applied to zero forcing variants as well as other common graph parameters. Zero forcing was developed in this community to be an upper bound to the nullity of any real symmetric matrix with off-diagonal zero pattern determined by the adjacency matrix of a given graph. Understanding zero forcing sets has applications to understanding eigenvalues and eigenvectors of these matrices. 
Reconfiguration is also a description of how Leslie Hogben has helped to change our community for the better, one step at a time by facilitating community and research opportunity to graduate students, new researchers and faculty at undergraduate institutions. I will share my perspective on her reconfiguration efforts.
\end{ilasabstract}
     \hypertarget{down0314}{}\begin{ilasabstract}
   \talktitle{nan}
    
    \textbf{Colin Fox}, \info{15:00\textrm{--}15:30 @ SC0009 (June 26, Thursday)} \hfill \hyperlink{up0314}{$\Uparrow$}
    
    (in {\color{mstitle}MS4: Linear algebra methods for inverse problems and data assimilation})
        
        \mtskip
    nan\end{ilasabstract}
     \hypertarget{down0021}{}\begin{ilasabstract}
   \talktitle{On entaglement, separability and their computability}
    
    \textbf{Shmuel Friedland}, \info{12:00\textrm{--}12:30 @ SC0014 (June 23, Monday)} \hfill \hyperlink{up0021}{$\Uparrow$}
    
    (in {\color{mstitle}MS8: Tensor and quantum information science})
        
        \mtskip
    In this talk we survey some results on entanglement and separablity of general, symmetric (bosons),  skew-symmetric (fermions) tensors, and their computability.  We will survey briefly some results that I coauthored.\end{ilasabstract}
     \hypertarget{down0340}{}\begin{ilasabstract}
   \talktitle{nan}
    
    \textbf{Takeshi Fukaya}, \info{14:00\textrm{--}14:30 @ SC2006 (June 26, Thursday)} \hfill \hyperlink{up0340}{$\Uparrow$}
    
    (in {\color{mstitle}MS13: Advances in QR factorizations})
        
        \mtskip
    nan\end{ilasabstract}
     \hypertarget{down0169}{}\begin{ilasabstract}
   \talktitle{Approximation of reciprocal matrices by consistent matrices}
    
    \textbf{Susana Furtado}, \info{13:30\textrm{--}14:00 @ SC1001 (June 24, Tuesday)} \hfill \hyperlink{up0169}{$\Uparrow$}
    
    (in {\color{mstitle}MS21: Linear algebra techniques in graph theory})
        
        \mtskip
    An $n$-by-$n$ matrix $A=[a_{ij}]$ is said to be a pairwise comparison matrix (PC matrix)
or a reciprocal matrix if it is positive and $a_{ij}=\frac{1}{%
a_{ji}},$ for all $i,j=1,\ldots ,n.$ If, in addition, $a_{ik}a_{kj}=a_{ij}$ for
all $i,j,k,$ the matrix is said to be consistent. Such a matrix is of the
form $ww^{(-T)}$ for some positive vector $w$, in which $w^{(-T)}$ is the transpose of the entrywise
inverse of $w$.

PC matrices play an important role in decision making, namely in models for ranking
alternatives, as the Analytic Hierarchy Process,  proposed by Saaty (1977). 
In these models, a PC matrix represents independent, pairwise, ratio
comparisons among $n$ alternatives and a cardinal ranking vector should be
obtained from it. The consistent matrix constructed from this vector should
be a good approximation of the PC matrix. So, it is desirable to choose a
ranking vector from the set of efficient vectors, as, otherwise, there would
be a positive vector such that the consistent matrix constructed from it
better approximates the PC matrix in at least one entry and is not worse in
all other entries. 

It is known that a positive vector $w$ is efficient for a PC matrix $A$ if
and only if a certain directed graph associated with $A$ and $w$ is strongly
connected. Based on this result, here we give a description of the set of
efficient vectors for a PC matrix as a union of a finite number of
convex sets and discuss some of its consequences. In particular, tight lower
and upper bound matrices for the consistent matrix constructed from an
efficient vector are given. 

(This is a joint work with Charles~Johnson.) 
\end{ilasabstract}
     \hypertarget{down0144}{}\begin{ilasabstract}
   \talktitle{Inequalities on spectral geometric mean and application for relative entropy}
    
    \textbf{Shigeru Furuichi}, \info{10:30\textrm{--}11:00 @ SC2006 (June 24, Tuesday)} \hfill \hyperlink{up0144}{$\Uparrow$}
    
    (in {\color{mstitle}MS3: Matrix inequalities with applications})
        
        \mtskip
    \begin{bibunit}
        The weighted geometric mean is defined by $A\sharp_t B:=A^{\frac{1}{2}}\left(A^{-\frac{1}{2}}BA^{-\frac{1}{2}}\right)^t A^{\frac{1}{2}}$ for positive invertible operators $A,B$ and $0\leq t\leq 1.$ 
The weighted spectral geometric mean was defined in \cite{Lee1} by
\[A{{{\rm sp}}_{t}}B={{\left( {{A}^{-1}}\sharp B \right)}^{t}}A{{\left( {{A}^{-1}}\sharp B \right)}^{t}},\quad (0\leq t\leq 1).\]
Operator inequalities between the weighted spectral geometric mean $A{\rm sp}_t B$ and the weighted geometric mean $A\sharp_tB$ have compared in \cite{MFS2023} with the generalized Kantorovich constant for $0<m<M$ and $t\in \mathbb{R}$:
\[K\left( m,M,t \right)=\frac{\left(m{{M}^{t}}-M{{m}^{t}}\right)}{\left( t-1 \right)\left( M-m \right)}{{\left( \frac{t-1}{t}\frac{{{M}^{t}}-{{m}^{t}}}{m{{M}^{t}}-M{{m}^{t}}} \right)}^{t}}.\]
	
In this talk, we consider the case $t\notin (0,1)$. We firstly state the relations of $A\hat\sharp_t B$ and $A{{\hat{\rm sp}}_{t}}B$ for $t\notin (0,1)$ in L\"{o}wner order with the generalized Kantorovich constant. Norm inequalities are also shown for them.
We also give log--majorization and the bounds for the Tsallis relative entropy as its application.
Our talk is based on the results in \cite{FS2024}.


\begin{thebibliography}{5}

\bibitem{Lee1}
 H. Lee and Y. Lim, {\it Metric and spectral geometric means on symmetric cones}, Kyungpook Math. J., {\bf47}(1) (2007), 133--150.

\bibitem{MFS2023}
H. R. Moradi, S. Furuichi and M. Sababheh, {\it Operator spectral geometric versus geometric mean}, 	Linear Multilinear Algebra, {\bf 72}(2024), 997--1016.


\bibitem{FS2024} S. Furuichi and Y. Seo, {\it Some inequalities for spectral geometric mean
with applications}, Linear Multilinear Algebra, \url{doi.org/10.1080/03081087.2024.2433512}.
\end{thebibliography}
        \end{bibunit}
        \end{ilasabstract}
     \hypertarget{down0274}{}\begin{ilasabstract}
   \talktitle{Reversibility problem for real quaternion matrices}
    
    \textbf{Angelo Galimba}, \info{10:30\textrm{--}11:00 @ SC0008 (June 26, Thursday)} \hfill \hyperlink{up0274}{$\Uparrow$}
    
    (in {\color{mstitle}MS31: Matrix decompositions and applications})
        
        \mtskip
    In a group $G$, we say that $g \in G$ is \emph{reversible} if there exists $s \in G$ such that $s^{-1}gs = g^{-1}$. Moreover, such $g$ is \emph{strongly reversible} if $s$ is an involution, i.e., $s^2 = e$ where $e$ is the identity of the group $G$. In this talk, we study the reversibility problem in the group of invertible matrices over real quaternions $\mathbb{H}$, $GL(n, \mathbb{H})$. We completely classify up to standard Jordan form all real quaternion matrices in $GL(n,\mathbb{H})$ that are strongly reversible.\end{ilasabstract}
     \hypertarget{down0341}{}\begin{ilasabstract}
   \talktitle{nan}
    
    \textbf{Mark Gates}, \info{14:30\textrm{--}15:00 @ SC2006 (June 26, Thursday)} \hfill \hyperlink{up0341}{$\Uparrow$}
    
    (in {\color{mstitle}MS13: Advances in QR factorizations})
        
        \mtskip
    nan\end{ilasabstract}
     \hypertarget{down0026}{}\begin{ilasabstract}
   \talktitle{On the numerical solution of nonLocal boundary value problems by matrix function computations}
    
    \textbf{Luca Gemignani}, \info{11:30\textrm{--}12:00 @ SC1003 (June 23, Monday)} \hfill \hyperlink{up0026}{$\Uparrow$}
    
    (in {\color{mstitle}MS26: Utilizing structure to achieve low-complexity algorithms for data science, engineering, and physics})
        
        \mtskip
    Given a matrix $A\in \mathbb R^{s\times s}$ and a  vector $\mathbf {f} \in \mathbb{ R } ^s,$ under mild assumptions the non-local boundary value problem 
\begin{eqnarray*}
    &&\odv{\mathbf{u}}{\tau} = A \mathbf{u}, \quad 0<\tau<1,   \label{l1} \\
  &&\displaystyle \int_0^1 \mathbf{u}(\tau) \,\mathrm{d}\tau = \mathbf {f}, \label{l2}
\end{eqnarray*}
admits as unique solution
\[
 \mathbf{u}(\tau)= q(\tau,A) \mathbf {f}, \quad q(\tau,w)= \frac{w e^{w\tau}}{e^w -1}.
 \]
This talk  deals with efficient numerical methods for computing the action
of $q(\tau,A)$ on a vector,  when $A$ is a large and
  sparse matrix.  Methods based on the Fourier expansion of $q(\tau,w)$
are considered. First, we place
these methods in the classical framework of Krylov-Lanczos
(polynomial-rational) techniques for accelerating Fourier series.
This allows us to apply the convergence results developed in this
context to our function. Second, we design some  new acceleration schemes for computing $q(\tau,A) \mathbf {f}$. Numerical results are presented to show the effectiveness of
the proposed algorithms.


\emph{This is joint work with Lidia Aceto.}\end{ilasabstract}
     \hypertarget{down0256}{}\begin{ilasabstract}
   \talktitle{nan}
    
    \textbf{Dario Giandinoto}, \info{10:30\textrm{--}11:00 @ SC1003 (June 25, Wednesday)} \hfill \hyperlink{up0256}{$\Uparrow$}
    
    (in {\color{mstitle}MS19: Explicit and hidden asymptotic structures, GLT Analysis, and applications})
        
        \mtskip
    nan\end{ilasabstract}
     \hypertarget{down0272}{}\begin{ilasabstract}
   \talktitle{Spectral properties of stochastic block model}
    
    \textbf{Van Su Giap}, \info{11:00\textrm{--}11:30 @ SC4011 (June 25, Wednesday)} \hfill \hyperlink{up0272}{$\Uparrow$}
    
    (in {\color{mstitle}MS1: Embracing new opportunities in numerical linear algebra})
        
        \mtskip
    The stochastic block model (SBM) is an extension of the Erd\H{o}s-R\'{e}nyi graph and has applications in numerous fields, such as data analysis, recovering community structure in graph data and social networks. In this paper, we consider the normal central SBM adjacency matrix with $K$ communities of arbitrary sizes. We derive an explicit formula for the limiting empirical spectral density function when the size of the matrix tends to infinity. We also obtain an upper bound for the operator norm of such random matrices by means of the Stieltjes transform and random matrix theory.
\end{ilasabstract}
     \hypertarget{down0372}{}\begin{ilasabstract}
   \talktitle{Stable extraction of eigenpairs from a subspace for generalized eigenvalue problems}
    
    \textbf{Miryam Gnazzo}, \info{16:30\textrm{--}17:00 @ SC1003 (June 26, Thursday)} \hfill \hyperlink{up0372}{$\Uparrow$}
    
    (in {\color{mstitle}MS14: Pencils, polynomial, and rational matrices})
        
        \mtskip
    In this talk, we consider generalized eigenvalue problems associated with pencils in the form $A v = \lambda B v$, with $A,B \in \mathbb{R}^{m \times m}$. In several frameworks, we may be interested in the numerical approximation of a prescribed subset of eigenvectors, possibly coming from a good approximation of part of the eigenspace. This extraction of eigenpairs from a subspace can be done employing a class of methods often called oblique projectors. They consist in solving a different generalized eigenvalue problem $P^TAQv = \lambda P^T B Qv$, for suitable $P,Q\in \mathbb{R}^{m \times n}$ and $n<m$. A popular example of oblique projectors is Rayleigh-Ritz. However, it can be observed that this method is not backward stable. The goal of this talk consists in proposing alternative methods within the family of oblique projectors, with the additional property of being backward stable. Moreover, in settings where $m \gg n$, we present a randomized version of this technique, obtained via the solution a sketched version of the generalized eigenvalue problem.
\end{ilasabstract}
     \hypertarget{down0230}{}\begin{ilasabstract}
   \talktitle{nan}
    
    \textbf{Laureano Gonzalez-Vega}, \info{16:30\textrm{--}17:00 @ SC2006 (June 24, Tuesday)} \hfill \hyperlink{up0230}{$\Uparrow$}
    
    (in {\color{mstitle}MS30: Bohemian matrices: Theory, applications, and explorations})
        
        \mtskip
    nan\end{ilasabstract}
     \hypertarget{down0053}{}\begin{ilasabstract}
   \talktitle{Tensors structures in single-shot quantum information: from convex splits to induced divergences}
    
    \textbf{Gilad Gour}, \info{14:30\textrm{--}15:00 @ SC0014 (June 23, Monday)} \hfill \hyperlink{up0053}{$\Uparrow$}
    
    (in {\color{mstitle}MS8: Tensor and quantum information science})
        
        \mtskip
    Recent advances in quantum information theory reveal the power of tensor structures in communication and coding tasks. In this talk, I present two results that reformulate key protocols using tools from matrix analysis: an equality-based version of the convex split lemma, relying on collision mutual information derived from the sandwiched R\'enyi relative entropy of order $2$, and the induced divergence, a new family of smoothed quantum divergences. These developments yield sharper achievability bounds for state merging, splitting, and communication over quantum channels, while highlighting the central role of the collision relative entropy in quantum information.\end{ilasabstract}
     \hypertarget{down0384}{}\begin{ilasabstract}
   \talktitle{Commutativity concepts relative to transformation/matrix groups and semi-FTvN systems}
    
    \textbf{Muddappa Gowda}, \info{16:30\textrm{--}17:00 @ SC2006 (June 26, Thursday)} \hfill \hyperlink{up0384}{$\Uparrow$}
    
    (in {\color{mstitle}MS28: From matrix theory to Euclidean Jordan algebras, FTvN systems, and beyond})
        
        \mtskip
    We introduce the concepts of {\it commutativity} relative to a transformation/matrix group and {\it strong commutativity} in the setting of a semi-FTvN system and show their appearance as optimality conditions in certain optimization problems. In the setting of a semi-FTvN system (in particular, in an FTvN system), we show that strong commutativity implies commutativity and observe that in the special case of Euclidean Jordan algebra, commutativity and strong commutativity concepts reduce, respectively, to those of operator and strong operator commutativity. We demonstrate that every complete hyperbolic polynomial induces a semi-FTvN system. By way of an application, we describe several commutation principles. 

\end{ilasabstract}
     \hypertarget{down0310}{}\begin{ilasabstract}
   \talktitle{Some factorizations in the complex symplectic group}
    
    \textbf{Daryl Granario}, \info{15:00\textrm{--}15:30 @ SC0008 (June 26, Thursday)} \hfill \hyperlink{up0310}{$\Uparrow$}
    
    (in {\color{mstitle}MS31: Matrix decompositions and applications})
        
        \mtskip
    As one of the classical Lie groups, the complex symplectic group is fundamental not only in mathematics but also in various related fields. We look at some important matrix decompositions when restricted to the complex symplectic group. In particular, we look at how we can use symplectic versions of canonical forms to derive decompositions such as the one given by Ballantine concerning products of positive definite matrices, and the one given by Sourour concerning the spectra of factors in a product of matrices.
\end{ilasabstract}
     \hypertarget{down0063}{}\begin{ilasabstract}
   \talktitle{On the eigenvalues of the graphs $D(5, q)$}
    
    \textbf{Himanshu Gupta}, \info{15:00\textrm{--}15:30 @ SC1005 (June 23, Monday)} \hfill \hyperlink{up0063}{$\Uparrow$}
    
    (in {\color{mstitle}MS15: Graphs and their eigenvalues: Celebrating the work of Fan Chung Graham})
        
        \mtskip
    In 1995, Lazebnik and Ustimenko introduced the family of $q$-regular graphs $D(k, q)$, which is defined for any positive integer $k$ and prime power $q$. The connected components of the graph $D(k, q)$ have provided the best-known general lower bound on the size of a graph for any given order and girth to this day. Furthermore, Ustimenko conjectured that the second largest eigenvalue of $D(k, q)$ is always less than or equal to $2\sqrt{q}$, indicating that the graphs $D(k, q)$ are almost Ramanujan graphs. In this talk, we will discuss some recent progress on this conjecture. This includes the result that the second largest eigenvalue of $D(5, q)$ is less than or equal to $2\sqrt{q}$ when $q$ is an odd prime power. This is joint work with Vladislav Taranchuk.
\end{ilasabstract}
     \hypertarget{down0264}{}\begin{ilasabstract}
   \talktitle{Minimum number of distinct eigenvalues of Johnson and Hamming graphs}
    
    \textbf{Himanshu Gupta}, \info{11:30\textrm{--}12:00 @ SC2001 (June 25, Wednesday)} \hfill \hyperlink{up0264}{$\Uparrow$}
    
    (in {\color{mstitle}MS2: Combinatorial matrix theory})
        
        \mtskip
    This talk focuses on the inverse eigenvalue problem for graphs (IEPG), which seeks to determine the possible spectra of symmetric matrices associated with a given graph $G$. These matrices have off-diagonal non-zero entries corresponding to the edges of $G$, while diagonal entries are unrestricted. A key parameter in IEPG is $q(G)$, the minimum number of distinct eigenvalues among such matrices. The Johnson and Hamming graphs are well-studied families of graphs with many interesting combinatorial and algebraic properties. We prove that every Johnson graph admits a signed adjacency matrix with exactly two distinct eigenvalues, establishing that its $q$-value is two. Additionally, we explore the behavior of $q(G)$ for Hamming graphs. This is a joint work with Shaun Fallat, Allen Herman, and Johnna Parenteau. 
\end{ilasabstract}
     \hypertarget{down0377}{}\begin{ilasabstract}
   \talktitle{nan}
    
    \textbf{Tracy Hall}, \info{17:00\textrm{--}17:30 @ SC1005 (June 26, Thursday)} \hfill \hyperlink{up0377}{$\Uparrow$}
    
    (in {\color{mstitle}MS17: Graphs and matrices in honor of Leslie Hogben's retirement})
        
        \mtskip
    nan\end{ilasabstract}
     \hypertarget{down0329}{}\begin{ilasabstract}
   \talktitle{A Ritz method for solution of parametric generalized EVPs}
    
    \textbf{Antti Hannukainen}, \info{14:30\textrm{--}15:00 @ SC1003 (June 26, Thursday)} \hfill \hyperlink{up0329}{$\Uparrow$}
    
    (in {\color{mstitle}MS14: Pencils, polynomial, and rational matrices})
        
        \mtskip
    This talk deals with approximate solution of generalized eigenvalue problem with coefficient matrix that is an affine function of $d$-parameters. The coefficient matrix is assumed to be symmetric positive definite and spectrally equivalent to an average matrix for all parameters in a given set. We develop a Ritz method for rapidly approximating the eigenvalues on the spectral interval of interest $(0,\Lambda)$ for given parameter value. The Ritz subspace is the same for all parameters and it is designed based on the observation that any eigenvector can be split into two components. The first component belongs to a subspace spanned by some eigenvectors of the average matrix. The second component is defined by a correction operator that is a $d+1$ dimensional analytic function. We use this structure and build our Ritz subspace from eigenvectors of the average matrix and samples of the correction operator. The samples are evaluated at interpolation points related to a sparse polynomial interpolation method. We show that the resulting Ritz subspace can approximate eigenvectors of the original problem related to the spectral interval of interest with the same accuracy as the sparse polynomial interpolation approximates the correction operator. Bound for Ritz eigenvalue error follows from this and known results. Theoretical results are illustrated by numerical examples. The advantage of our approach is that the analysis treats multiple eigenvalues and eigenvalue crossings that typically have posed technical challenges in similar works.
\end{ilasabstract}
     \hypertarget{down0128}{}\begin{ilasabstract}
   \talktitle{Isometries on groups of invertible elements in Fourier-Stieltjes algebras}
    
    \textbf{Osamu Hatori}, \info{11:30\textrm{--}12:00 @ SC0012 (June 24, Tuesday)} \hfill \hyperlink{up0128}{$\Uparrow$}
    
    (in {\color{mstitle}MS12: Preserver problems, I})
        
        \mtskip
    If open subgroups of the groups of invertible elements in two Fourier-Stieltjes algebras are isometric as metric spaces, then the underlying locally compact groups are topologically isomorphic.
\end{ilasabstract}
     \hypertarget{down0183}{}\begin{ilasabstract}
   \talktitle{High-order eulerian numbers and matrices}
    
    \textbf{Tian-Xiao He}, \info{14:30\textrm{--}15:00 @ SC2001 (June 24, Tuesday)} \hfill \hyperlink{up0183}{$\Uparrow$}
    
    (in {\color{mstitle}MS25: Enumerative/algebraic combinatorics and matrices})
        
        \mtskip
    We study the properties of the higher-order Eulerian numbers and the higher-order Eulerian matrices. The row generating functions and the row sums of the higher-order Eulerian matrices are given. We also define higher-order Eulerian fractions and their alternative forms. Some properties of higher-order Eulerian fractions are expressed using differentials and integrals. The inversion relationships between second-order Eulerian numbers and Stirling numbers of the second and first kinds are given. We also give exact expressions for the entries of higher-order Eulerian matrices.\end{ilasabstract}
     \hypertarget{down0062}{}\begin{ilasabstract}
   \talktitle{Degenerate eigenvalues for the non-backtracking matrix}
    
    \textbf{Kristin Heysse}, \info{14:30\textrm{--}15:00 @ SC1005 (June 23, Monday)} \hfill \hyperlink{up0062}{$\Uparrow$}
    
    (in {\color{mstitle}MS15: Graphs and their eigenvalues: Celebrating the work of Fan Chung Graham})
        
        \mtskip
    The non-backtracking matrix is the transition matrix for a walk in a graph which cannot traverse an edge twice in immediate succession. The spectral information of this matrix has seen great interest, notably in applications of network analysis. The non-symmetric nature of this matrix allows for graphs without a full basis of eigenvectors, which results in nontrivial Jordan chains. In this talk, we will consider the Jordan form of this matrix and construct infinite families of graphs with nontrivial Jordan chains. 
\end{ilasabstract}
     \hypertarget{down0013}{}\begin{ilasabstract}
   \talktitle{Joint concavity/convexity of matrix trace functions for geometric type means}
    
    \textbf{Fumio Hiai}, \info{11:00\textrm{--}11:30 @ SC0009 (June 23, Monday)} \hfill \hyperlink{up0013}{$\Uparrow$}
    
    (in {\color{mstitle}MS29: Matrix functions and related topics})
        
        \mtskip
    In this talk we consider the quasi extensions of the weighted geometric type means, including
\begin{align*}
&G_{\alpha,p}(A,B):=(A^p\#_\alpha B^p)^{1/p},
\ \mbox{the quasi-weighted geometric mean}, \\
&SG_{\alpha,p}(A,B):=(A^p\#_\alpha B^p)^{1/p},
\ \mbox{the quasi-weighted spectral geometric mean}, \\
&R_{\alpha,p}(A,B):=\bigl(B^{{\frac{1-\alpha}{2}}p}A^{\alpha p}B^{{\frac{1-\alpha}{2}}p}\bigr)^{1/p},
\ \mbox{the R\'enyi mean}, \\
&LE_\alpha(A,B):=\exp(\alpha\log A+(1-\alpha)\log B),
\ \mbox{the weighted Log-Euclidean mean},
\end{align*}
where $\alpha>0$ is the weight parameter and $p>0$ is the parameter of the quasi extension. We aim at determining the range of the parameters $\alpha,p$ under which the trace function $\mathrm{Tr}\,\mathcal{M}_{\alpha,p}(A,B)$ is jointly concave (also jointly convex) for each $\mathcal{M}_{\alpha,p}$ from the above quasi-weighted geometric type means. Our discussions are in strong relation to the monotonicity property (or date-processing inequality) of quantum divergences in quantum information.
\end{ilasabstract}
     \hypertarget{down0248}{}\begin{ilasabstract}
   \talktitle{Tingley's problem concerning the direct sum of extremely C-regular subspaces with the $\ell^p$-norm.}
    
    \textbf{Daisuke Hirota}, \info{11:00\textrm{--}11:30 @ SC0012 (June 25, Wednesday)} \hfill \hyperlink{up0248}{$\Uparrow$}
    
    (in {\color{mstitle}MS35: Preserver Problems, II})
        
        \mtskip
    Tingley's problem asks whether every surjective isometry between the unit spheres of two Banach spaces can be 
extended to a surjective real-linear \nobreak{isometry} between the whole spaces. 
Let $\{A_{\lambda}\}_{\lambda\in\Lambda}$ be a family of uniformly closed, extremely C-regular subspaces,  
and 
let $p$ be a real number such that $1<p<\infty$ with $p\neq 2$. 
We denote by $A_{\Lambda}^{p}$ the Banach space of formed by the direct sum of $\{A_{\lambda}\}_{\lambda\in \Lambda}$ 
equipped with the norm $\|{\bm{f}}\|_{p}=\left(\sum_{\lambda\in \Lambda}\|{\bm{f}}_{\lambda}\|_{\infty}^{p}\right)^{1/p}$ for ${\bm{f}}\in A_{\Lambda}^{p}$. \\
\quad In this presentation, I will speak on the fact that
 if $\Delta$ is a surjective isometry between two unit spheres $S(A_{M}^{p})$ and $S(A_{N}^{p})$ of the Banach spaces 
 $A_{M}^{p}$ and $A_{N}^{p}$, 
 then $\Delta$ admits an extension to a surjective real-linear isometry between the whole spaces.\end{ilasabstract}
     \hypertarget{down0174}{}\begin{ilasabstract}
   \talktitle{An optimal preconditioned MINRES method for nonsymmetric multilevel block Toeplitz systems with applications}
    
    \textbf{Sean Hon}, \info{14:00\textrm{--}14:30 @ SC1003 (June 24, Tuesday)} \hfill \hyperlink{up0174}{$\Uparrow$}
    
    (in {\color{mstitle}MS19: Explicit and hidden asymptotic structures, GLT Analysis, and applications})
        
        \mtskip
    In this talk, we introduce a novel preconditioning strategy for solving multilevel block Toeplitz systems, with applications to nonlocal evolutionary fractional diffusion equations. We show that the Hermitian part of certain nonsymmetric systems serves as an ideal preconditioner. By transforming the nonsymmetric block Toeplitz system into a symmetric block Hankel system using a symmetrization technique, we propose a symmetric positive definite block Tau preconditioner that is efficiently implemented using the discrete sine transform. We prove that this approach enables mesh-size-independent convergence with eigenvalues of the preconditioned matrices contained in specific intervals around $\pm 1$. Numerical results will be presented to demonstrate the method's efficiency in terms of iterations and computation time. This is joint work with Grigorios Tachyridis.\end{ilasabstract}
     \hypertarget{down0296}{}\begin{ilasabstract}
   \talktitle{Spectral community detection in geometric random graphs}
    
    \textbf{Carlos Hoppen}, \info{11:00\textrm{--}11:30 @ SC2001 (June 26, Thursday)} \hfill \hyperlink{up0296}{$\Uparrow$}
    
    (in {\color{mstitle}MS2: Combinatorial matrix theory})
        
        \mtskip
    This talk is about community detection in graphs defined by the Soft Geometric Block Model (SGBM). Suppose that $n=k \ell$ points are classified into $k \geq 2$ communities with $\ell$ points in each community. These $n$ points are embedded independently and with uniform probability as points $X_1,\ldots,X_n$ of the $d$-dimensional flat unit torus $\mathbf{T^d}$. A random graph $G$ is generated so that the edge $\{i,j\}$ appears with probability $F_{in}(||X_i-X_j||)$ if $i$ and $j$ belong to the same community, and with probability $F_{out}(||X_i-X_j||)$  if $i$ and $j$ belong to different communities. Under some technical conditions about $F_{in}$ and $F_{out}$, we shall prove that asymptotically almost surely there is a set of $k-1$ eigenvalues of the adjacency matrix $A(G)$ whose eigenspaces allow us to correctly identify the members of each community. This strategy gives a spectral algorithm for community detection in random geometric graphs. The talk is based on joint work with Konstantin Avrachenkov (INRIA-Sophia Antipolis), Luiz Emilio Allem (UFRGS), Hariprasad Manjunath (Chanakya University), and Lucas Siviero Sibemberg (UFRGS), and generalizes a result of Avrachenkov, Bobu and Dreveton [Avrachenkov, K., Bobu, A., Dreveton, M., \emph{Higher-Order Spectral Clustering for Geometric Graphs}, Journal of Fourier Analysis and Applications {\bf 27} (2021), article 22.], which deals with the case of $k=2$ communities. 
\end{ilasabstract}
     \hypertarget{down0252}{}\begin{ilasabstract}
   \talktitle{nan}
    
    \textbf{Pawel Horodecki}, \info{11:30\textrm{--}12:00 @ SC0014 (June 25, Wednesday)} \hfill \hyperlink{up0252}{$\Uparrow$}
    
    (in {\color{mstitle}MS8: Tensor and quantum information science})
        
        \mtskip
    nan\end{ilasabstract}
     \hypertarget{down0187}{}\begin{ilasabstract}
   \talktitle{nan}
    
    \textbf{Di Hou}, \info{14:30\textrm{--}15:00 @ SC2006 (June 24, Tuesday)} \hfill \hyperlink{up0187}{$\Uparrow$}
    
    (in {\color{mstitle}MS32: Advances in matrix manifold optimization})
        
        \mtskip
    nan\end{ilasabstract}
     \hypertarget{down0239}{}\begin{ilasabstract}
   \talktitle{Lanczos with compression for symmetric Lyapunov equations}
    
    \textbf{Francesco Hrobat}, \info{17:00\textrm{--}17:30 @ SC4011 (June 24, Tuesday)} \hfill \hyperlink{up0239}{$\Uparrow$}
    
    (in {\color{mstitle}MS23: Advances in Krylov subspace methods and their applications})
        
        \mtskip
    In this work, we present a low-memory variant of the Lanczos algorithm for the solution of the Lyapunov equation
\begin{equation}\label{eqn:lyap}
AX + XA = \boldsymbol{c}\boldsymbol{c}^T,    
\end{equation}
where $A$ is a large-scale symmetric positive-definite matrix and $\boldsymbol{c}$ is a vector. 

The classical Lanczos method consists in building an orthonormal basis $\mathbf{Q}_M$ for the polynomial Krylov subspace
\[ \mathcal{K}_M(A,\boldsymbol{c}) = span(\boldsymbol{c},A\boldsymbol{c}, \dots, A^{M-1}\boldsymbol{c}) \]
and in approximating the solution $X$ with $\mathbf{Q}_MX_M\mathbf{Q}_M^T$, where $X_M$ solves the projected equation
\[ \mathbf{Q}_M^TA\mathbf{Q}_M X_M + X_M\mathbf{Q}_M^TA\mathbf{Q}_M = \| \boldsymbol{c} \|_F^2\boldsymbol{e}_1\boldsymbol{e}_1^T.\]
The Lanczos algorithm often requires a relatively large $M$ to obtain a good approximation of the solution, which can lead to memory issues due to the storage demands of $\mathbf{Q}_M$. Furthermore, the solution $X$ can be well approximated by a low-rank matrix, whose rank is significantly smaller than $M$, i.e. the dimension of the polynomial Krylov subspace. 

An alternative approach is to use a rational Krylov subspace instead of a polynomial one. Using the Zolotarev poles as the poles of the rational Krylov subspace, it is possible to approximate the solution $X$ by a low-rank matrix with the guarantee that the residual has norm smaller than a prescribed quantity [$2$]. The rank of the computed approximate solution is usually close to the numerical rank of the real solution. The main drawback is that this method requires solving multiple shifted linear systems involving the matrix $A$, which is prohibitive if $A$ is large. 

Mimicking the approach in [$3$], our method employs a polynomial Krylov subspace to approximate the solution of \eqref{eqn:lyap} while leveraging rational Krylov subspaces associated with small matrices to compress the Lanczos basis. This method accesses $A$ only through matrix-vector products and requires the storage of only a few vectors from the polynomial Krylov subspace instead of the entire Lanczos basis, producing an approximate solution whose rank is independent of the dimension of the involved polynomial Krylov subspace.

The computational cost of the proposed algorithm is dominated by the construction of the Lanczos basis, and the compression steps do not require additional matrix-vector products involving $A$. Furthermore, theoretical results demonstrate that the algorithm achieves an approximation error comparable to that of the standard Lanczos algorithm, with an additional error term that can be bounded a priori using Zolotarev numbers. In practice, this additional error is negligible compared to the Lanczos error.

Numerical experiments show that the behavior of the proposed algorithm is comparable to that of the Lanczos algorithm without reorthogonalization, both in terms of matrix-vector products and quality of the approximated solution. Comparisons with existing low-memory variants of the Lanczos method demonstrate competitive performance in terms of accuracy, computational cost, and runtime.


\begin{itemize}
\item[1] A. A. Casulli, F. H., D. Kressner, Lanczos with Rational Krylov compression for symmetric Lyapunov equations, In preparation.
\item[2] B. Beckermann, An error analysis for rational Galerkin projection applied to the Sylvester
equation, SIAM J. Numer. Anal., 49 (2011), pp. 2430--2450. 
\item[3] A. A. Casulli and I. Simunec. A low-memory Lanczos method with rational
Krylov compression for matrix functions, arXiv, 2024.
\end{itemize} 
\end{ilasabstract}
     \hypertarget{down0359}{}\begin{ilasabstract}
   \talktitle{Quantum complementarity: A novel resource for exclusion}
    
    \textbf{Chung-Yun Hsieh}, \info{16:00\textrm{--}16:30 @ SC0012 (June 26, Thursday)} \hfill \hyperlink{up0359}{$\Uparrow$}
    
    (in {\color{mstitle}MS7: Linear algebra and quantum information science})
        
        \mtskip
    Complementarity is a phenomenon explaining several core features of quantum theory, such as the well-known uncertainty principle. Roughly speaking, two objects are said to be {\em complementary} if being certain about one of them necessarily forbids useful knowledge about the other. Two quantum measurements that do not commute form an example of complementary measurements, and this phenomenon can also be defined for ensembles of states. Although a key quantum feature, it is unclear whether complementarity can be understood more operationally, as a necessary resource in some quantum information task. Here we show this is the case, and relates to a novel task which we term {\em $\eta$-unambiguous exclusion}. As well as giving complementarity a clear operational definition, this also uncovers the foundational underpinning of unambiguous exclusion tasks for the first time. \\\\
Reference: C.-Y. Hsieh, R. Uola, P. Skrzypczyk, {\em Quantum complementarity: A novel resource for unambiguous exclusion and encryption}, arXiv:2309.11968.
\end{ilasabstract}
     \hypertarget{down0361}{}\begin{ilasabstract}
   \talktitle{nan}
    
    \textbf{Hsien-Yi Hsieh}, \info{17:00\textrm{--}17:30 @ SC0012 (June 26, Thursday)} \hfill \hyperlink{up0361}{$\Uparrow$}
    
    (in {\color{mstitle}MS7: Linear algebra and quantum information science})
        
        \mtskip
    nan\end{ilasabstract}
     \hypertarget{down0208}{}\begin{ilasabstract}
   \talktitle{nan}
    
    \textbf{Shao-Hua Hu}, \info{17:30\textrm{--}18:00 @ SC0012 (June 24, Tuesday)} \hfill \hyperlink{up0208}{$\Uparrow$}
    
    (in {\color{mstitle}MS7: Linear algebra and quantum information science})
        
        \mtskip
    nan\end{ilasabstract}
     \hypertarget{down0040}{}\begin{ilasabstract}
   \talktitle{Coordinate testing for general sufficient dimension reduction methods}
    
    \textbf{Shih-Hao Huang}, \info{11:00\textrm{--}11:30 @ SC4011 (June 23, Monday)} \hfill \hyperlink{up0040}{$\Uparrow$}
    
    (in {\color{mstitle}MS20: Manifold learning and statistical applications})
        
        \mtskip
    In modern data analysis, the number of covariates is often large, and the relationship between covariates and response is often complex. Parametric regression risks model misspecification, while nonparametric regression suffers from the curse of dimensionality. Sufficient dimension reduction (SDR) regression provides a flexible alternative, summarizing covariate effects through a few linear combinations without imposing a specific functional form. While SDR methods have been extensively studied, coordinate testing, which assesses the contribution of a set of linear combinations of covariates, has been largely overlooked. To address this gap, we propose a novel method that transforms the coordinate testing problem into a dimension testing problem by applying appropriate residualization. Since dimension tests are well-established, this method allows practitioners to leverage existing inference tools within the SDR framework.
\end{ilasabstract}
     \hypertarget{down0133}{}\begin{ilasabstract}
   \talktitle{nan}
    
    \textbf{Huajun Huang}, \info{11:00\textrm{--}11:30 @ SC1001 (June 24, Tuesday)} \hfill \hyperlink{up0133}{$\Uparrow$}
    
    (in {\color{mstitle}MS10: Matrix means and related topics})
        
        \mtskip
    nan\end{ilasabstract}
     \hypertarget{down0151}{}\begin{ilasabstract}
   \talktitle{Application of VoxelMorph and SynthMorph for multitype 3D medical image registration}
    
    \textbf{Yu-Jie Huang}, \info{11:00\textrm{--}11:30 @ SC4011 (June 24, Tuesday)} \hfill \hyperlink{up0151}{$\Uparrow$}
    
    (in {\color{mstitle}MS20: Manifold learning and statistical applications})
        
        \mtskip
    In this work, we explore the application of VoxelMorph and SynthMorph for 3D medical image registration. VoxelMorph provides a deep learning-based framework that utilizes deformable pairwise registration, enabling the alignment of complex anatomical structures through continuous deformation fields. By modeling transformations as smooth manifolds, VoxelMorph facilitates precise, non-rigid alignment, making it well-suited for capturing intricate deformations in medical images. SynthMorph further extends this approach, offering a robust method for image registration across diverse MRI contrasts, effectively handling variations in image intensities without relying on acquired imaging data. We investigate how these deformable models can be integrated into existing medical imaging workflows to enhance multi-type image registration, improving the alignment of images with varying anatomical and contrast properties.\end{ilasabstract}
     \hypertarget{down0191}{}\begin{ilasabstract}
   \talktitle{nan}
    
    \textbf{Longxiu Huang}, \info{14:30\textrm{--}15:00 @ SC3001 (June 24, Tuesday)} \hfill \hyperlink{up0191}{$\Uparrow$}
    
    (in {\color{mstitle}MS18: New methods in numerical multilinear algebra})
        
        \mtskip
    nan\end{ilasabstract}
     \hypertarget{down0015}{}\begin{ilasabstract}
   \talktitle{Near-order relation of power means}
    
    \textbf{Jinmi Hwang}, \info{12:00\textrm{--}12:30 @ SC0009 (June 23, Monday)} \hfill \hyperlink{up0015}{$\Uparrow$}
    
    (in {\color{mstitle}MS29: Matrix functions and related topics})
        
        \mtskip
    On the setting of positive definite operators we study the near-order properties of power means such as the quasi-arithmetic mean (H\"{o}lder mean) and R\'{e}nyi power mean. We see the monotonicity of spectral geometric mean and Wasserstein mean on parameters with respect to the near-order and the near-order relationship between the spectral geometric mean and Wasserstein mean. Furthermore, the monotonicity of quasi-arithmetic mean on parameters and the convergence of R\'{e}nyi power mean to the log-Euclidean mean with respect to the near-order have been established.\end{ilasabstract}
     \hypertarget{down0345}{}\begin{ilasabstract}
   \talktitle{nan}
    
    \textbf{Martina Iannacito}, \info{14:30\textrm{--}15:00 @ SC3001 (June 26, Thursday)} \hfill \hyperlink{up0345}{$\Uparrow$}
    
    (in {\color{mstitle}MS16: Approximations and errors in Krylov-based solvers})
        
        \mtskip
    nan\end{ilasabstract}
     \hypertarget{down0065}{}\begin{ilasabstract}
   \talktitle{nan}
    
    \textbf{Bruno Iannazzo}, \info{14:30\textrm{--}15:00 @ SC2001 (June 23, Monday)} \hfill \hyperlink{up0065}{$\Uparrow$}
    
    (in {\color{mstitle}MS5: Advances in matrix equations: Theory, computations, and applications})
        
        \mtskip
    nan\end{ilasabstract}
     \hypertarget{down0176}{}\begin{ilasabstract}
   \talktitle{Quasi-boundary regularization for space-time fractional diffusion equations with variable coefficients}
    
    \textbf{Asim Ilyas}, \info{15:00\textrm{--}15:30 @ SC1003 (June 24, Tuesday)} \hfill \hyperlink{up0176}{$\Uparrow$}
    
    (in {\color{mstitle}MS19: Explicit and hidden asymptotic structures, GLT Analysis, and applications})
        
        \mtskip
    This work addresses the inverse problem of reconstructing a source term in a space-time fractional diffusion equation with variable coefficients, using final time observations and a quasi-boundary value regularization method. The equation under consideration incorporates a Caputo fractional derivative in space and a tempered fractional derivative in time, both of order between 0 and 1. These types of equations are particularly relevant in various applied fields. To numerically approximate the regularized problem, we employ a finite difference scheme, which results in a large-scale two-by-two block linear system. The study presents theoretical insights into the spectral properties of both non-preconditioned and preconditioned matrix sequences, using tools from Toeplitz and Generalized Locally Toeplitz (GLT) theory. Notably, the preconditioner, derived from the GLT framework, is introduced and analyzed here for the first time in this context. Numerical experiments are conducted to validate the theoretical findings, followed by concluding remarks.
\end{ilasabstract}
     \hypertarget{down0349}{}\begin{ilasabstract}
   \talktitle{nan}
    
    \textbf{Akira Imakura}, \info{14:30\textrm{--}15:00 @ SC4011 (June 26, Thursday)} \hfill \hyperlink{up0349}{$\Uparrow$}
    
    (in {\color{mstitle}MS1: Embracing new opportunities in numerical linear algebra})
        
        \mtskip
    nan\end{ilasabstract}
     \hypertarget{down0046}{}\begin{ilasabstract}
   \talktitle{The weighted power difference mean and its generalization}
    
    \textbf{Masatoshi Ito}, \info{14:00\textrm{--}14:30 @ SC0009 (June 23, Monday)} \hfill \hyperlink{up0046}{$\Uparrow$}
    
    (in {\color{mstitle}MS29: Matrix functions and related topics})
        
        \mtskip
    Pal, Singh, Moslehian and Aujla (2016) introduced the weighted logarithmic mean for two positive numbers
 or operators on a complex Hilbert space, which is based on an extension of the Hermite-Hadamard inequality.
Furuichi and Minculete (2020) obtained a refinement of the inequality by Pal et al.
On the other hand, we discussed relations among some weighted operator means
 by considering the notion of a transpose symmetric path of weighted means,
 and we introduced the weighted Heinz mean.
In this talk, based on these arguments,
 we  newly introduce the weighted power difference mean and get relations
 among the weighted power, power difference and arithmetic means.
Moreover, we generalize these results from the viewpoint of a transpose symmetric path.

\end{ilasabstract}
     \hypertarget{down0122}{}\begin{ilasabstract}
   \talktitle{nan}
    
    \textbf{Tanvi Jain}, \info{11:30\textrm{--}12:00 @ SC0008 (June 24, Tuesday)} \hfill \hyperlink{up0122}{$\Uparrow$}
    
    (in {\color{mstitle}MS9: Total positivity})
        
        \mtskip
    nan\end{ilasabstract}
     \hypertarget{down0123}{}\begin{ilasabstract}
   \talktitle{New weighted spectral geometric mean and quantum divergence}
    
    \textbf{Miran Jeong}, \info{10:30\textrm{--}11:00 @ SC0009 (June 24, Tuesday)} \hfill \hyperlink{up0123}{$\Uparrow$}
    
    (in {\color{mstitle}MS29: Matrix functions and related topics})
        
        \mtskip
    A new class of weighted spectral geometric means has recently been introduced. In this talk, we present its inequalities in terms of the L\"{o}wner order, operator
norm, and trace.
Moreover, we establish a log-majorization relationship between the new spectral geometric mean, and the R\'{e}nyi relative operator entropy.
We also give the quantum divergence of the quantity, given by the difference of trace values between the
arithmetic mean and new spectral geometric mean.
Finally, we study the barycenter that minimizes the weighted sum of quantum divergences for given variables.
\end{ilasabstract}
     \hypertarget{down0383}{}\begin{ilasabstract}
   \talktitle{nan}
    
    \textbf{Juyoung Jeong}, \info{16:00\textrm{--}16:30 @ SC2006 (June 26, Thursday)} \hfill \hyperlink{up0383}{$\Uparrow$}
    
    (in {\color{mstitle}MS28: From matrix theory to Euclidean Jordan algebras, FTvN systems, and beyond})
        
        \mtskip
    nan\end{ilasabstract}
     \hypertarget{down0033}{}\begin{ilasabstract}
   \talktitle{Median eigenvalues of subcubic graphs}
    
    \textbf{Zilin Jiang}, \info{12:00\textrm{--}12:30 @ SC2001 (June 23, Monday)} \hfill \hyperlink{up0033}{$\Uparrow$}
    
    (in {\color{mstitle}MS2: Combinatorial matrix theory})
        
        \mtskip
    We present a resolution to conjectures by Fowler, Pisanski, and Mohar regarding the median eigenvalues of subcubic (chemical) graphs. Specifically, we prove that the median eigenvalues of every connected graph with maximum degree at most three, except for the Heawood graph, lie within the interval $[-1, 1]$. This result has significant implications in mathematical chemistry, particularly in the analysis of molecular orbital models, and extends prior work on bipartite chemical graphs.
\end{ilasabstract}
     \hypertarget{down0308}{}\begin{ilasabstract}
   \talktitle{Product of skew-involutions}
    
    \textbf{Jesus Paolo Joven}, \info{14:00\textrm{--}14:30 @ SC0008 (June 26, Thursday)} \hfill \hyperlink{up0308}{$\Uparrow$}
    
    (in {\color{mstitle}MS31: Matrix decompositions and applications})
        
        \mtskip
    We show that every $2n$-by-$2n$ matrix over a field $\mathbb{F}$ with determinant 1 is a product of (i) four or fewer skew-involutions ($A^2 = -I$) provided $\mathbb{F} \neq \mathbb{Z}_3$, and (ii) eight or fewer skew-involutions if $\mathbb{F} = \mathbb{Z}_3$ and $n > 1$. We also show that every real symplectic matrix is a product of six real symplectic skew-involutions. 
\end{ilasabstract}
     \hypertarget{down0322}{}\begin{ilasabstract}
   \talktitle{nan}
    
    \textbf{Rudra Kamat}, \info{15:00\textrm{--}15:30 @ SC0014 (June 26, Thursday)} \hfill \hyperlink{up0322}{$\Uparrow$}
    
    (in {\color{mstitle}MS5: Advances in matrix equations: Theory, computations, and applications})
        
        \mtskip
    nan\end{ilasabstract}
     \hypertarget{down0070}{}\begin{ilasabstract}
   \talktitle{A semi-definite optimization method for maximizing the shared band gap of topologicalpPhotonic crystals}
    
    \textbf{Chiu-Yen Kao}, \info{14:00\textrm{--}14:30 @ SC3001 (June 23, Monday)} \hfill \hyperlink{up0070}{$\Uparrow$}
    
    (in {\color{mstitle}MS1: Embracing new opportunities in numerical linear algebra})
        
        \mtskip
    Topological photonic crystals (PCs) can support robust edge modes to transport electromagnetic energy in an efficient manner. Such edge modes are the eigenmodes of the PDE operator for a joint optical structure formed by connecting together two photonic crystals with distinct topological invariants, and the corresponding eigenfrequencies are located in the shared band gap of two individual photonic crystals. This work is concerned with maximizing the shared band gap of two photonic crystals with different topological features in order to increase the bandwidth of the edge modes. We develop a semi-definite optimization framework for the underlying optimal design problem, which enables efficient update of dielectric functions at each time step while respecting symmetry constraints and, when necessary, the constraints on topological invariants. At each iteration, we perform sensitivity analysis 
of the band gap function and the topological invariant constraint function to linearize the optimization problem and solve a convex semi-definite programming (SDP) problem efficiently. Numerical examples show that the proposed algorithm is superior in generating optimized optical structures with robust edge modes. (This is joint work with Junshan Lin at Auburn University and Braxton Osting at University of Utah)\end{ilasabstract}
     \hypertarget{down0044}{}\begin{ilasabstract}
   \talktitle{Unimodality preservation by ratios of functional series and integral transforms}
    
    \textbf{Dmitrii Karp}, \info{14:30\textrm{--}15:00 @ SC0008 (June 23, Monday)} \hfill \hyperlink{up0044}{$\Uparrow$}
    
    (in {\color{mstitle}MS9: Total positivity})
        
        \mtskip
    Elementary, but very useful lemma due to Biernacki and Krzy\.{z} (1955) asserts that the ratio of two power series inherits monotonicity from that of the sequence of ratios of their respective  coefficients. Over the last two decades it has been realized that, under some additional assumptions, similar claims hold for more general ratios of series and integral
transforms as well as for unimodality in place of monotonicity. In the talk, we discuss conditions on the functional sequence and the kernel of an integral transform ensuring the preservation property. Numerous series and integral transforms appearing in applications satisfy our sufficient conditions, including Dirichlet, factorial (and $q$-factorial) series, inverse factorial series, Laplace, Mellin and generalized Stieltjes transforms, among many others.  We illustrate our results by ratios of generalized hypergeometric functions and Nuttall's $Q$ functions.  The key role in our considerations is played by the notion of sign regularity.

The talk is based on the the joint work with Anna Vishnyakova and Yi Zhang. 
\end{ilasabstract}
     \hypertarget{down0120}{}\begin{ilasabstract}
   \talktitle{nan}
    
    \textbf{Olga Katkova}, \info{10:30\textrm{--}11:00 @ SC0008 (June 24, Tuesday)} \hfill \hyperlink{up0120}{$\Uparrow$}
    
    (in {\color{mstitle}MS9: Total positivity})
        
        \mtskip
    nan\end{ilasabstract}
     \hypertarget{down0140}{}\begin{ilasabstract}
   \talktitle{Nonbacktracking random walks: mixing rate, Kemeny's constant, and beyond}
    
    \textbf{Mark Kempton}, \info{11:30\textrm{--}12:00 @ SC1005 (June 24, Tuesday)} \hfill \hyperlink{up0140}{$\Uparrow$}
    
    (in {\color{mstitle}MS15: Graphs and their eigenvalues: Celebrating the work of Fan Chung Graham})
        
        \mtskip
    Random walks on graphs are ubiquitous in spectral graph theory, both in helping us understand graphs and for applications in graph algorithms.  Considerable research has been done in the last two decades around understanding how forbidding backtracking affects the random walk.  We will discuss past and current research on how forbidding backtracking affects mixing rates, hitting times, Kemeny's constant, and other aspects of random walks, and we will discuss some future directions for exploration.
\end{ilasabstract}
     \hypertarget{down0138}{}\begin{ilasabstract}
   \talktitle{A geometric adaptation of the Chung-Lu graph model}
    
    \textbf{Franklin Kenter}, \info{10:30\textrm{--}11:00 @ SC1005 (June 24, Tuesday)} \hfill \hyperlink{up0138}{$\Uparrow$}
    
    (in {\color{mstitle}MS15: Graphs and their eigenvalues: Celebrating the work of Fan Chung Graham})
        
        \mtskip
    The Chung-Lu graph model specifies the expected degree of each vertex in a graph and, provided the degree distribution is not excessively skewed, generates a random graph where the existence of each edge is determined independently.

A common characteristic of many ``real-world'' graphs is a degree distribution that follows an inverse power law. Specifically, the number of vertices with degree $x$ is proportional to $x^\beta$, where $\beta$ typically ranges between 1 and 3. The Chung-Lu model offers a straightforward approach to capturing this power-law behavior in synthetic networks.

We extend the Chung-Lu model to random geometric graphs. In this extension, each vertex is assigned both an expected degree and a random position in Euclidean space according to a probability distribution. Once the vertices are placed, the realization of each edge occurs independently of others. We rigorously establish the conditions necessary to ensure that the assigned degrees align with the expected degrees. This geometric Chung-Lu model is tested on the connectome of the \emph{Drosophila} medulla (fruit fly), where the random model successfully replicates the graph-theoretical structure of the original network, including the eigenvalues of various graph-theoretic matrices.

This is joint work with Susama Agarwala.
\end{ilasabstract}
     \hypertarget{down0378}{}\begin{ilasabstract}
   \talktitle{Leaky forcing: extending zero forcing results to a fault-tolerant setting}
    
    \textbf{Franklin Kenter}, \info{17:30\textrm{--}18:00 @ SC1005 (June 26, Thursday)} \hfill \hyperlink{up0378}{$\Uparrow$}
    
    (in {\color{mstitle}MS17: Graphs and matrices in honor of Leslie Hogben's retirement})
        
        \mtskip
    We study a recent variation of zero forcing called leaky forcing.  Zero forcing is a propagation process on a network whereby some nodes are initially blue with all others white. Blue vertices can ``force'' a white neighbor to become blue if all other  neighbors are blue. The goal is to find the minimum number of initially blue vertices to eventually force all vertices blue after exhaustively applying the forcing rule above.
Leaky forcing is a fault-tolerant variation of zero forcing where certain vertices (not necessarily initially blue) cannot force. The goal in this context is to find the minimum number of initially blue vertices needed that can eventually force all vertices to be blue, {\it regardless} of which small number of vertices can't force.

This work extends results from zero forcing in terms of leaky forcing. New results regarding leaky forcing presented here include:
\begin{itemize}
\item Complete determination of all leaky forcing numbers for all unicyclic graphs.
\item Robust upper bounds for generalized Petersen graphs.
\item Bounds for the effect of both edge removal and vertex removal.
\item A complete characterization for which connected graphs have the maximum possible $1$-leaky forcing number (i.e., when $Z_{(1)}(G) = |V(G)|-1$).
\end{itemize}

This is joint work with 
Beth Bjorkman, Lei Cao, Ryan Moruzzi, Carolyn Reinhart and Violeta Vasilevska and is part of the AIM Mathematical Research Communities. 
\end{ilasabstract}
     \hypertarget{down0257}{}\begin{ilasabstract}
   \talktitle{Geometric means of HPD GLT matrix-sequences: structure, invertibility, and convergence}
    
    \textbf{Muhammad Faisal Khan}, \info{11:00\textrm{--}11:30 @ SC1003 (June 25, Wednesday)} \hfill \hyperlink{up0257}{$\Uparrow$}
    
    (in {\color{mstitle}MS19: Explicit and hidden asymptotic structures, GLT Analysis, and applications})
        
        \mtskip
    \begin{bibunit}
        In this work, we extend our previous analysis on the spectral distribution of the geometric mean of matrix-sequences formed by Hermitian Positive Definite (HPD) matrices, under the framework of Generalized Locally Toeplitz (GLT) $\ast$-algebra. Building on prior results~\cite{ahmad2025matrix}, we now investigate whether the assumption of invertibility of the GLT symbols almost everywhere is essential. Motivated by the fact that inversion is often required due to matrix non-commutativity, we explore the scenario where the symbols commute, aiming to relax the invertibility condition. Furthermore, we study the Karcher mean of more than two HPD GLT matrix-sequences, focusing on how an initial guess that itself belongs to the GLT algebra influences the convergence of the iterative computation. Numerical experiments support our theoretical claims and demonstrate improved convergence behavior under structured initialization.

Finally, we extend the theoretical results to the multilevel block case (for $r = 1$, $d \geq 1$), offering a a broader generalization and deeper numerical validation. 

\vspace{2em}

\begin{thebibliography}{99}

\bibitem{ahmad2025matrix}
Ahmad, D.; Khan, M.F.; Serra-Capizzano, S. \textit{Matrix-Sequences of Geometric Means in the Case of Hidden (Asymptotic) Structures.} Mathematics 2025, 13, 393.
\url{https://doi.org/10.3390/math13030393}.

\bibitem{barbarino2020a}
G.~Barbarino, C.~Garoni, and S.~Serra-Capizzano, \textit{Block generalized locally Toeplitz sequences: theory and applications in the unidimensional case},
Electronic Transactions on Numerical Analysis, 53 (2020), pp.~28--112.

\bibitem{barbarino2020b}
G.~Barbarino, C.~Garoni, and S.~Serra-Capizzano,
\textit{Block generalized locally Toeplitz sequences: theory and applications in the multidimensional case},
Electronic Transactions on Numerical Analysis, 53 (2020), pp.~113--216.

\bibitem{garoni2017}
C.~Garoni and S.~Serra-Capizzano,
\textit{Generalized Locally Toeplitz Sequences: Theory and Applications, Vol.~I},
Springer, Cham, 2017.

\bibitem{garoni2018}
C.~Garoni and S.~Serra-Capizzano,
\textit{Generalized Locally Toeplitz Sequences: Theory and Applications, Vol.~II},
Springer, Cham, 2018.


\end{thebibliography}
        \end{bibunit}
        \end{ilasabstract}
     \hypertarget{down0012}{}\begin{ilasabstract}
   \talktitle{nan}
    
    \textbf{Apoorva Khare}, \info{12:00\textrm{--}12:30 @ SC0008 (June 23, Monday)} \hfill \hyperlink{up0012}{$\Uparrow$}
    
    (in {\color{mstitle}MS9: Total positivity})
        
        \mtskip
    nan\end{ilasabstract}
     \hypertarget{down0136}{}\begin{ilasabstract}
   \talktitle{nan}
    
    \textbf{Kshitij Khare}, \info{11:00\textrm{--}11:30 @ SC1003 (June 24, Tuesday)} \hfill \hyperlink{up0136}{$\Uparrow$}
    
    (in {\color{mstitle}MS26: Utilizing structure to achieve low-complexity algorithms for data science, engineering, and physics})
        
        \mtskip
    nan\end{ilasabstract}
     \hypertarget{down0189}{}\begin{ilasabstract}
   \talktitle{nan}
    
    \textbf{Joe Kileel}, \info{13:30\textrm{--}14:00 @ SC3001 (June 24, Tuesday)} \hfill \hyperlink{up0189}{$\Uparrow$}
    
    (in {\color{mstitle}MS18: New methods in numerical multilinear algebra})
        
        \mtskip
    nan\end{ilasabstract}
     \hypertarget{down0023}{}\begin{ilasabstract}
   \talktitle{Quasi-Wasserstein mean of positive definite matrices}
    
    \textbf{Sejong Kim}, \info{11:30\textrm{--}12:00 @ SC1001 (June 23, Monday)} \hfill \hyperlink{up0023}{$\Uparrow$}
    
    (in {\color{mstitle}MS10: Matrix means and related topics})
        
        \mtskip
    The typical examples of Kubo-Ando's operator means are the weighted arithmetic, geometric, and harmonic means. In particular, they are interpolated by the power means (introduced by Lim and Palfia) monotonically in terms of the Loewner order. On the other hand, there are other important means of non-Kubo-Ando's operator means such as the weighted spectral geometric and Wasserstein means. We define quasi-Wasserstein means, which interpolate the weighted spectral geometric and Wasserstein means monotonically in terms of the near-order. We also study their properties including trace and norm inequalities.
\end{ilasabstract}
     \hypertarget{down0283}{}\begin{ilasabstract}
   \talktitle{nan}
    
    \textbf{Sooyeong Kim}, \info{10:30\textrm{--}11:00 @ SC0014 (June 26, Thursday)} \hfill \hyperlink{up0283}{$\Uparrow$}
    
    (in {\color{mstitle}MS8: Tensor and quantum information science})
        
        \mtskip
    nan\end{ilasabstract}
     \hypertarget{down0295}{}\begin{ilasabstract}
   \talktitle{A centrality measure for cut edges in undirected graphs
}
    
    \textbf{Steve Kirkland}, \info{10:30\textrm{--}11:00 @ SC2001 (June 26, Thursday)} \hfill \hyperlink{up0295}{$\Uparrow$}
    
    (in {\color{mstitle}MS2: Combinatorial matrix theory})
        
        \mtskip
    We consider an edge centrality measure, introduced by Altafini et al, which is based on Kemeny's constant for a connected undirected graph. We revisit that centrality measure for the case of cut edges, providing an intuitive  interpretation for it, and showing how it can be computed using tools from combinatorial matrix theory. Explicit expressions for the edge centralities are given for certain types of trees. Joint work with Dario Bini, Guy Latouche and Beatrice Meini.  
\end{ilasabstract}
     \hypertarget{down0135}{}\begin{ilasabstract}
   \talktitle{A low-complexity algorithm to search for Legendre pairs
}
    
    \textbf{Ilias Kotsireas}, \info{10:30\textrm{--}11:00 @ SC1003 (June 24, Tuesday)} \hfill \hyperlink{up0135}{$\Uparrow$}
    
    (in {\color{mstitle}MS26: Utilizing structure to achieve low-complexity algorithms for data science, engineering, and physics})
        
        \mtskip
    Legendre pairs constitute an important combinatorial object that can be used to construct Hadamard matrices. 
In our LAA paper, we studied the matrix equation of Legendre pairs and its properties, focusing on the spectra of the matrices involved, utilizing Gershgorin circles. Legendre pairs are characterized by two invariants related to the discrete Fourier transform (DFT) matrix. We propose a low-complexity fast Fourier transform (FFT)-like algorithm to compute the product of the DFT matrix with each sequence of the Legendre pair. By utilizing the FFT-like algorithm, we present a low-time complexity algorithm to search for Legendre pairs. We present numerical results from our C implementation of the FFT-like algorithm, which offers a lower time complexity for finding Legendre pairs compared to traditional combinatorial algorithms. 
\\\\
This is a joint work with Sirani M. Perera.
\end{ilasabstract}
     \hypertarget{down0352}{}\begin{ilasabstract}
   \talktitle{nan}
    
    \textbf{Hitesh Kumar}, \info{16:30\textrm{--}17:00 @ SC0008 (June 26, Thursday)} \hfill \hyperlink{up0352}{$\Uparrow$}
    
    (in {\color{mstitle}MS24: Nonnegative and related families of matrices})
        
        \mtskip
    nan\end{ilasabstract}
     \hypertarget{down0247}{}\begin{ilasabstract}
   \talktitle{On bijections which strongly preserve Birkhoff-James orthogonality on finite-dimensional $C^*$-algebras}
    
    \textbf{Bojan Kuzma}, \info{10:30\textrm{--}11:00 @ SC0012 (June 25, Wednesday)} \hfill \hyperlink{up0247}{$\Uparrow$}
    
    (in {\color{mstitle}MS35: Preserver Problems, II})
        
        \mtskip
    We classify bijections which in both directions preserve Birkhoff-James orthogonality on finite-dimensional $C^*$-algebras. It turns out that such maps are  real-linear isometries  multiplied by a  central-valued (possibly nonlinear) function. The result differs from smooth normed spaces, where every such preserver is a (conjugate)linear isometry multiplied by a  scalar-valued function.
Our main technique  is using left symmetric elements, relative to Birkhoff-James orthogonality, of certain subspaces. 

This is a joint work With Srdjan Stefanovi\'c
\end{ilasabstract}
     \hypertarget{down0321}{}\begin{ilasabstract}
   \talktitle{nan}
    
    \textbf{Patrick Kürschner}, \info{14:30\textrm{--}15:00 @ SC0014 (June 26, Thursday)} \hfill \hyperlink{up0321}{$\Uparrow$}
    
    (in {\color{mstitle}MS5: Advances in matrix equations: Theory, computations, and applications})
        
        \mtskip
    nan\end{ilasabstract}
     \hypertarget{down0285}{}\begin{ilasabstract}
   \talktitle{nan}
    
    \textbf{Zehua Lai}, \info{11:30\textrm{--}12:00 @ SC0014 (June 26, Thursday)} \hfill \hyperlink{up0285}{$\Uparrow$}
    
    (in {\color{mstitle}MS8: Tensor and quantum information science})
        
        \mtskip
    nan\end{ilasabstract}
     \hypertarget{down0223}{}\begin{ilasabstract}
   \talktitle{Inverse eigenvalue problem for discrete Schrödinger operators of a graph}
    
    \textbf{Anzila Laikhuram}, \info{17:00\textrm{--}17:30 @ SC1005 (June 24, Tuesday)} \hfill \hyperlink{up0223}{$\Uparrow$}
    
    (in {\color{mstitle}MS17: Graphs and matrices in honor of Leslie Hogben's retirement})
        
        \mtskip
    The Inverse Eigenvalue Problem of a Graph (IEPG) determines whether a set of real numbers can be realized as the spectrum of a real symmetric matrix in $\mathcal{S}(G)$, where $\mathcal{S}(G)$ consists of symmetric matrices associated with a graph $G$, having non-zero off-diagonal entries only for adjacent vertices. Extending this, the Inverse Eigenvalue Problem for discrete Schr\"odinger operators investigates if a given set of real numbers can be the spectrum of a matrix in $\ddot{\mathcal{S}}(G)$, where off-diagonal entries are negative for edges and zero otherwise, with unrestricted diagonal entries. We begin with connected graphs on 4 vertices. For a list of real numbers $\lambda_1 < \lambda_2 \leq \lambda_3 \leq \lambda_4$, we determine if there exists a matrix in $\ddot{\mathcal{S}}(G)$ with spectrum $\{\lambda_1, \lambda_2, \lambda_3, \lambda_4\}$ and the smallest eigenvalue simple. We identify feasible ordered multiplicity lists, noting that not all are valid.   Our investigation extends to families of graphs such as paths ($P_n$), and complete graphs ($K_n$). For $P_n$, any set of $n$ distinct real numbers is realizable as a spectrum in $\ddot{\mathcal{S}}(P_n)$. For $K_n$, any ordered list $\lambda_1 < \lambda_2 \leq \dots \leq \lambda_n$ is achievable. \end{ilasabstract}
     \hypertarget{down0225}{}\begin{ilasabstract}
   \talktitle{Domino tilings, domino shuffling, and the nabla operator}
    
    \textbf{Yi-Lin Lee}, \info{16:00\textrm{--}16:30 @ SC2001 (June 24, Tuesday)} \hfill \hyperlink{up0225}{$\Uparrow$}
    
    (in {\color{mstitle}MS25: Enumerative/algebraic combinatorics and matrices})
        
        \mtskip
    In this talk, I will present a $q,t$-generalization of domino tilings of certain regions $R_\lambda$, indexed by partitions $\lambda$, weighted according to generalized area and dinv statistics. These statistics arise from the $q,t$-Catalan combinatorics and Macdonald polynomials. We present a formula for the generating polynomial of these domino tilings in terms of the Bergeron--Garsia nabla operator. When $\lambda = (n^n)$ is a square shape, domino tilings of $R_\lambda$ are equivalent to those of the Aztec diamond of order $n$. In this case, we give a new product formula for the resulting polynomials by domino shuffling and its connection with alternating sign matrices. In particular, we obtain a combinatorial proof of the joint symmetry of the generalized area and dinv statistics. This is based on joint work with Ian Cavey.\end{ilasabstract}
     \hypertarget{down0265}{}\begin{ilasabstract}
   \talktitle{$J$-selfadoint matrix means and their indefinite inequalities}
    
    \textbf{Rute Lemos}, \info{10:30\textrm{--}11:00 @ SC2006 (June 25, Wednesday)} \hfill \hyperlink{up0265}{$\Uparrow$}
    
    (in {\color{mstitle}MS3: Matrix inequalities with applications})
        
        \mtskip
    Consider the indefinite inner product induced by a non trivial involutive Hermitian matrix $J$, 
which endows the matrix algebra of $n$-square complex matrices with a partial order relation between $J$-selfadjoint matrices.
Indefinite inequalities are given in this setup, involving the $J$-selfadjoint $\alpha$-weighted geometric matrix mean. 
In particular, an indefinite version of Ando–Hiai inequality is proved to be equivalent to Furuta inequality of indefinite type.

This talk is based on a joint work with Nat\'alia Bebiano and Gra\c ca Soares.

This work is supported by the
Center for Research and Development in Mathematics and Applications (CIDMA) under the
Portuguese Foundation for Science and Technology 
(FCT, https://ror.org/00snfqn58)   
Multi-Annual Financing Program for R\&D Units.


%Linear algebra is a fundamental branch of mathematics that underpins numerous scientific and technological advancements. It provides essential tools for solving systems of equations, transforming geometric spaces, and analyzing data. With applications in engineering, computer science, physics, economics, and machine learning, linear algebra plays a crucial role in modern innovations, from image processing to artificial intelligence. 
\end{ilasabstract}
     \hypertarget{down0147}{}\begin{ilasabstract}
   \talktitle{nan}
    
    \textbf{Gilad Lerman}, \info{10:30\textrm{--}11:00 @ SC3001 (June 24, Tuesday)} \hfill \hyperlink{up0147}{$\Uparrow$}
    
    (in {\color{mstitle}MS18: New methods in numerical multilinear algebra})
        
        \mtskip
    nan\end{ilasabstract}
     \hypertarget{down0054}{}\begin{ilasabstract}
   \talktitle{nan}
    
    \textbf{Chi-Kwong Li}, \info{15:00\textrm{--}15:30 @ SC0014 (June 23, Monday)} \hfill \hyperlink{up0054}{$\Uparrow$}
    
    (in {\color{mstitle}MS8: Tensor and quantum information science})
        
        \mtskip
    nan\end{ilasabstract}
     \hypertarget{down0195}{}\begin{ilasabstract}
   \talktitle{Reduced Krylov basis methods}
    
    \textbf{Yuwen Li}, \info{14:30\textrm{--}15:00 @ SC4011 (June 24, Tuesday)} \hfill \hyperlink{up0195}{$\Uparrow$}
    
    (in {\color{mstitle}MS23: Advances in Krylov subspace methods and their applications})
        
        \mtskip
    The reduced basis method is the dominating numerical solver for a family of parametrized PDEs. In this talk, I will present our new reduced basis algorithm based on preconditioned Krylov subspace methods such as the conjugate gradient method, generalized minimum residual method, and bi-conjugate gradient method. The proposed methods use a preconditioned Krylov subspace method for a high-fidelity discretization of one parameter instance to generate orthogonal basis vectors of the reduced basis subspace. Then the family of large-scale discrete parameter-dependent problems are approximately solved in the low-dimensional Krylov subspace. The material in my talk is based on joint works with Ludmil Zikatanov and Cheng Zuo.
\end{ilasabstract}
     \hypertarget{down0236}{}\begin{ilasabstract}
   \talktitle{nan}
    
    \textbf{Yung-Ta Li}, \info{17:30\textrm{--}18:00 @ SC3001 (June 24, Tuesday)} \hfill \hyperlink{up0236}{$\Uparrow$}
    
    (in {\color{mstitle}MS11: Structured matrix computations and its applications})
        
        \mtskip
    nan\end{ilasabstract}
     \hypertarget{down0267}{}\begin{ilasabstract}
   \talktitle{nan}
    
    \textbf{Zhongshan Jason Li}, \info{11:30\textrm{--}12:00 @ SC2006 (June 25, Wednesday)} \hfill \hyperlink{up0267}{$\Uparrow$}
    
    (in {\color{mstitle}MS3: Matrix inequalities with applications})
        
        \mtskip
    nan\end{ilasabstract}
     \hypertarget{down0346}{}\begin{ilasabstract}
   \talktitle{nan}
    
    \textbf{Zhao Li}, \info{15:00\textrm{--}15:30 @ SC3001 (June 26, Thursday)} \hfill \hyperlink{up0346}{$\Uparrow$}
    
    (in {\color{mstitle}MS16: Approximations and errors in Krylov-based solvers})
        
        \mtskip
    nan\end{ilasabstract}
     \hypertarget{down0071}{}\begin{ilasabstract}
   \talktitle{nan}
    
    \textbf{Matthew M. Lin}, \info{14:30\textrm{--}15:00 @ SC3001 (June 23, Monday)} \hfill \hyperlink{up0071}{$\Uparrow$}
    
    (in {\color{mstitle}MS1: Embracing new opportunities in numerical linear algebra})
        
        \mtskip
    nan\end{ilasabstract}
     \hypertarget{down0072}{}\begin{ilasabstract}
   \talktitle{Fast SDDRE-based maneuvering-target interception at prespecified orientation}
    
    \textbf{Li-Gang Lin}, \info{15:00\textrm{--}15:30 @ SC3001 (June 23, Monday)} \hfill \hyperlink{up0072}{$\Uparrow$}
    
    (in {\color{mstitle}MS1: Embracing new opportunities in numerical linear algebra})
        
        \mtskip
    This talk considers the 3-D guidance law based on target lead angle information and the state-dependent differential Riccati equation (SDDRE) scheme. In an application-oriented manner, it presents theories to significantly improve critical computational performance and thus aims at a fast implementation for impact-angle-constrained interception of agile maneuvering targets. More specifically, regarding the two major computational burdens using SDDRE, we have replaced the burden in numerical applicability checking by a simple, equivalent, and closed-form condition for the entire state space, which is actually the dominant burden as supported by complexity analysis and extensive validations. Notably, the proposed analysis not only complements the early findings of applicability guarantee in literature, but also promotes the efficiency of the proposed philosophy when compared to the classic method, where the latter has caused concerns/reservations due to its feasibility and difficulty. On the other hand, we have largely mitigated the second major burden of SDDRE by -- after exhaustive trials -- selecting the most efficient Riccati-equation solver until the latest benchmarks. Such evaluations are: 1) in favor of a much-less-known achievement, rather than the common QR-based benchmark and 2) subject to both numerical and hardware experiments including, notably, implementations on microcontrollers and field-programmable gate arrays.
\end{ilasabstract}
     \hypertarget{down0074}{}\begin{ilasabstract}
   \talktitle{Distributed $t$-SNE}
    
    \textbf{Szu-Han Lin}, \info{14:30\textrm{--}15:00 @ SC4011 (June 23, Monday)} \hfill \hyperlink{up0074}{$\Uparrow$}
    
    (in {\color{mstitle}MS20: Manifold learning and statistical applications})
        
        \mtskip
    The $t$-distributed stochastic neighbor embedding method  ($t$-SNE, Maaten
and Hinton, 2008) has gained popularity for data exploration, particularly for its highly effective visualization of high-dimensional data, offering valuable insights before analysis. However, with a computational complexity of $O(n^2)$, its applicability to large datasets is limited. In practical applications, the Barnes-Hut $t$-SNE  (BH $t$-SNE, Maaten, 2014), a $t$-SNE variant with computational complexity $O(n\log n)$, is commonly employed for large datasets due to its high computational efficiency. In this work, we propose a divide-and-conquer approach that further reduces the computational complexity to $O(n)$, significantly lowering both computational time and memory usage by processing only subsets of the data at a time. Implementing the divide-and-conquer approach requires careful parameter adjustments to ensure asymptotic equivalence to the original $t$-SNE. We provide theoretical proof of this convergence and support our findings with simulation studies on the MNIST dataset. In summary, this work offers a scalable solution for applying $t$-SNE to extremely large datasets, maintaining its consistency and efficiency.
\end{ilasabstract}
     \hypertarget{down0221}{}\begin{ilasabstract}
   \talktitle{Inverse fiedler vector problem of a graph
}
    
    \textbf{Jephian C.-H. Lin}, \info{16:00\textrm{--}16:30 @ SC1005 (June 24, Tuesday)} \hfill \hyperlink{up0221}{$\Uparrow$}
    
    (in {\color{mstitle}MS17: Graphs and matrices in honor of Leslie Hogben's retirement})
        
        \mtskip
    Given a graph and one of its weighted Laplacian matrix, a Fiedler vector is an eigenvector with respect to the second smallest eigenvalue. The Fiedler vectors have been used widely for graph partitioning, graph drawing, spectral clustering, and suggesting the center of a network.  The inverse Fiedler vector problem studies the possible Fiedler vectors for different weighted Laplacian matrices of a given graph.  For a given tree, we characterize all possible Fiedler vectors among its weighted Laplacian matrix.  As an application, the characteristic set can be anywhere on a tree, except for the set containing a single leaf.  For a given cycle, we characterize all possible eigenvectors corresponding to the second or the third smallest eigenvalue.
\end{ilasabstract}
     \hypertarget{down0261}{}\begin{ilasabstract}
   \talktitle{Engaging students with collaborative tasks in Linear Algebra
}
    
    \textbf{Jephian C.-H. Lin}, \info{11:30\textrm{--}12:00 @ SC1005 (June 25, Wednesday)} \hfill \hyperlink{up0261}{$\Uparrow$}
    
    (in {\color{mstitle}MS27: Linear algebra education})
        
        \mtskip
    The main challenge in teaching linear algebra, or mathematics in general, in Taiwan is the lack of motivation. Many students have been conditioned to view learning as solely a means to achieve good exam scores, while higher education should focus on more than just grades. Hands-on activities can help spark students' interest in learning. In this talk, we will present several collaborative tasks that encourage students to work and learn together.
\end{ilasabstract}
     \hypertarget{down0324}{}\begin{ilasabstract}
   \talktitle{nan}
    
    \textbf{Yeh-Chi Lin}, \info{14:00\textrm{--}14:30 @ SC1001 (June 26, Thursday)} \hfill \hyperlink{up0324}{$\Uparrow$}
    
    (in {\color{mstitle}MS34: Combinatorics, association scheme, and graphs})
        
        \mtskip
    nan\end{ilasabstract}
     \hypertarget{down0271}{}\begin{ilasabstract}
   \talktitle{Nonnegative rank-2 approximations -- on choosing a starting point for ANLS}
    
    \textbf{Etna Lindy}, \info{10:30\textrm{--}11:00 @ SC4011 (June 25, Wednesday)} \hfill \hyperlink{up0271}{$\Uparrow$}
    
    (in {\color{mstitle}MS1: Embracing new opportunities in numerical linear algebra})
        
        \mtskip
    \begin{bibunit}
        Given a nonnegative $m \times n$ matrix $X$, its nonnegative rank-$r$ approximation (NMF-$r$) consists of two nonnegative matrices, $A $ and $B$ of sizes $m\times r$ and $n\times r$, such that the Frobenius norm of $X - AB^\top$ is minimized. The rank-2 case is special in the sense that while the problem is NP-hard for general $r$ and somewhat trivial in the case of $r=1$, the complexity of rank-2 NMF is not known. Furthermore, NMF-2 is equivalent to rank-2 optimization with a non-negativity constraint, which is not true for larger $r$.
NMF-2 is commonly approached by solving for the components $A$ and $B$ separately in an alternating fashion (ANLS). This means solving several independent nonnegative least squares problems, which can be done exactly with relatively low computational cost in the rank-2 case. The ANLS method is not guaranteed to converge to the global minimum, and choosing a good starting point is crucial. Specifically, the ANLS method often seems to converge to a trivial solution with a column of zeros when the starting point is chosen carelessly. 
We suggest using a certain angular coordinate representation of the matrices $A$ and $B$ and then optimizing over these new sets of coordinates.
This approach has the benefit of introducing only a few zeros, which is a good property for a starting point for ANLS.
The tests we have implemented suggest that our method gives a better approximation than other existing methods such as SPA \cite{gillis2014hierarchical} while maintaining the same level of computational complexity.
Ideally the method could be generalized for $r > 2$, but it is still under consideration how this could be done.

\begin{thebibliography}{9}
\bibitem{gillis2014hierarchical}
N. Gillis, D. Kuang, H. Park, ''Hierarchical clustering of hyperspectral images using rank-two nonnegative matrix factorization'', {\it IEEE Transactions on Geoscience and
Remote Sensing} (2014)
\end{thebibliography}
        \end{bibunit}
        \end{ilasabstract}
     \hypertarget{down0067}{}\begin{ilasabstract}
   \talktitle{Spherical volume-preserving parameterization via energy minimization}
    
    \textbf{Shu-Yung Liu}, \info{14:00\textrm{--}14:30 @ SC2006 (June 23, Monday)} \hfill \hyperlink{up0067}{$\Uparrow$}
    
    (in {\color{mstitle}MS22: Linear algebra applications in computational geometry})
        
        \mtskip
    For a simplicial $3$-manifold, such as a tetrahedral mesh sampled from a solid brain, a spherical volume-preserving parameterization is a bijective mapping onto the unit solid sphere. In this presentation, we introduce a novel energy functional to measure the volume distortion of such mappings and propose an associated minimization method to obtain volume-preserving parameterizations. Our method is theoretically guaranteed to converge globally and demonstrates improved effectiveness compared to a state-of-the-art method. Finally, we present its application in brain imaging, showcasing its real-world utility.\end{ilasabstract}
     \hypertarget{down0235}{}\begin{ilasabstract}
   \talktitle{nan}
    
    \textbf{Ching-Sung Liu}, \info{17:00\textrm{--}17:30 @ SC3001 (June 24, Tuesday)} \hfill \hyperlink{up0235}{$\Uparrow$}
    
    (in {\color{mstitle}MS11: Structured matrix computations and its applications})
        
        \mtskip
    nan\end{ilasabstract}
     \hypertarget{down0287}{}\begin{ilasabstract}
   \talktitle{nan}
    
    \textbf{Chia-An Liu}, \info{11:00\textrm{--}11:30 @ SC1001 (June 26, Thursday)} \hfill \hyperlink{up0287}{$\Uparrow$}
    
    (in {\color{mstitle}MS34: Combinatorics, association scheme, and graphs})
        
        \mtskip
    nan\end{ilasabstract}
     \hypertarget{down0303}{}\begin{ilasabstract}
   \talktitle{nan}
    
    \textbf{Xiaobo Liu}, \info{11:30\textrm{--}12:00 @ SC3001 (June 26, Thursday)} \hfill \hyperlink{up0303}{$\Uparrow$}
    
    (in {\color{mstitle}MS16: Approximations and errors in Krylov-based solvers})
        
        \mtskip
    nan\end{ilasabstract}
     \hypertarget{down0367}{}\begin{ilasabstract}
   \talktitle{nan}
    
    \textbf{Yuan-Hsun Lo}, \info{16:00\textrm{--}16:30 @ SC1001 (June 26, Thursday)} \hfill \hyperlink{up0367}{$\Uparrow$}
    
    (in {\color{mstitle}MS25: Enumerative/algebraic combinatorics and matrices})
        
        \mtskip
    nan\end{ilasabstract}
     \hypertarget{down0354}{}\begin{ilasabstract}
   \talktitle{On cones of polynomials preserving nonnegative matrices}
    
    \textbf{Raphael Loewy}, \info{17:30\textrm{--}18:00 @ SC0008 (June 26, Thursday)} \hfill \hyperlink{up0354}{$\Uparrow$}
    
    (in {\color{mstitle}MS24: Nonnegative and related families of matrices})
        
        \mtskip
    We consider polynomials preserving nonnegative matrices. Let $n$ be a positive integer and

\begin{equation*}
{\mathcal{P}}_{n}=\{p \in {{\mathbb{R}}}[x]: p(A)\geq 0 \mbox{ for all~}  A \geq 0, A \in {\mathbb{R}}^{n,n}\}.
\end{equation*}
${\mathcal{P}}_{n}$ was defined by Loewy and London, motivated by the Nonnegative Inverse Eigenvalue Problem (NIEP), but is of independent interest. Clearly, ${\mathcal{P}}_{n}$ is a closed, convex cone.

\medskip
Given any polynomial $p$, we identify $p$ with its sequence of coefficients. In order to consider only finite dimensional vector spaces, we restrict the degree of the polynomials. Given a positive integer $m$, define

\begin{equation*}
{\mathcal{P}}_{n,m}=\{p \in {\mathcal{P}}_{n} : degree(p) \leq m \}.
\end{equation*}
Then, ${\mathcal{P}}_{n,m}$ can be thought as a cone in ${\mathbb{R}}^{m+1}$.

\medskip
It is clear that any polynomial with nonnegative coefficients is in ${\mathcal{P}}_{n}$. Therefore, the number, relative size and distribution of the negative coefficients of polynomials in ${\mathcal{P}}_{n}$ are of significant interest. Clark and Paparella showed that if $p \in {\mathcal{P}}_{n}$, then its first and last $n$ coefficients must be nonnegative. Hence, ${\mathcal{P}}_{n,m}$ is a simplicial cone, for any $0 < m <2n$.

\medskip
Let $m \geq 2n$. Then, ${\mathcal{P}}_{n,m}$ contains polynomials with negative coefficients. For example, there exists $a > 0$ such that $1+x+x^{2}+\cdots+x^{n-1}-ax^{n}+x^{n+1}+\cdots+x^{2n} \in {\mathcal{P}}_{n,2n}$ (the question of the optimal $a$ is of interest).
It follows that the structure of ${\mathcal{P}}_{n,m}$ is nontrivial. We consider its face structure, and in particular the one dimensional faces, that is, the extreme rays. Using some information on the possible coefficients of polynomials in  ${\mathcal{P}}_{n,m}$, we show that  ${\mathcal{P}}_{n,m}$ is not polyhedral, that is, contains infinitely many extreme rays. Additional preliminary results on the faces are obtained.\end{ilasabstract}
     \hypertarget{down0309}{}\begin{ilasabstract}
   \talktitle{Product of two involutions in special linear groups}
    
    \textbf{Tejbir Lohan}, \info{14:30\textrm{--}15:00 @ SC0008 (June 26, Thursday)} \hfill \hyperlink{up0309}{$\Uparrow$}
    
    (in {\color{mstitle}MS31: Matrix decompositions and applications})
        
        \mtskip
    An element of a group is called an involution if its square equals the identity element. Decomposing a group element into a product of involutions has applications in various areas of mathematics, with a particular focus on elements that can be expressed as the product of two involutions, known as \textit{strongly reversible}, \textit{strongly real}, or \textit{bireflectional} elements. Classifying strongly reversible elements in a group is a problem of broad interest. It is known that an element of the general linear group over a field is strongly reversible if and only if it is similar to its inverse. However, this result does not hold for special linear groups over a field or a division ring. In this talk, we will use the notion of reversibility to classify the strongly reversible elements in the complex special linear group and the quaternionic special linear group. This talk is based on joint work with Krishnendu Gongopadhyay and Chandan Maity.
\end{ilasabstract}
     \hypertarget{down0173}{}\begin{ilasabstract}
   \talktitle{Spectral analysis, approximation, and preconditioning for block structured matrix-sequences}
    
    \textbf{Valerio Loi}, \info{13:30\textrm{--}14:00 @ SC1003 (June 24, Tuesday)} \hfill \hyperlink{up0173}{$\Uparrow$}
    
    (in {\color{mstitle}MS19: Explicit and hidden asymptotic structures, GLT Analysis, and applications})
        
        \mtskip
    Large block-structured matrices with Toeplitz-type blocks of different sizes frequently arise in various applications, but pose computational issues when solving the associated linear systems. In our setting, the matrices \(A_n\) are composed of (block rectangular) Toeplitz blocks defined by rectangular \(s \times t\) matrix-valued generating functions, and can be viewed as a generalization of classical GLT (Generalized Locally Toeplitz) sequences. Under mild assumptions on the block dimensions, the asymptotic distribution of the singular values of the associated matrix sequences is recently known. Moreover, when the singular value symbol is Hermitian, the spectral symbol coincides with the singular value symbol. Starting from the tools used to determine this singular value distribution, we develop a general preconditioning framework to construct simplified block matrices that approximate the original matrices. These simplified matrices offer two key advantages:\\
1. They maintain the same singular value distributions as \(\{A_n\}_{n}\); \\
2. They enable the solution of linear systems in \(\mathcal{O}(n \log n)\) arithmetic operations.\\
In this way, we propose a natural preconditioning strategy for linear systems with coefficient matrix \(A_n\). We provide detailed singular value and spectral analyses of the preconditioned matrix sequences and validate our approach through numerical experiments concerning the convergence of various (preconditioned) Krylov solvers.
\end{ilasabstract}
     \hypertarget{down0028}{}\begin{ilasabstract}
   \talktitle{Cospectral constructions for the generalized distance matrix}
    
    \textbf{Kate Lorenzen}, \info{11:00\textrm{--}11:30 @ SC1005 (June 23, Monday)} \hfill \hyperlink{up0028}{$\Uparrow$}
    
    (in {\color{mstitle}MS15: Graphs and their eigenvalues: Celebrating the work of Fan Chung Graham})
        
        \mtskip
    The generalized distance matrix of a graph is a matrix in which every entry is function of the distance between the vertices. With special choices of the function, the generalized distance matrix family includes the adjacency matrix and distance matrix. Surprisingly, some pairs of graph are cospectral independent of the choice of function.  We present a construction that builds on Godsil-McKay Switching to produce cospectral pairs for the generalized distance matrix connecting cospectral constructions for many different graph matrices.
\end{ilasabstract}
     \hypertarget{down0188}{}\begin{ilasabstract}
   \talktitle{nan}
    
    \textbf{Xiang Lu}, \info{15:00\textrm{--}15:30 @ SC2006 (June 24, Tuesday)} \hfill \hyperlink{up0188}{$\Uparrow$}
    
    (in {\color{mstitle}MS32: Advances in matrix manifold optimization})
        
        \mtskip
    nan\end{ilasabstract}
     \hypertarget{down0206}{}\begin{ilasabstract}
   \talktitle{nan}
    
    \textbf{Bing-Ze Lu}, \info{16:30\textrm{--}17:00 @ SC0012 (June 24, Tuesday)} \hfill \hyperlink{up0206}{$\Uparrow$}
    
    (in {\color{mstitle}MS7: Linear algebra and quantum information science})
        
        \mtskip
    nan\end{ilasabstract}
     \hypertarget{down0348}{}\begin{ilasabstract}
   \talktitle{Uncertainty principle of condition number}
    
    \textbf{Tzon-Tzer Lu}, \info{14:00\textrm{--}14:30 @ SC4011 (June 26, Thursday)} \hfill \hyperlink{up0348}{$\Uparrow$}
    
    (in {\color{mstitle}MS1: Embracing new opportunities in numerical linear algebra})
        
        \mtskip
    Schaback in 1995 has proved the Uncertainty Principle of radial basis function interpolation, which states that the condition number and the error cannot be both kept small at the same time. Hence it is a Trade-off Principle. It seems to violet our cognition about ill/well-conditioned problems. In this talk we like to extend this principle, i.e. there is no case where the error and the condition number are both small, from interpolation matrices to arbitrary linear systems. The underlying theory behind the conflict between accuracy and stability flips our misconceptions about condition numbers and provides us with a brand-new interpretation of it. 
\end{ilasabstract}
     \hypertarget{down0368}{}\begin{ilasabstract}
   \talktitle{Ternary circulant almost orthogonal arrays with (near) D-optimality and good binary sequence pairs}
    
    \textbf{Xiao-Nan Lu}, \info{16:30\textrm{--}17:00 @ SC1001 (June 26, Thursday)} \hfill \hyperlink{up0368}{$\Uparrow$}
    
    (in {\color{mstitle}MS25: Enumerative/algebraic combinatorics and matrices})
        
        \mtskip
    Circulant almost orthogonal arrays (CAOAs) are a type of circulant matrices used in the statistical design of fMRI experiments. D-optimality, which maximizes the determinant of the information matrix of a design, plays a key role in ensuring efficiency of experiments.
This talk will focus on ternary CAOAs with strength $2$, emphasizing the characterization of D-optimal and near D-optimal CAOAs and investigating their relations with binary sequence pairs with good correlation properties.
\end{ilasabstract}
     \hypertarget{down0374}{}\begin{ilasabstract}
   \talktitle{nan}
    
    \textbf{Ding Lu}, \info{17:30\textrm{--}18:00 @ SC1003 (June 26, Thursday)} \hfill \hyperlink{up0374}{$\Uparrow$}
    
    (in {\color{mstitle}MS14: Pencils, polynomial, and rational matrices})
        
        \mtskip
    nan\end{ilasabstract}
     \hypertarget{down0035}{}\begin{ilasabstract}
   \talktitle{Density-equalizing quasiconformal surface and volmeteric parameterization}
    
    \textbf{Ronald Lok Ming Lui}, \info{11:30\textrm{--}12:00 @ SC2006 (June 23, Monday)} \hfill \hyperlink{up0035}{$\Uparrow$}
    
    (in {\color{mstitle}MS22: Linear algebra applications in computational geometry})
        
        \mtskip
    This talk explores various methods for computing bijective density-equalizing quasiconformal mappings for both surface and volumetric parameterizations. The primary objective is to achieve parameterizations of geometric shapes—whether 2D surfaces or 3D volumes—that minimize local geometric distortion while adhering to a prescribed density distribution of vertices. The density diffusion process is modeled as a quasiconformal flow, enabling effective control over local geometric distortions and ensuring mapping bijectivity. The talk will cover the underlying numerical algorithms and showcase experimental results. This work is supported by the Hong Kong Research Grants Council General Research Fund (Project ID: 14310224).
\end{ilasabstract}
     \hypertarget{down0137}{}\begin{ilasabstract}
   \talktitle{A low-complexity structured neural network approach for dynamical systems}
    
    \textbf{Sirani M. Perera}, \info{11:30\textrm{--}12:00 @ SC1003 (June 24, Tuesday)} \hfill \hyperlink{up0137}{$\Uparrow$}
    
    (in {\color{mstitle}MS26: Utilizing structure to achieve low-complexity algorithms for data science, engineering, and physics})
        
        \mtskip
    Data-driven learning is advancing rapidly, offering a new perspective on understanding dynamic systems. On the other hand, traditional methods for solving chaotic and highly non-linear systems face challenges in computational efficiency due to their inherent complexity and dynamics. Fortunately, neural networks excel in solving highly non-linear systems, showing exceptional performance.  
\\\\
In this talk, we propose a neural network approach to update the states of dynamical systems through a structured operator known as a Hankel operator -- an operator characterized by a Hankel structure. Our goal is to develop an optimal, low-complexity learning algorithm that utilizes time-delay measurements to forecast future states effectively. By the conclusion of the talk, we will demonstrate how this operator can be employed to model state-space dynamical systems, enabling predictions and insights into future dynamics compared to conventional techniques: SINDy and HAVOK followed by the feedforward neural networks.
\\\\
This is a joint work with Hansaka Aluvihare, Levi Lingsch, and Xianqi Li. This work was funded by the Division of Mathematical Sciences at the National Science Foundation with the award numbers 2410676, 2410677, \& 2410678.
\end{ilasabstract}
     \hypertarget{down0038}{}\begin{ilasabstract}
   \talktitle{Stochastic iterative methods for solving tensor linear systems}
    
    \textbf{Anna Ma}, \info{11:30\textrm{--}12:00 @ SC3001 (June 23, Monday)} \hfill \hyperlink{up0038}{$\Uparrow$}
    
    (in {\color{mstitle}MS18: New methods in numerical multilinear algebra})
        
        \mtskip
    Solving linear systems is a crucial subroutine and challenge in data science and scientific computing. Classical approaches for solving linear systems assume that data is readily available and small enough to be stored in memory. However, in the large-scale data setting, data may be so large that only partitions (e.g., single rows/columns of the matrix/tensor) can be utilized at a time. In this presentation, we discuss the advantages and role of randomization in iterative methods for approximating the solution to large-scale linear systems. Time permitting, we will also discuss our recent work on applications to solving systems involving higher-dimensional arrays, or tensors. Unlike previously proposed randomized iterative strategies, such as the tensor randomized Kaczmarz method (row slice method) or the tensor Gauss-Seidel method (column slice method), which are natural extensions of their matrix counterparts, our approach delves into a distinct scenario utilizing frontal slice sketching.
\end{ilasabstract}
     \hypertarget{down0055}{}\begin{ilasabstract}
   \talktitle{Spherical triangular configurations with invariant geometric mean}
    
    \textbf{Luís Machado}, \info{14:00\textrm{--}14:30 @ SC1001 (June 23, Monday)} \hfill \hyperlink{up0055}{$\Uparrow$}
    
    (in {\color{mstitle}MS10: Matrix means and related topics})
        
        \mtskip
    The goal is to characterize all configurations of three distinct points on a finite-dimensional Riemannian manifold that share the same geometric mean and to develop efficient computation methods to obtain such configurations. The geometric mean typically minimizes the sum of squared geodesic distances to the data points. This approach has been applied to various manifolds, such as the \(n\)-sphere \(S^n\), the orthogonal group, hyperbolic space, and the cone of positive symmetric matrices.  \\
To keep the scope manageable, we focus on the standard sphere \(S^2\) in \(\mathbb{R}^3\) with three points, introducing new ideas beyond minimizing squared geodesic distances. These ideas also emerge from known formulas for the mean of points forming regular geodesic polygons, such as equilateral geodesic triangles. As an initial step, we apply this approach to Euclidean spaces. \\
Theoretical results are supported by numerical experiments and illustrated with meaningful plots.\\ \\
\end{ilasabstract}
     \hypertarget{down0214}{}\begin{ilasabstract}
   \talktitle{Sign characteristic in the inverse problem for Hermitian matrix polynomials}
    
    \textbf{Steve Mackey}, \info{16:30\textrm{--}17:00 @ SC1001 (June 24, Tuesday)} \hfill \hyperlink{up0214}{$\Uparrow$}
    
    (in {\color{mstitle}MS14: Pencils, polynomial, and rational matrices})
        
        \mtskip
    The sign characteristic is a structural feature of Hermitian matrix polynomials
that is important for both theory and applications. 
It consists of a plus or minus sign associated to each elementary divisor
corresponding to a real or infinite eigenvalue;
these $\pm$ signs are invariants under unimodular congruence.
Motivated by the generic eigenstructure problem
for Hermitian matrix polynomials,
it is natural to consider the special scenario 
when all eigenvalues are simple.
In this case these signs can be naturally ordered to form a \emph{sign sequence}. 
 For an $n \times n$ Hermitian polynomial of degree $d$
with $dn$ simple real eigenvalues, for example, 
there are $2^{dn}$ possible sign sequences.
However, most of these sign sequences 
cannot be realized by any degree $d$ \,Hermitian polynomial. 
Is it possible to characterize exactly which sign sequences 
are realizable and which are not? 
And does the degree play any role in the story?

This talk completely settles these questions, 
discussing several new constraints on signs 
beyond the well-known signature constraint (1),
clarifying the dichotomy between even and odd degrees 
in the characterization,
as well as describing an underlying group of symmetries 
on the collection of all sign sequences 
that sheds light on the characterization question. 
In addition,
this characterization of realizable sign sequences enables a complete solution 
of the inverse problem for Hermitian matrix polynomials of all degrees, 
albeit only in the generic scenario when all eigenvalues are simple.

% \vspace*{-10mm}
\noindent 
(1){V. Mehrmann, V. Noferini, F. Tisseur, and H. Xu},
 {\em On the sign characteristics of Hermitian matrix polynomials}, 
    Linear Alg. App., 511 (2016), pp.~328--364.
\end{ilasabstract}
     \hypertarget{down0294}{}\begin{ilasabstract}
   \talktitle{Exciting eigenvectors: seeing is believing}
    
    \textbf{Steve Mackey}, \info{11:30\textrm{--}12:00 @ SC1005 (June 26, Thursday)} \hfill \hyperlink{up0294}{$\Uparrow$}
    
    (in {\color{mstitle}MS27: Linear algebra education})
        
        \mtskip
    There is a simple, inexpensive, easy-to-build, and easy-to-operate device 
(adapted from (1)) 
that can be used to demonstrate to students the physical reality of eigenvectors.
Steve will begin the talk by showing you that device, 
and briefly discussing some of its properties.
% and tell a bit about how I have used this in various settings, 
% both undergraduate and graduate, since the 1980's. 
Although he has used it primarily in lecture/demonstration mode, 
there is considerable scope for adapting this 
to a more hands-on, direct-engagement-by-students mode.  
In the second half of the talk, 
Raf will tell you about his recent classroom experience
with this device, used in exactly that way.

(1) {H. V. McIntosh} 
{\em Matrix Analysis II: Further Introduction and some Applications to Physical Problems},  
1952.
\end{ilasabstract}
     \hypertarget{down0193}{}\begin{ilasabstract}
   \talktitle{Error formulas for block rational Krylov approximations of matrix functions}
    
    \textbf{Stefano Massei}, \info{13:30\textrm{--}14:00 @ SC4011 (June 24, Tuesday)} \hfill \hyperlink{up0193}{$\Uparrow$}
    
    (in {\color{mstitle}MS23: Advances in Krylov subspace methods and their applications})
        
        \mtskip
    We investigate explicit expressions for the error associated with the block rational Krylov approximation of matrix functions. Two formulas are proposed, both derived from  characterizations of the block FOM residual. The first formula employs a block generalization of the residual polynomial, while the second leverages the block collinearity of the residuals. A posteriori error bounds based on the knowledge 
of spectral information of the argument are derived and tested on a set of examples. Notably, both error formulas and their corresponding upper bounds do not require the evaluation of contour integrals.  
\end{ilasabstract}
     \hypertarget{down0249}{}\begin{ilasabstract}
   \talktitle{nan}
    
    \textbf{Izuho Matsuzaki}, \info{11:30\textrm{--}12:00 @ SC0012 (June 25, Wednesday)} \hfill \hyperlink{up0249}{$\Uparrow$}
    
    (in {\color{mstitle}MS35: Preserver Problems, II})
        
        \mtskip
    nan\end{ilasabstract}
     \hypertarget{down0237}{}\begin{ilasabstract}
   \talktitle{Rational approximations of fractional power operators applied to preconditioning}
    
    \textbf{Mariarosa Mazza}, \info{16:00\textrm{--}16:30 @ SC4011 (June 24, Tuesday)} \hfill \hyperlink{up0237}{$\Uparrow$}
    
    (in {\color{mstitle}MS23: Advances in Krylov subspace methods and their applications})
        
        \mtskip
    In this talk, we consider the Riesz operator \( -(- \Delta)^{\frac{\alpha}{2}} \), \(\alpha \in (1, 2]\), which arises in fractional models such as anomalous diffusion, and develop effective preconditioners for its efficient numerical solution. First, we approximate \( -(-\Delta)^{\frac{\alpha}{2}} \) as the fractional power of a discretized Laplacian using the Matrix Transfer Technique and represent the result in integral form via the Dunford-Taylor integral representation. Various quadrature rules are then explored to approximate the integral, leading to rational approximations of the fractional power operator.  
This approach enables us to construct preconditioners expressed as a sum of \( m \) inverses of shifted Laplacian matrices, where \( m \) depends on the chosen quadrature scheme. For \(\alpha\) close to $2$, it is well known that the Laplacian itself serves as an effective preconditioner with linear computational cost. However, as \(\alpha\) decreases toward 1, its performance deteriorates, requiring more specialized approaches. Using Gauss-Jacobi quadrature, we show that our preconditioner significantly improves performance for \(\alpha\) close to $1$, even with a modest \( m \), while maintaining the same computational complexity as the Laplacian.  
To further enhance efficiency, we investigate the use of exponentially convergent quadrature rules to minimize the number of required inverses while achieving optimal preconditioning performance. Specifically, we examine both sinc and Gauss-Laguerre quadratures and demonstrate that, with appropriate parameter tuning, both approaches outperform the Gauss-Jacobi one, ensuring numerical optimality.\end{ilasabstract}
     \hypertarget{down0220}{}\begin{ilasabstract}
   \talktitle{nan}
    
    \textbf{David Meadon}, \info{17:30\textrm{--}18:00 @ SC1003 (June 24, Tuesday)} \hfill \hyperlink{up0220}{$\Uparrow$}
    
    (in {\color{mstitle}MS19: Explicit and hidden asymptotic structures, GLT Analysis, and applications})
        
        \mtskip
    nan\end{ilasabstract}
     \hypertarget{down0064}{}\begin{ilasabstract}
   \talktitle{nan}
    
    \textbf{Beatrice Meini}, \info{14:00\textrm{--}14:30 @ SC2001 (June 23, Monday)} \hfill \hyperlink{up0064}{$\Uparrow$}
    
    (in {\color{mstitle}MS5: Advances in matrix equations: Theory, computations, and applications})
        
        \mtskip
    nan\end{ilasabstract}
     \hypertarget{down0047}{}\begin{ilasabstract}
   \talktitle{Convergence properties of sequences related to the Ando-Li-Mathias construction and to the weighted Cheap mean}
    
    \textbf{Jie Meng}, \info{14:30\textrm{--}15:00 @ SC0009 (June 23, Monday)} \hfill \hyperlink{up0047}{$\Uparrow$}
    
    (in {\color{mstitle}MS29: Matrix functions and related topics})
        
        \mtskip
    Sequences defining a weighted matrix geometric mean are investigated and their convergence speed is analyzed.
The superlinear convergence of a weighted mean based on the Ando-Li-Mathias (ALM) construction is proved. A weighted Cheap mean is defined and conditions on the weights for linear or superlinear convergence of order at least three are provided.
\end{ilasabstract}
     \hypertarget{down0056}{}\begin{ilasabstract}
   \talktitle{nan}
    
    \textbf{Vatsalkumar Mer}, \info{14:30\textrm{--}15:00 @ SC1001 (June 23, Monday)} \hfill \hyperlink{up0056}{$\Uparrow$}
    
    (in {\color{mstitle}MS10: Matrix means and related topics})
        
        \mtskip
    nan\end{ilasabstract}
     \hypertarget{down0292}{}\begin{ilasabstract}
   \talktitle{Computational labs to enhance linear algebra intuition}
    
    \textbf{Mike Michailidis}, \info{10:30\textrm{--}11:00 @ SC1005 (June 26, Thursday)} \hfill \hyperlink{up0292}{$\Uparrow$}
    
    (in {\color{mstitle}MS27: Linear algebra education})
        
        \mtskip
    Linear algebra is a key topic in mathematics and a core component of science and engineering education. In this talk, we will explore the role of programming in enhancing linear algebra education. Specifically, we will report on a study about a proof-based second course in linear algebra encompassing various topics, including vector spaces, finite-dimensional vector spaces, linear maps, polynomials, inner product spaces, operators on inner product spaces, eigenvalues, and eigenvectors. The existing course was reorganized to incorporate six labs and a final project using MATLAB, to help students explore numerical linear algebra and its applications via programming. 22 mathematics and engineering students were enrolled in this course and 15 agreed to participate in this study, the findings of which we will report and present during this talk. 
\end{ilasabstract}
     \hypertarget{down0019}{}\begin{ilasabstract}
   \talktitle{A new class of distances between pure quantum states}
    
    \textbf{Tomasz Miller}, \info{11:00\textrm{--}11:30 @ SC0014 (June 23, Monday)} \hfill \hyperlink{up0019}{$\Uparrow$}
    
    (in {\color{mstitle}MS8: Tensor and quantum information science})
        
        \mtskip
    For any $n \times n$ distance matrix $(E_{ij})$, we show that the map
\begin{align*}
d(\textup{\textbf{x}},\textup{\textbf{y}}) := \sqrt{\sum_{i<j}E_{ij}^2|x_iy_j - x_jy_i|^2},
\end{align*}
where $\textup{\textbf{x}},\textup{\textbf{y}} \in {\mathbb C}^n$ are unit vectors, gives rise to a bona fide distance on the projective space ${\mathbb P}({\mathbb C}^n)$, a far-reaching generalization of the standard distance on ${\mathbb P}({\mathbb C}^n)$ arising from the wedge product $\|\textup{\textbf{x}} \wedge \textup{\textbf{y}}\| = \sqrt{1 - |\langle \textup{\textbf{x}} | \textup{\textbf{y}} \rangle|^2}$. We also discuss how this result carries over to the $n \rightarrow +\infty$ case, offering a way to `lift' the metric structure from some underlying metric measure space $(X,D,\mu)$ to the projective Hilbert space ${\mathbb P}(L^2(X,\mu))$. The talk builds on and extends the results of [1].

[1] R. Bistro\'{n}, M. Eckstein, S. Friedland, TM, K. \.{Z}yczkowski, \textit{A new class of distances on complex projective spaces}, Linear Algebra Appl., 2024\end{ilasabstract}
     \hypertarget{down0275}{}\begin{ilasabstract}
   \talktitle{The algebra generated by nilpotent elements in a matrix centralizer}
    
    \textbf{Eloise Misa}, \info{11:00\textrm{--}11:30 @ SC0008 (June 26, Thursday)} \hfill \hyperlink{up0275}{$\Uparrow$}
    
    (in {\color{mstitle}MS31: Matrix decompositions and applications})
        
        \mtskip
    For an arbitrary square matrix $S$, denote by $C(S)$ the centralizer of $S$, and by $C(S)_N$ the set of all nilpotent elements in $C(S)$.
In this paper, we use the Weyr canonical form to study the subalgebra $\mathcal{A}(S)$ of $C(S)$ generated by $C(S)_N$. We give a necessary and/or sufficient condition such that $A \in C(S)$ is a sum or product of nilpotent matrices in $C(S)$. We determine conditions on $S$ such that $C(S)_N$ is a subalgebra of $C(S)$, that is, when $\mathcal{A}(S)=C(S)_N$. We also determine conditions on $S$ such that the subalgebra generated by $C(S)_N$ is $C(S)$, that is, when $\mathcal{A}(S)=C(S)$.\end{ilasabstract}
     \hypertarget{down0164}{}\begin{ilasabstract}
   \talktitle{Surjective isometries on Banach spaces with derivatives}
    
    \textbf{Takeshi Miura}, \info{15:00\textrm{--}15:30 @ SC0012 (June 24, Tuesday)} \hfill \hyperlink{up0164}{$\Uparrow$}
    
    (in {\color{mstitle}MS12: Preserver problems, I})
        
        \mtskip
    We shall give the characterization of surjective, possibly nonlinear,
isometries from Banach spaces with derivatives.
This unifies the former results on isometries on the following
Banach spaces:
\begin{enumerate}
\item
$C^1([0,1])$ of all continuously differentiable complex-valued functions
on the closed interval $[0,1]$.

\item
The Banach space of all continuous extensions to $\overline{\mathbb{D}}$,
the closure of the open unit disc $\mathbb{D}$,
of all analytic functions on $\mathbb{D}$,
which can be extended to continuous functions on $\overline{\mathbb{D}}$.

\item
The Banach space of all Gelfand transforms of analytic functions
on $\mathbb{D}$ whose derivatives are bounded on $\mathbb{D}$.
\end{enumerate}
\end{ilasabstract}
     \hypertarget{down0347}{}\begin{ilasabstract}
   \talktitle{Some norm bounds on the complimentary error matrix functions}
    
    \textbf{Shinya Miyajima}, \info{13:30\textrm{--}14:00 @ SC4011 (June 26, Thursday)} \hfill \hyperlink{up0347}{$\Uparrow$}
    
    (in {\color{mstitle}MS1: Embracing new opportunities in numerical linear algebra})
        
        \mtskip
    \begin{bibunit}
        Let $A \in \mathbf{C}^{n \times n}$ be a square matrix such that ${\rm Re}(\lambda) > |{\rm Im}(\lambda)|$, $\forall \lambda \in \sigma(A)$, where $\sigma(A)$ is the spectrum of $A$. 
The error matrix function ${\rm erf}(A)$ and complimentary error matrix function ${\rm erfc}(A)$, which are introduced in \cite{Cortes}, are defined as 
$$
{\rm erf}(A) := \frac{2A}{\sqrt{\pi}}\int_{0}^{1} e^{-(Av)^{2}} dv \quad \mbox{and} \quad {\rm erfc}(A) := \frac{2A}{\sqrt{\pi}}\int_{1}^{\infty} e^{-(Av)^{2}} dv,
$$
respectively. 

One of the most important application of the complimentary error matrix function is the solution to systems of partial differential equation. 
Let $u_0, u(x,t) \in \mathbf{C}^n$. 
It is proven in \cite{Cortes} that the solution to semi-finite coupled diffusion problem 
\begin{eqnarray*}
u_{t} &=& A^2u_{xx}, \ x>0, \ t>0, \\
u(x,0) &=& 0, \ x>0, \quad 
u(0,t) = u_{0}, \ t>0, \quad 
u(x,t) \to 0, \ \textrm{as} \ x\rightarrow\infty, \ t > 0
\end{eqnarray*}
can be represented by using the complimentary error matrix function as follows:
$$
u(x,t) = {\rm erfc}\left(\frac{A^{-1}x}{2\sqrt{t}}\right)u_{0}, \quad x>0, \quad t>0. 
$$

Analogously to the scaler case, the functions ${\rm erf}(A)$ and ${\rm erfc}(A)$ satisfy the property ${\rm erf}(A) + {\rm erfc}(A) = I$, where $I$ is the $n \times n$ identity matrix. 
According to the Taylor expansion of $e^{-(Av)^{2}}$ where $v \in [0,1]$, and integrating term by term, we obtain 
$$
{\rm erf}(A) = \frac{2A}{\sqrt{\pi}}\sum_{k=0}^{\infty}\frac{(-1)^k A^{2k}}{k! (2k+1)}, 
$$
which is the Taylor expansion of ${\rm erf}(A)$ \cite{Cortes}. 
From this expansion and ${\rm erf}(A) + {\rm erfc}(A) = I$, we obtain 
$$
{\rm erfc}(A) = I - \frac{2A}{\sqrt{\pi}}\sum_{k=0}^{\infty}\frac{(-1)^k A^{2k}}{k! (2k+1)}. 
$$
In \cite{Cortes}, an upper bound on $\|{\rm erfc}(A)\|_2$ under the condition ${\rm Re}(\lambda) > |{\rm Im}(\lambda)|$, $\forall \lambda \in \sigma(A)$ has been derived as a corollary of the fact that ${\rm erfc}(A)$ is well-defined. 

The purpose of this talk is to present the following norm bounds: 
\begin{itemize}
\item a new upper bound on $\|{\rm erfc}(A)\|_2$ under a condition which is different from ${\rm Re}(\lambda) > |{\rm Im}(\lambda)|$, $\forall \lambda \in \sigma(A)$, 
\item upper bounds on $\|{\rm erf}(A) - {\rm erf}(B)\|_2$ and $\|{\rm erfc}(A) - {\rm erfc}(B)\|_2$, where $B \in \mathbf{C}^{n \times n}$, under an assumption, and 
\item upper bounds on 
$$
\left\|{\rm erf}(A) - \frac{2A}{\sqrt{\pi}}\sum_{k=0}^{m}\frac{(-1)^k A^{2k}}{k! (2k+1)}\right\|_2 \ \mbox{and} \ 
\left\|{\rm erfc}(A) - \left(I - \frac{2A}{\sqrt{\pi}}\sum_{k=0}^{m}\frac{(-1)^kA^{2k}}{k!(2k+1)}\right)\right\|_2,
$$ 
where $m$ is a nonnegative integer, under an assumption. 
\end{itemize}
We report results of numerical experiments in order to observe how much larger the presented bounds are compared to the corresponding norms. 
This talk is based on the joint work with Prof. Amir Sadeghi in Islamic Azad University. 

\begin{thebibliography}{5}
\bibitem{Cortes}
J. Cort$\acute{{\rm e}}$s, R. Company, L. J$\acute{{\rm o}}$dar, 
The complementary error matrix function and its role solving coupled diffusion mathematical models, 
Math. Comput. Modell, 42(9--10), 1023--1034 (2005). 
\end{thebibliography}
        \end{bibunit}
        \end{ilasabstract}
     \hypertarget{down0018}{}\begin{ilasabstract}
   \talktitle{nan}
    
    \textbf{Lajos Molnár}, \info{12:00\textrm{--}12:30 @ SC0012 (June 23, Monday)} \hfill \hyperlink{up0018}{$\Uparrow$}
    
    (in {\color{mstitle}MS12: Preserver problems, I})
        
        \mtskip
    nan\end{ilasabstract}
     \hypertarget{down0180}{}\begin{ilasabstract}
   \talktitle{nan}
    
    \textbf{Hermie Monterde}, \info{15:00\textrm{--}15:30 @ SC1005 (June 24, Tuesday)} \hfill \hyperlink{up0180}{$\Uparrow$}
    
    (in {\color{mstitle}MS17: Graphs and matrices in honor of Leslie Hogben's retirement})
        
        \mtskip
    nan\end{ilasabstract}
     \hypertarget{down0351}{}\begin{ilasabstract}
   \talktitle{nan}
    
    \textbf{Hermie Monterde}, \info{16:00\textrm{--}16:30 @ SC0008 (June 26, Thursday)} \hfill \hyperlink{up0351}{$\Uparrow$}
    
    (in {\color{mstitle}MS24: Nonnegative and related families of matrices})
        
        \mtskip
    nan\end{ilasabstract}
     \hypertarget{down0143}{}\begin{ilasabstract}
   \talktitle{Spectral upper bounds for the Grundy number of a graph}
    
    \textbf{Emanuel Juliano Morais Silva}, \info{11:30\textrm{--}12:00 @ SC2001 (June 24, Tuesday)} \hfill \hyperlink{up0143}{$\Uparrow$}
    
    (in {\color{mstitle}MS2: Combinatorial matrix theory})
        
        \mtskip
    The Grundy number of a graph is the minimum number of colors needed to properly color the graph using the first-fit greedy algorithm regardless of the initial vertex ordering. Computing the Grundy number of a graph is an NP-Hard problem. There is a characterization in terms of induced subgraphs: a graph has a Grundy number at least k if and only if it contains a $k$-atom. 
In this talk, we focus on a natural quotient matrix of the adjacency matrix of $k$-atoms and use its combinatorial properties to derive bounds on the Grundy number in terms of the largest eigenvalue and the size of the graph.
This talk is based on a joint work with Gabriel Coutinho and Thiago Assis.
\end{ilasabstract}
     \hypertarget{down0129}{}\begin{ilasabstract}
   \talktitle{Structure-preserving model order reduction of linear time-varying port-Hamiltonian systems}
    
    \textbf{Riccardo Morandin}, \info{10:30\textrm{--}11:00 @ SC0014 (June 24, Tuesday)} \hfill \hyperlink{up0129}{$\Uparrow$}
    
    (in {\color{mstitle}MS6: Model reduction})
        
        \mtskip
    Many physical processes can be naturally modeled using port-Hamiltonian (pH) systems, which are inherently passive and stable, and allow for structure-preserving interconnection, making them particularly suitable for the modeling of complex networks. Furthermore, many dedicated numerical methods have been developed to exploit and preserve the structure of pH systems, e.g. for space- and time-discretization, and model order reduction (MOR).
In our work, we focus on the structure-preserving MOR of linear time-varying (LTV) pH systems. LTV systems appear quite naturally in many applications, e.g. in the linearization of nonlinear systems around non-stationary reference solutions, or when some of the system parameters are time-dependent.
In this talk we introduce a general approach based on (Petrov)-Galerkin projection for the structure-preserving MOR of LTV-pH systems. This includes (but is not limited to) the extension of the effort constraint method to LTV-pH systems. Furthermore, we combine balancing and projection to obtain a reduced model that is guaranteed to be pH. We exhibit numerical experiments to validate our algorithms.
\end{ilasabstract}
     \hypertarget{down0127}{}\begin{ilasabstract}
   \talktitle{On the Scottish Book Problem 155 by Mazur and Sternbach}
    
    \textbf{Michiya Mori}, \info{11:00\textrm{--}11:30 @ SC0012 (June 24, Tuesday)} \hfill \hyperlink{up0127}{$\Uparrow$}
    
    (in {\color{mstitle}MS12: Preserver problems, I})
        
        \mtskip
    The Scottish Book was a notebook used by mathematicians of the Lw\'ow School of Mathematics in Poland to collect unsolved problems in mathematics.
Problem 155 of the Scottish Book asks whether every bijection $U\colon X\to Y$ between two Banach spaces $X, Y$ with the property that, each point of $X$ has a neighborhood on which $U$ is isometric, is globally isometric on $X$. 
In this talk, I will explain that this is true under the additional assumption that $X$ is separable and the weaker assumption of surjectivity instead of bijectivity.
\end{ilasabstract}
     \hypertarget{down0253}{}\begin{ilasabstract}
   \talktitle{nan}
    
    \textbf{Milán Mosonyi}, \info{10:30\textrm{--}11:00 @ SC1001 (June 25, Wednesday)} \hfill \hyperlink{up0253}{$\Uparrow$}
    
    (in {\color{mstitle}MS10: Matrix means and related topics})
        
        \mtskip
    nan\end{ilasabstract}
     \hypertarget{down0286}{}\begin{ilasabstract}
   \talktitle{nan}
    
    \textbf{Akihiro Munemasa}, \info{10:30\textrm{--}11:00 @ SC1001 (June 26, Thursday)} \hfill \hyperlink{up0286}{$\Uparrow$}
    
    (in {\color{mstitle}MS34: Combinatorics, association scheme, and graphs})
        
        \mtskip
    nan\end{ilasabstract}
     \hypertarget{down0262}{}\begin{ilasabstract}
   \talktitle{An approach to computing maximum multiplicity of eigenvalues in graphs}
    
    \textbf{Shahla Nasserasr}, \info{10:30\textrm{--}11:00 @ SC2001 (June 25, Wednesday)} \hfill \hyperlink{up0262}{$\Uparrow$}
    
    (in {\color{mstitle}MS2: Combinatorial matrix theory})
        
        \mtskip
    For a simple graph $G$, the maximum multiplicity of an eigenvalue among all symmetric matrices with the graph of $G$ is denoted by $M(G)$. It is known that for trees, $M(G)$, can be computed by removing certain vertices to reduce the tree to paths. We propose a method to generalize this approach to graphs. Using this generalized method, we compute the maximum multiplicity for unicyclic graphs and several other families of graphs.
This is joint work with Charles R. Johnson, Ant\'onio Leal-Duarte and Carlos M. Saiago.
\end{ilasabstract}
     \hypertarget{down0039}{}\begin{ilasabstract}
   \talktitle{Optimal matrix-mimetic tensor algebras via variable projection
}
    
    \textbf{Elizabeth Newman}, \info{12:00\textrm{--}12:30 @ SC3001 (June 23, Monday)} \hfill \hyperlink{up0039}{$\Uparrow$}
    
    (in {\color{mstitle}MS18: New methods in numerical multilinear algebra})
        
        \mtskip
    Many data are naturally represented as multiway arrays or tensors, and as a result, multilinear data analysis tools have revolutionized feature extraction and data compression. Despite the success of tensor-based approaches, fundamental linear algebra properties often break down in higher dimensions. Recent advances in matrix-mimetic tensor algebra in have made it possible to preserve linear algebraic properties and, as a result, to obtain optimal representations of multiway data. Matrix-mimeticity arises from interpreting tensors as t-linear operators, which in turn are parameterized by invertible linear transformations. The choice of transformation is critical to representation quality, and thus far, has been made heuristically. In this talk, we will learn data-dependent, orthogonal transformations by leveraging the optimality of matrix-mimetic representations. In particular, we will exploit the coupling between transformations and optimal tensor representations using variable projection. We will highlight the efficacy of our proposed approach on image compression and reduced order modeling tasks.
\end{ilasabstract}
     \hypertarget{down0148}{}\begin{ilasabstract}
   \talktitle{A tensor alternating Anderson--Richardson method for solving multilinear systems with $ \mathcal{M} $-tensors}
    
    \textbf{Jing Niu}, \info{11:00\textrm{--}11:30 @ SC3001 (June 24, Tuesday)} \hfill \hyperlink{up0148}{$\Uparrow$}
    
    (in {\color{mstitle}MS18: New methods in numerical multilinear algebra})
        
        \mtskip
    It is well-known that a multilinear system with a nonsingular $ \mathcal{M} $-tensor and a positive right-hand side has a unique positive solution.
Tensor splitting methods are efficient because they do not require computing the Jacobian matrix.
Anderson acceleration is also a Jacobian-free technique. 
The Alternating Anderson--Richardson (AAR) method is also a Jacobian-free method for solving linear systems.
Inspired by the AAR method,
we propose a tensor AAR method for solving multilinear systems.
Specifically, we first present a tensor Richardson method, then apply Anderson acceleration and derive a tensor Anderson--Richardson method, finally, we periodically employ the tensor Anderson--Richardson method within the tensor Richardson method and propose a tensor AAR method.
Numerical experiments show that the proposed method outperforms some tensor splitting methods.
\end{ilasabstract}
     \hypertarget{down0213}{}\begin{ilasabstract}
   \talktitle{Invertible bases and root vectors for analytic matrix-valued functions}
    
    \textbf{Vanni Noferini}, \info{16:00\textrm{--}16:30 @ SC1001 (June 24, Tuesday)} \hfill \hyperlink{up0213}{$\Uparrow$}
    
    (in {\color{mstitle}MS14: Pencils, polynomial, and rational matrices})
        
        \mtskip
    I will revisit the notion of a minimal basis from the viewpoint of the theory of modules over a commutative ring. I will define the concept of an invertible basis and link it to the Main Theorem in the famous paper [G. D. Forney Jr., SIAM J. Control 13, 493-520, 1975]. When the underlying ring $R$ is an elementary divisor domain, the submodules that have an invertible basis are precisely the free pure submodules of $R^n$. As an application, I will consider the ring $\mathcal{A}$ of functions that are analytic on $\Omega \subseteq \mathbb{C}$, where $\Omega$ is either a connected compact set or a connected open set. I will show that, for all matrices $M \in \mathcal{A}^{m \times n}$, $\ker M \cap \mathcal{A}^n$ is a free $\mathcal{A}$-module that admits an invertible basis, or equivalently a basis that is full rank upon evaluation at every $\lambda \in \Omega$. This provides a tool to define maximal sets of root vectors at $\lambda$, and in particular to meaningfully define eigenvectors also for analytic matrices that do not have full rank.
\end{ilasabstract}
     \hypertarget{down0327}{}\begin{ilasabstract}
   \talktitle{Nearest $\Omega$-stable pencil with Riemannian optimization}
    
    \textbf{Lauri Nyman}, \info{13:30\textrm{--}14:00 @ SC1003 (June 26, Thursday)} \hfill \hyperlink{up0327}{$\Uparrow$}
    
    (in {\color{mstitle}MS14: Pencils, polynomial, and rational matrices})
        
        \mtskip
    In this talk, we consider the problem of finding the nearest $\Omega$-stable pencil to a given square pencil $A+xB \in \mathbb{C}^{n \times n}$, where a pencil is called $\Omega$-stable if it is regular and all of its eigenvalues belong to the closed set $\Omega$. We propose a new method, based on the Schur form of a matrix pair and Riemannian optimization over the manifold $U(n) \times U(n)$, that is, the Cartesian product of the unitary group with itself. While the developed theory holds for any closed set $\Omega$, we focus on two cases that are the most common in applications: Hurwitz stability and Schur stability. 
\end{ilasabstract}
     \hypertarget{down0277}{}\begin{ilasabstract}
   \talktitle{nan}
    
    \textbf{Ryan O'Loughlin}, \info{10:30\textrm{--}11:00 @ SC0009 (June 26, Thursday)} \hfill \hyperlink{up0277}{$\Uparrow$}
    
    (in {\color{mstitle}MS33: Norms of matrices, numerical range, applications of functional analysis to matrix theory})
        
        \mtskip
    nan\end{ilasabstract}
     \hypertarget{down0014}{}\begin{ilasabstract}
   \talktitle{Generalization of B\"ottcher-Wenzel inequality and its application}
    
    \textbf{Hiromichi Ohno}, \info{11:30\textrm{--}12:00 @ SC0009 (June 23, Monday)} \hfill \hyperlink{up0014}{$\Uparrow$}
    
    (in {\color{mstitle}MS29: Matrix functions and related topics})
        
        \mtskip
    The B\"ottcher-Wenzel inequality states that the Hilbert-Schmidt norm of the commutator of matrices $A$ and $B$ is less than or equal to $\sqrt{2}$ times the product of the Hilbert-Schmidt norms of $A$ and $B$. In this talk, we discuss generalizations of the B\"ottcher-Wenzel inequality in which the Hilbert-Schmidt norm is replaced by a weighted Hilbert-Schmidt norm. An application to the uncertainty relation is also considered.\end{ilasabstract}
     \hypertarget{down0163}{}\begin{ilasabstract}
   \talktitle{Periodic surjective isometries on Banach algebras}
    
    \textbf{Shiho Oi}, \info{14:30\textrm{--}15:00 @ SC0012 (June 24, Tuesday)} \hfill \hyperlink{up0163}{$\Uparrow$}
    
    (in {\color{mstitle}MS12: Preserver problems, I})
        
        \mtskip
    We study periodic surjective isometries on $C^{*}$-algebras.  We establish a relation between the complex spectrum of periodic surjective isometries on Banach algebras  and provide several examples that illustrate the range of possibilities that can occur for the complex spectrum of the isometry and classical spectrum of the Jordan $\ast$-isomorphism.
\end{ilasabstract}
     \hypertarget{down0181}{}\begin{ilasabstract}
   \talktitle{Bounded Littlewood identities for cylindric Schur functions and related combinatorics}
    
    \textbf{Soichi Okada}, \info{13:30\textrm{--}14:00 @ SC2001 (June 24, Tuesday)} \hfill \hyperlink{up0181}{$\Uparrow$}
    
    (in {\color{mstitle}MS25: Enumerative/algebraic combinatorics and matrices})
        
        \mtskip
    The bounded Littlewood identities are determinant formulas for 
the sum of Schur functions indexed by partitions with bounded height. 
These have interesting combinatorial consequences involving 
standard Young tableaux of bounded height. In this talk, we 
give affine analogs of the bounded Littlewood identities, which 
are determinant formulas for sums of cylindric Schur functions. 
As an application, we obtain equinumerous results between cylindric 
standard Young tableaux and partial matchings.

This talk is based on a joint work with JiSun Huh, Jang Soo Kim, 
and Christian Krattenthaler.
\end{ilasabstract}
     \hypertarget{down0343}{}\begin{ilasabstract}
   \talktitle{nan}
    
    \textbf{Eda Oktay}, \info{13:30\textrm{--}14:00 @ SC3001 (June 26, Thursday)} \hfill \hyperlink{up0343}{$\Uparrow$}
    
    (in {\color{mstitle}MS16: Approximations and errors in Krylov-based solvers})
        
        \mtskip
    nan\end{ilasabstract}
     \hypertarget{down0385}{}\begin{ilasabstract}
   \talktitle{Jordan and isometric cone automorphisms in Euclidean Jordan algebras}
    
    \textbf{Michael Orlitzky}, \info{17:00\textrm{--}17:30 @ SC2006 (June 26, Thursday)} \hfill \hyperlink{up0385}{$\Uparrow$}
    
    (in {\color{mstitle}MS28: From matrix theory to Euclidean Jordan algebras, FTvN systems, and beyond})
        
        \mtskip
    Every symmetric cone $K$ arises as the cone of squares in a Euclidean
Jordan algebra $V$. As $V$ is a real inner-product space, we may
denote by $\operatorname{Isom}\left(V\right)$ its group of
isometries. The groups $\operatorname{JAut}\left(V\right)$ of its
Jordan-algebra automorphisms and $\operatorname{Aut}\left(K\right)$
of the linear cone automorphisms are then related. For certain inner
products,
%
\begin{equation*}
  \operatorname{JAut}\left(V\right)
  =
  \operatorname{Aut}\left(K\right)
  \cap
  \operatorname{Isom}\left(V\right).
\end{equation*}
%
We characterize the inner products for which this holds.
\end{ilasabstract}
     \hypertarget{down0057}{}\begin{ilasabstract}
   \talktitle{nan}
    
    \textbf{Hiroyuki Osaka}, \info{15:00\textrm{--}15:30 @ SC1001 (June 23, Monday)} \hfill \hyperlink{up0057}{$\Uparrow$}
    
    (in {\color{mstitle}MS10: Matrix means and related topics})
        
        \mtskip
    nan\end{ilasabstract}
     \hypertarget{down0266}{}\begin{ilasabstract}
   \talktitle{nan}
    
    \textbf{Hiroyuki Osaka}, \info{11:00\textrm{--}11:30 @ SC2006 (June 25, Wednesday)} \hfill \hyperlink{up0266}{$\Uparrow$}
    
    (in {\color{mstitle}MS3: Matrix inequalities with applications})
        
        \mtskip
    nan\end{ilasabstract}
     \hypertarget{down0364}{}\begin{ilasabstract}
   \talktitle{nan}
    
    \textbf{Davide Palitta}, \info{16:30\textrm{--}17:00 @ SC0014 (June 26, Thursday)} \hfill \hyperlink{up0364}{$\Uparrow$}
    
    (in {\color{mstitle}MS5: Advances in matrix equations: Theory, computations, and applications})
        
        \mtskip
    nan\end{ilasabstract}
     \hypertarget{down0190}{}\begin{ilasabstract}
   \talktitle{nan}
    
    \textbf{Junjun Pan}, \info{14:00\textrm{--}14:30 @ SC3001 (June 24, Tuesday)} \hfill \hyperlink{up0190}{$\Uparrow$}
    
    (in {\color{mstitle}MS18: New methods in numerical multilinear algebra})
        
        \mtskip
    nan\end{ilasabstract}
     \hypertarget{down0305}{}\begin{ilasabstract}
   \talktitle{nan}
    
    \textbf{Haesun Park}, \info{11:00\textrm{--}11:30 @ SC4011 (June 26, Thursday)} \hfill \hyperlink{up0305}{$\Uparrow$}
    
    (in {\color{mstitle}MS1: Embracing new opportunities in numerical linear algebra})
        
        \mtskip
    nan\end{ilasabstract}
     \hypertarget{down0312}{}\begin{ilasabstract}
   \talktitle{nan}
    
    \textbf{Mirjeta Pasha}, \info{14:00\textrm{--}14:30 @ SC0009 (June 26, Thursday)} \hfill \hyperlink{up0312}{$\Uparrow$}
    
    (in {\color{mstitle}MS4: Linear algebra methods for inverse problems and data assimilation})
        
        \mtskip
    nan\end{ilasabstract}
     \hypertarget{down0328}{}\begin{ilasabstract}
   \talktitle{On the numerical stability of compact Krylov methods}
    
    \textbf{Javier Perez}, \info{14:00\textrm{--}14:30 @ SC1003 (June 26, Thursday)} \hfill \hyperlink{up0328}{$\Uparrow$}
    
    (in {\color{mstitle}MS14: Pencils, polynomial, and rational matrices})
        
        \mtskip
    Computational methods like TOAR, CORK, and their variants help solve large polynomial, rational, and more generally, nonlinear eigenvalue problems. 
These methods apply (rational) Arnoldi to a matrix or pencil with a special Kronecker structure.
In this talk, I will present some results on the numerical stability of these methods and compare them to  (rational) Arnoldi applied without exploiting the Kronecker structure.\end{ilasabstract}
     \hypertarget{down0280}{}\begin{ilasabstract}
   \talktitle{Linear preservers of rank one projections}
    
    \textbf{Lucijan Plevnik}, \info{10:30\textrm{--}11:00 @ SC0012 (June 26, Thursday)} \hfill \hyperlink{up0280}{$\Uparrow$}
    
    (in {\color{mstitle}MS35: Preserver Problems, II})
        
        \mtskip
    Let $\mathcal H$ be a complex Hilbert space and let ${\mathcal F}_{s}(\mathcal H)$ be the real vector space of all self-adjoint finite rank operators on $H$.
We will present the description of linear maps on ${\mathcal F}_{s}(\mathcal H)$ sending rank one projections to rank one projections.
Such maps are either induced by a linear or conjugate-linear isometry on $\mathcal H$ or constant on the set of rank one projections.

We will also discuss linear maps ${\mathcal F}_{s}(\mathcal H) \to {\mathcal F}_{s}(\mathcal K)$ sending rank one projections to projections of a fixed rank.
In the case $\dim \mathcal H = 2$, we will present the description of such maps, including a new kind of (injective) maps additionally to the previously mentioned one.
In the case $\dim \mathcal H = 3$, we will show by an example that such maps may be neither injective nor constant on the set of rank one projections.
\end{ilasabstract}
     \hypertarget{down0229}{}\begin{ilasabstract}
   \talktitle{nan}
    
    \textbf{Jamie Pommersheim}, \info{16:00\textrm{--}16:30 @ SC2006 (June 24, Tuesday)} \hfill \hyperlink{up0229}{$\Uparrow$}
    
    (in {\color{mstitle}MS30: Bohemian matrices: Theory, applications, and explorations})
        
        \mtskip
    nan\end{ilasabstract}
     \hypertarget{down0017}{}\begin{ilasabstract}
   \talktitle{Linear preservers of parallel pairs}
    
    \textbf{Edward Poon}, \info{11:30\textrm{--}12:00 @ SC0012 (June 23, Monday)} \hfill \hyperlink{up0017}{$\Uparrow$}
    
    (in {\color{mstitle}MS12: Preserver problems, I})
        
        \mtskip
    Two vectors $x$,$y$ in a normed space $(\mathcal{X}, \| \cdot \|)$ are said to be parallel if there exists a scalar $c$ with modulus 1 such that $\|x+c y\| = \|x\| + \|y\|$.
We consider the case of norms on a matrix space (in particular the Ky Fan $k$-norms and the $k$-numerical radius) and characterize the linear bijections on this matrix space which preserve parallel pairs of matrices.
\end{ilasabstract}
     \hypertarget{down0194}{}\begin{ilasabstract}
   \talktitle{Krylov subspace methods an the $\star$-algebra}
    
    \textbf{Stefano Pozza}, \info{14:00\textrm{--}14:30 @ SC4011 (June 24, Tuesday)} \hfill \hyperlink{up0194}{$\Uparrow$}
    
    (in {\color{mstitle}MS23: Advances in Krylov subspace methods and their applications})
        
        \mtskip
    \begin{bibunit}
        The $\star$-algebra \cite{Man24} has been successfully employed to solve a system of linear ODEs with time-dependent coefficients by transforming the ODE into a system of linear $\star$-equations \cite{BonPozVan23,GisPoz}. Many classical linear algebra methods and decomposition can be recovered in the $\star$-algebra. The recovered methods are a time-dependent version of the original ones. For example, the $\star$-Lanczos algorithm \cite{GisPoz} generalizes the (classical) non-Hermitian Lanczos algorithm and extends the well-known connections between matrix moments, and the Lanczos algorithm \cite{GolMeu,PozPraStr} to the time-dependent case.
The relation between the classical Krylov subspace methods and their $\star$-counterparts can be used as a way to compute the solution of ODEs by the $\star$-algebra approach and, vice versa, to analyze classical Krylov subspace methods for time-dependent matrices with possible applications to perturbed matrices.

\begin{thebibliography}{99}
\bibitem{BonPozVan23}
Bonhomme, C., Pozza, P., Van Buggenhout, N.:
A new fast numerical method for the generalized Rosen-Zener model. arXiv:2311.04144 [math.NA] (2023)

\bibitem{GisPoz}
Giscard, P-L., Pozza, S., A Lanczos-like method for non-autonomous linear ordinary differential equations. Boll. Unione Mat. Ital 16, 81--102 (2023)

\bibitem{GolMeu}
Golub, G.H., Meurant, G.: Matrices, Moments and Quadrature with Applications. Princeton University Press, Princeton (2010)

\bibitem{PozPraStr}
Pozza, S., Prani\'c, M.S., Strako\v s, Z.: Gauss quadrature for quasi-definite linear functionals. IMA J. Numer. Anal. 37(3), 1468--1495 (2017)

\bibitem{Man24}
Ryckebusch, M.: A Fr\'echet-Lie group on distributions. arXiv:2307.09037 [math.FA] (2024).

\end{thebibliography}
        \end{bibunit}
        \end{ilasabstract}
     \hypertarget{down0199}{}\begin{ilasabstract}
   \talktitle{nan}
    
    \textbf{Shivaramakrishna Pragada}, \info{17:00\textrm{--}17:30 @ SC0008 (June 24, Tuesday)} \hfill \hyperlink{up0199}{$\Uparrow$}
    
    (in {\color{mstitle}MS24: Nonnegative and related families of matrices})
        
        \mtskip
    nan\end{ilasabstract}
     \hypertarget{down0132}{}\begin{ilasabstract}
   \talktitle{nan}
    
    \textbf{Miklós Pálfia}, \info{10:30\textrm{--}11:00 @ SC1001 (June 24, Tuesday)} \hfill \hyperlink{up0132}{$\Uparrow$}
    
    (in {\color{mstitle}MS10: Matrix means and related topics})
        
        \mtskip
    nan\end{ilasabstract}
     \hypertarget{down0149}{}\begin{ilasabstract}
   \talktitle{nan}
    
    \textbf{Jingmei Qiu}, \info{11:30\textrm{--}12:00 @ SC3001 (June 24, Tuesday)} \hfill \hyperlink{up0149}{$\Uparrow$}
    
    (in {\color{mstitle}MS18: New methods in numerical multilinear algebra})
        
        \mtskip
    nan\end{ilasabstract}
     \hypertarget{down0141}{}\begin{ilasabstract}
   \talktitle{Idempotent alternating sign matrices}
    
    \textbf{Rachel Quinlan}, \info{10:30\textrm{--}11:00 @ SC2001 (June 24, Tuesday)} \hfill \hyperlink{up0141}{$\Uparrow$}
    
    (in {\color{mstitle}MS2: Combinatorial matrix theory})
        
        \mtskip
    An alternating sign matrix (ASM) is a square $(0,1,-1)$-matrix in which the nonzero entries alternate between 1 and $-1$ and sum to 1, within each row and column. Permutation matrices are examples of ASMs, and ASMs generalize permutation matrices in several apparently natural but unexpected ways. Every multiplicative group of nonsingular ASMs is a group of permutation matrices, but the set of all $n\times n$ ASMs also contains multiplicative groups of singular matrices. The identity element $E$ of such a group is an idempotent ASM, it satisfies $E^2=E$. In this talk we will discuss some methods for construction of idempotent ASMs, and identify the minimum rank of an idempotent ASM of specified size. \\
This is joint work with Cian O'Brien.
\end{ilasabstract}
     \hypertarget{down0293}{}\begin{ilasabstract}
   \talktitle{Project work in an undergraduate linear algebra course}
    
    \textbf{Rachel Quinlan}, \info{11:00\textrm{--}11:30 @ SC1005 (June 26, Thursday)} \hfill \hyperlink{up0293}{$\Uparrow$}
    
    (in {\color{mstitle}MS27: Linear algebra education})
        
        \mtskip
    This talk will report on the work of some students in a free-form project
that was one component of assessment in an undergraduate linear algebra course taken by second year students at the University of Galway. Students were encouraged to connect their linear algebra knowledge to their other interests and to submit work in any medium of their choice. Submissions included work connected to education and lifelong learning, poetry, craftwork and the visual arts, along with many topics more prominently associated with linear algebra. The talk will share some highlights, and discuss some of the motivation for the inclusion of this project element and some of the learning outcomes for the instructor.

\end{ilasabstract}
     \hypertarget{down0244}{}\begin{ilasabstract}
   \talktitle{Exploring the numerical range of block Toeplitz operators}
    
    \textbf{Brooke Randell}, \info{10:30\textrm{--}11:00 @ SC0009 (June 25, Wednesday)} \hfill \hyperlink{up0244}{$\Uparrow$}
    
    (in {\color{mstitle}MS33: Norms of matrices, numerical range, applications of functional analysis to matrix theory})
        
        \mtskip
    We will discuss the numerical range of a family of Toeplitz operators with symbol function \(\phi(z)=A_0+zA_1\), where \(A_0\) and \(A_1\) are \(2 \times 2\) matrices with complex-valued entries. A special case of a result proved by Bebiano and Spitkovsky in 2011 states that the closure of the numerical range of the Toeplitz operator \(T_{\phi(z)}\) is the convex hull of \(\{W(\phi(z)): z \in \partial \mathbb{D}\}\). Here, \(W(\phi(z))\) denotes the numerical range of \(\phi(z)\). We combine this result with the envelope algorithm to describe the boundary of the convex hull of \(\{W(\phi(z)): z \in \partial \mathbb{D}\}\). We also place specific conditions on the matrices \(A_0\) and \(A_1\) so that \(\{W(\phi(z)): z \in \partial \mathbb{D}\}\) is a set of potentially degenerate circular disks. The convex hull of \(\{W(\phi(z)): z \in \partial \mathbb{D}\}\) takes on a wide variety of shapes, including the convex hull of lima\c{c}ons.\end{ilasabstract}
     \hypertarget{down0356}{}\begin{ilasabstract}
   \talktitle{nan}
    
    \textbf{Vishwas Rao}, \info{16:30\textrm{--}17:00 @ SC0009 (June 26, Thursday)} \hfill \hyperlink{up0356}{$\Uparrow$}
    
    (in {\color{mstitle}MS4: Linear algebra methods for inverse problems and data assimilation})
        
        \mtskip
    nan\end{ilasabstract}
     \hypertarget{down0182}{}\begin{ilasabstract}
   \talktitle{Principal minors of tree distance matrices}
    
    \textbf{Harry Richman}, \info{14:00\textrm{--}14:30 @ SC2001 (June 24, Tuesday)} \hfill \hyperlink{up0182}{$\Uparrow$}
    
    (in {\color{mstitle}MS25: Enumerative/algebraic combinatorics and matrices})
        
        \mtskip
    Suppose $D$ is the distance matrix of a tree. 
Graham and Pollack showed that the determinant of $D$ satisfies a surprising identity that depends only on the number of vertices in the given tree.
We generalize this result to a combinatorial identity for the determinant of any principal submatrix of $D$.
This new identity involves counts of spanning forests and is proved by use of potential-theoretic concepts on graphs.
\end{ilasabstract}
     \hypertarget{down0284}{}\begin{ilasabstract}
   \talktitle{nan}
    
    \textbf{Albert Rico}, \info{11:00\textrm{--}11:30 @ SC0014 (June 26, Thursday)} \hfill \hyperlink{up0284}{$\Uparrow$}
    
    (in {\color{mstitle}MS8: Tensor and quantum information science})
        
        \mtskip
    nan\end{ilasabstract}
     \hypertarget{down0240}{}\begin{ilasabstract}
   \talktitle{Krylov techniques for estimating spectral gaps of sparse symmetric matrices}
    
    \textbf{Michele Rinelli}, \info{17:30\textrm{--}18:00 @ SC4011 (June 24, Tuesday)} \hfill \hyperlink{up0240}{$\Uparrow$}
    
    (in {\color{mstitle}MS23: Advances in Krylov subspace methods and their applications})
        
        \mtskip
    We propose and analyze an algorithm for identifying spectral gaps of a real symmetric matrix $A$ by simultaneously approximating the traces of spectral projectors associated with multiple different spectral slices. Our method utilizes Hutchinson's stochastic trace estimator together with the Lanczos algorithm to approximate quadratic forms involving spectral projectors.\\
Instead of focusing on determining the gap between two particular consecutive eigenvalues of $A$, we aim to find all gaps that are wider than a specified threshold. By examining the problem from this perspective, and thoroughly analyzing both the Hutchinson and the Lanczos components of the algorithm, we obtain error bounds that allow us to determine the numbers of Hutchinson's sample vectors and Lanczos iterations needed to ensure the detection of all gaps above the target width with high probability.\\
We conclude that the most efficient strategy is to always use a single random sample vector for Hutchinson's estimator and concentrate all computational effort in the Lanczos algorithm. Our numerical experiments demonstrate the efficiency and reliability of this approach.
\end{ilasabstract}
     \hypertarget{down0037}{}\begin{ilasabstract}
   \talktitle{A multilinear Nyström algorithm for low-rank approximation of tensors in Tucker format}
    
    \textbf{Leonardo Robol}, \info{11:00\textrm{--}11:30 @ SC3001 (June 23, Monday)} \hfill \hyperlink{up0037}{$\Uparrow$}
    
    (in {\color{mstitle}MS18: New methods in numerical multilinear algebra})
        
        \mtskip
    The Nystr\"om method offers an effective way to obtain low-rank approximation of SPD matrices and has been recently extended and analyzed to nonsymmetric matrices (leading to the generalized Nystr\"om method). It is a randomized, single-pass, streamable, cost-effective, and accurate alternative to the randomized SVD, and it facilitates the computation of several matrix low-rank factorizations. We take these advancements a step further by introducing a higher-order variant of Nystrom’s methodology tailored to approximating low-rank tensors in the Tucker format: the multilinear Nystr\"om technique. We show that, by introducing appropriate small modifications in the formulation of the higher-order method, strong stability properties can be obtained. This algorithm retains the key attributes of the generalized Nystr\"om method, positioning it as a viable substitute for the randomized higher-order SVD algorithm.\end{ilasabstract}
     \hypertarget{down0373}{}\begin{ilasabstract}
   \talktitle{Row completion of polynomial and rational matrices and partial prescription of their  structure}
    
    \textbf{Alicia Roca}, \info{17:00\textrm{--}17:30 @ SC1003 (June 26, Thursday)} \hfill \hyperlink{up0373}{$\Uparrow$}
    
    (in {\color{mstitle}MS14: Pencils, polynomial, and rational matrices})
        
        \mtskip
    The matrix completion problem consists in characterizing the existence of a matrix with certain properties when a submatrix is prescribed.  It is an important problem in Matrix Theory.
Completion problems of matrices frequently arise in applications, for instance in structural changes of the dynamics of a system or in pole placement problems in control  theory. They also appear in solutions of perturbation problems. 
This work is devoted to the row completion problem for polynomial and rational matrices. 
We characterize the existence of a polynomial matrix when its complete structural data  (the invariant factors, the invariants orders at infinity, and the column and row minimal indices) and some of its rows are prescribed, allowing the completed matrix to increase the degree.
The same problem is solved for rational matrices.
We have also  solved the corresponding problem of partial prescription of the complete structural data covering  all of the possibilities. We will show here some cases.
Obviously, the results obtained hold for the corresponding column  completion problems.


\end{ilasabstract}
     \hypertarget{down0313}{}\begin{ilasabstract}
   \talktitle{nan}
    
    \textbf{Fred Roosta}, \info{14:30\textrm{--}15:00 @ SC0009 (June 26, Thursday)} \hfill \hyperlink{up0313}{$\Uparrow$}
    
    (in {\color{mstitle}MS4: Linear algebra methods for inverse problems and data assimilation})
        
        \mtskip
    nan\end{ilasabstract}
     \hypertarget{down0162}{}\begin{ilasabstract}
   \talktitle{Preservation of orthogonality by unitary and isometric dilations}
    
    \textbf{Saikat Roy}, \info{14:00\textrm{--}14:30 @ SC0012 (June 24, Tuesday)} \hfill \hyperlink{up0162}{$\Uparrow$}
    
    (in {\color{mstitle}MS12: Preserver problems, I})
        
        \mtskip
    In this talk, we will focus on the orthogonality (in the sense of Birkhoff-James) relation of two contractions and that of their unitary (isometric) dilations. In particular, the theme of the talk is ``whether the orthogonality relation between two contractions is preserved by their unitary and isometric dilations." In the process, we consider the Sch\"{a}ffer unitary dilations, Ando dilation for a commuting pair of contractions and regular unitary dilations. In this direction, we construct the unitary dilation of a contraction and show that such kinds of unitary dilation preserve the orthogonality relation of two basic contractions.\end{ilasabstract}
     \hypertarget{down0069}{}\begin{ilasabstract}
   \talktitle{A method for searching for a globally optimal k-partition of higher-dimensional datasets}
    
    \textbf{Kristian Sabo}, \info{15:00\textrm{--}15:30 @ SC2006 (June 23, Monday)} \hfill \hyperlink{up0069}{$\Uparrow$}
    
    (in {\color{mstitle}MS22: Linear algebra applications in computational geometry})
        
        \mtskip
    Finding a globally optimal \( k \)-partition of a set is a highly complex optimization problem. In general, except for the special case of one-dimensional data (i.e., data with a single feature), no exact solution method exists. In the one-dimensional case, efficient methods leverage the fact that the problem can be reformulated as a global optimization task for a symmetric Lipschitz-continuous function, which can be solved using the global optimization algorithm DIRECT.   We propose a method for finding a globally optimal \( k \)-partition in the general case (\( n \)-dimensional data, \( n > 1 \)), extending an approach originally designed for solving Lipschitz global optimization for symmetric functions. Our method integrates a global optimization algorithm with linear constraints and the \( k \)-means algorithm. The global optimization component is used solely to generate a high-quality initial approximation for \( k \)-means.   We evaluated our approach on multiple artificial datasets and several benchmark examples from the UCI Machine Learning Repository. The results demonstrate that the proposed method is more efficient compared to some other methods from the literature. \end{ilasabstract}
     \hypertarget{down0318}{}\begin{ilasabstract}
   \talktitle{On linear preservers of semimonotone matrices}
    
    \textbf{Manideepa Saha}, \info{15:00\textrm{--}15:30 @ SC0012 (June 26, Thursday)} \hfill \hyperlink{up0318}{$\Uparrow$}
    
    (in {\color{mstitle}MS35: Preserver Problems, II})
        
        \mtskip
    A real square matrix $A$ is said to be {\it semimonotone} if for any entrywise nonnegative vector $x\neq 0$, there exists an index $i$ such that  $x_i$, the $i$-th entry of $x$, satisfies $x_i>0$ and $(Ax)_i\ge 0$, and $\mathbf{E}_{0}$ denotes the class of such matrices.  Semimonotone matrices are noteworthy because of their appearance in studying the linear complementarity problem (LCP), which plays an significant role in areas of  bimatrix games, linear programming, quadratic programming etc.   A linear  map $\mathcal{L}:\mathbb{R}^{n\times n}\to\mathbb{R}^{n\times n}$ is said to be an {\it$X$-(linear) preserver} if it preserves certain property $X$, that is, if $X\subset\mathbb{R}^{n\times n}$, then $\mathcal{L}(X)\subseteq X$. In case, $\mathcal{L}(X)= X$, then $\mathcal{L}$ is known as an {\it onto/strong $X$-(linear) preserver}. In this paper, we study strong/onto linear preservers of semimonotone  and almost semimonotone (all proper principal submatrices are semimonotone) matrices. In particular, we prove that an onto $\mathbf{E}_0$-preserver can be written as direct sum of monomial and generalized monominal matrices. We further characterize all strong/onto $\mathbf{E}_0$-preservers in terms of transposition, permutation similarity and positive diagonal equivalence transformations. At last, we provide three types of onto linear preservers that preservers almost semimonotone matrices.
\end{ilasabstract}
     \hypertarget{down0358}{}\begin{ilasabstract}
   \talktitle{nan}
    
    \textbf{Arvind K. Saibaba}, \info{17:30\textrm{--}18:00 @ SC0009 (June 26, Thursday)} \hfill \hyperlink{up0358}{$\Uparrow$}
    
    (in {\color{mstitle}MS4: Linear algebra methods for inverse problems and data assimilation})
        
        \mtskip
    nan\end{ilasabstract}
     \hypertarget{down0276}{}\begin{ilasabstract}
   \talktitle{The $\phi_S$ polar decomposition when $S$ is skew-symmetric}
    
    \textbf{Jenny Salinasan}, \info{11:30\textrm{--}12:00 @ SC0008 (June 26, Thursday)} \hfill \hyperlink{up0276}{$\Uparrow$}
    
    (in {\color{mstitle}MS31: Matrix decompositions and applications})
        
        \mtskip
    Let $F$ be a field with characteristic not equal to $2$, and $S \in M_{2n}(F)$ be skew-symmetric $(S^{\top}=-S)$ and nonsingular.
Let $\phi_S$ be the function defined by $\phi_S(A)=S^{-1}A^{\top}S$ for all $A \in M_{2n}(F)$.
Suppose $A \in M_{2n}(F)$.
We say that $A$ is $\phi_S$-orthogonal if $A$ is nonsingular and $\phi_S(A)=A^{-1}$; and $A$ is $\phi_S$-symmetric if $\phi_S(A)=A$.
We say that $A$ has a $\phi_S$ polar decomposition if $A=XY$ for some $\phi_S$-orthogonal $X \in M_{2n}(F)$ and $\phi_S$-symmetric $Y \in M_{2n}(F)$.
We give necessary and sufficient conditions for an $X \in M_{2n}(F)$ to have a $\phi_S$ polar decomposition.\end{ilasabstract}
     \hypertarget{down0125}{}\begin{ilasabstract}
   \talktitle{Convex matrix functions}
    
    \textbf{Takashi Sano}, \info{11:30\textrm{--}12:00 @ SC0009 (June 24, Tuesday)} \hfill \hyperlink{up0125}{$\Uparrow$}
    
    (in {\color{mstitle}MS29: Matrix functions and related topics})
        
        \mtskip
    I would like to talk about some topics related to convex matrix functions: 
Kraus matrices; conditionally negative semidefiniteness; inertia problems, strongly convex matrix functions, etc. 
\end{ilasabstract}
     \hypertarget{down0307}{}\begin{ilasabstract}
   \talktitle{On commutators of unipotent matrices of index $2$}
    
    \textbf{Juan Paolo Santos}, \info{13:30\textrm{--}14:00 @ SC0008 (June 26, Thursday)} \hfill \hyperlink{up0307}{$\Uparrow$}
    
    (in {\color{mstitle}MS31: Matrix decompositions and applications})
        
        \mtskip
    A commutator of unipotent matrices of index $2$ is a matrix of the form $XYX^{-1}Y^{-1}$, where $X$ and $Y$ are unipotent matrices of index $2$, that is, $X \ne I_n$, $Y \ne I_n$, and $(X-I_n)^2=(Y-I_n)^2=0_n$. If $n>2$ and $\mathbb{F}$ is a field with $|\mathbb{F}| \geq 4$, then it is shown that every $n \times n$ matrix over $\mathbb{F}$ with determinant $1$ is a product of at most four commutators of unipotent matrices of index $2$. Consequently, every $n \times n$ matrix over $\mathbb{F}$ with determinant $1$ is a product of at most eight unipotent matrices of index $2$. Conditions on $\mathbb{F}$ are given that improve the upper bound on the commutator factors from four to three or two. The situation for $n=2$ is also considered. This study reveals a connection between factorability into commutators of unipotent matrices and properties of $\mathbb{F}$ such as its characteristic or its set of perfect squares.\end{ilasabstract}
     \hypertarget{down0302}{}\begin{ilasabstract}
   \talktitle{nan}
    
    \textbf{Michael Saunders}, \info{11:00\textrm{--}11:30 @ SC3001 (June 26, Thursday)} \hfill \hyperlink{up0302}{$\Uparrow$}
    
    (in {\color{mstitle}MS16: Approximations and errors in Krylov-based solvers})
        
        \mtskip
    nan\end{ilasabstract}
     \hypertarget{down0030}{}\begin{ilasabstract}
   \talktitle{Perfect codes and spectrum of graphs - a brief survey
}
    
    \textbf{Lavanya Selvaganesh}, \info{12:00\textrm{--}12:30 @ SC1005 (June 23, Monday)} \hfill \hyperlink{up0030}{$\Uparrow$}
    
    (in {\color{mstitle}MS15: Graphs and their eigenvalues: Celebrating the work of Fan Chung Graham})
        
        \mtskip
    Perfect codes have been key subject of study since the emergence of coding theory in the late 1940s, and they continue to garner considerable interest even after more than sixty years. Hamming and Golay codes are well-known perfect codes, and their significance to information theory is widely recognized. A \textit{code} in a graph $X=(V,E)$ is a non-empty subset of $V$. Given an integer $t\geq 1$, the ball with radius $t$ and centre $u \in V$ is defined as $B_t(u, X) := \{v \in V: d(u, v) \leq t\}$, where $d(u,v)$ is the distance in $X$ between $u$ and $v$. A code $C\subseteq V$ is called a \textit{perfect t-error-correcting code} or a \textit{perfect t-code} in $X$ if the balls $B_t(u, X)$ with radius $t$ and centres $u\in C$ form a partition of $V$. In graph theory, $B_t(u, X)$ is called the $t$-neighbourhood of $u$ in $X$, each vertex in $B_t(u, X)$ is said to be \textit{$t$-dominated} by$u$, a perfect $t$-code in $X$ is called a \textit{perfect t-dominating set} of $X$, and a perfect 1-code in $X$ is called an \textit{efficient dominating set} or \textit{independent perfect dominating set}. In this talk, our goal is to highlight the interplay between the perfect 1-codes and the eigenvalues of graphs. In particular, we focus on the spectrum of special classes of graphs such as Cayley graphs and circulant graphs, and characterize a perfect-1-code in such graphs.  
%Note that this vector has $n/2r$ entries $-(2r-1)$ as we need $(2r) |C| = n$ for $C$ being a perfect code. All other entries of this eigenvector are $1$, so the entry-sum is $0$. 

\end{ilasabstract}
     \hypertarget{down0226}{}\begin{ilasabstract}
   \talktitle{Exploring graph characterization via specialized matrices
}
    
    \textbf{Lavanya Selvaganesh}, \info{16:30\textrm{--}17:00 @ SC2001 (June 24, Tuesday)} \hfill \hyperlink{up0226}{$\Uparrow$}
    
    (in {\color{mstitle}MS25: Enumerative/algebraic combinatorics and matrices})
        
        \mtskip
    Combinatorial matrix theory has found applications in various fields, particularly in spectral graph theory, which has many connections to areas such as chemistry and network analysis. Some of the well-studied objects in this context include the adjacency matrix, Laplacian matrix, signless Laplacian, and distance matrix. More recently, since 2013 and 2021, the eccentricity and neighborhood matrices, respectively, have garnered significant attention due  to several influential articles. A key aspect of spectral graph theory is identifying graphs that can be characterized by their matrix spectrum. We will present recent developments in the study of eccentricity and neighborhood matrices for special classes of graphs, including products of graphs, regular graphs, and multipartite graphs, among others. 

The eccentricity matrix of a connected graph $G$ is derived from its distance matrix by retaining only the largest nonzero entries in each row and column, setting all other entries to zero. This class of matrix was introduced as an application of graph theory to chemical structures.  It is well known that, unlike the distance and the adjacency matrix, the eccentricity matrix is not irreducible for all connected graphs, making it an important problem to identify classes of graphs for which it is irreducible. In this talk, we  investigate various properties of eccentricity matrix and their spectral characteristics. Expanding the scope, we delve into the irreducibility and spectrum of eccentricity matrix for some well-known families of \textit{distance-regular graphs}. 

The neighborhood matrix $\mathcal{NM}(G)$ of a graph $G$ is a graph matrix obtained by the product of adjacency and the Laplacian matrix and is also defined using the neighborhood sets of vertices of $G$. We show that the neighborhood matrix of a connected graph is irreducible.  We determine the $\mathcal{NM}$-eigenvalues of a complete multipartite graph and find that they are always real and non-positive. Moreover, the $\mathcal{NM}$-energy of a complete multipartite graph is twice the number of edges. Further, we show that no two complete multipartite graphs are $\mathcal{NM}$-cospectral.
\end{ilasabstract}
     \hypertarget{down0016}{}\begin{ilasabstract}
   \talktitle{Local order isomorphisms on operator and matrix domains}
    
    \textbf{Peter Semrl}, \info{11:00\textrm{--}11:30 @ SC0012 (June 23, Monday)} \hfill \hyperlink{up0016}{$\Uparrow$}
    
    (in {\color{mstitle}MS12: Preserver problems, I})
        
        \mtskip
    Let $H_n$ denote the set of all $n\times n$ complex hermitian matrices and $S_n$ the set of all $n \times n$ real symmetric matrices. A subset $U \subset H_n$ ($U \subset S_n$) is called a matrix domain if it is open and connected. The general form of maps $\phi : U \to H_n$ ($\phi : U \to S_n$) preserving the usual Loewner order in both directions will be discussed. We will also treat the infinite-dimensional case.
\end{ilasabstract}
     \hypertarget{down0157}{}\begin{ilasabstract}
   \talktitle{nan}
    
    \textbf{Yuki Seo}, \info{13:30\textrm{--}14:00 @ SC0009 (June 24, Tuesday)} \hfill \hyperlink{up0157}{$\Uparrow$}
    
    (in {\color{mstitle}MS29: Matrix functions and related topics})
        
        \mtskip
    nan\end{ilasabstract}
     \hypertarget{down0218}{}\begin{ilasabstract}
   \talktitle{nan}
    
    \textbf{Stefano Serra-Capizzano}, \info{16:30\textrm{--}17:00 @ SC1003 (June 24, Tuesday)} \hfill \hyperlink{up0218}{$\Uparrow$}
    
    (in {\color{mstitle}MS19: Explicit and hidden asymptotic structures, GLT Analysis, and applications})
        
        \mtskip
    nan\end{ilasabstract}
     \hypertarget{down0196}{}\begin{ilasabstract}
   \talktitle{Lanczos with compression for symmetric eigenvalue problems}
    
    \textbf{Nian Shao}, \info{15:00\textrm{--}15:30 @ SC4011 (June 24, Tuesday)} \hfill \hyperlink{up0196}{$\Uparrow$}
    
    (in {\color{mstitle}MS23: Advances in Krylov subspace methods and their applications})
        
        \mtskip
    Lanczos method with implicit restarting is one of the most successful methods for finding a few eigenpairs of a large-scale symmetric matrix.
Despite its widespread use, the core approach of employing polynomial filtering for restarting has remained unchanged over the past two decades.
In this talk, we introduce a novel compression strategy, called Lanczos with compression, as an alternative to restarting.
Unlike restarting, Lanczos with compression sacrifices the Krylov structure but preserves the subsequent Lanczos sequence.
Theoretical analysis demonstrates that the compression introduces only a small error compared to the standard Lanczos method.
Numerical results show that, in terms of matrix-vector products, our approach is never worse and often outperforms the restarting strategy, sometimes by a significant margin.
\end{ilasabstract}
     \hypertarget{down0219}{}\begin{ilasabstract}
   \talktitle{nan}
    
    \textbf{Boris Shapiro}, \info{17:00\textrm{--}17:30 @ SC1003 (June 24, Tuesday)} \hfill \hyperlink{up0219}{$\Uparrow$}
    
    (in {\color{mstitle}MS19: Explicit and hidden asymptotic structures, GLT Analysis, and applications})
        
        \mtskip
    nan\end{ilasabstract}
     \hypertarget{down0371}{}\begin{ilasabstract}
   \talktitle{nan}
    
    \textbf{Punit Sharma}, \info{16:00\textrm{--}16:30 @ SC1003 (June 26, Thursday)} \hfill \hyperlink{up0371}{$\Uparrow$}
    
    (in {\color{mstitle}MS14: Pencils, polynomial, and rational matrices})
        
        \mtskip
    nan\end{ilasabstract}
     \hypertarget{down0234}{}\begin{ilasabstract}
   \talktitle{nan}
    
    \textbf{Shih-Feng Shieh}, \info{16:30\textrm{--}17:00 @ SC3001 (June 24, Tuesday)} \hfill \hyperlink{up0234}{$\Uparrow$}
    
    (in {\color{mstitle}MS11: Structured matrix computations and its applications})
        
        \mtskip
    nan\end{ilasabstract}
     \hypertarget{down0394}{}\begin{ilasabstract}
   \talktitle{Lanczos with compression for symmetric matrix functions}
    
    \textbf{Igor Simunec}, \info{17:30\textrm{--}18:00 @ SC4011 (June 26, Thursday)} \hfill \hyperlink{up0394}{$\Uparrow$}
    
    (in {\color{mstitle}MS23: Advances in Krylov subspace methods and their applications})
        
        \mtskip
    In this talk we present a low-memory method for the approximation of the action of a symmetric matrix function $f(A)$ on a vector $b$, where the matrix $A$ is large and sparse. 
The algorithm that we propose combines the Lanczos method for $f(A) b$ with a basis compression procedure involving rational Krylov subspaces, which is employed whenever the basis grows beyond a certain size in order to reduce memory usage. 
This method has essentially the same convergence behaviour as Lanczos, with the addition of an error term that depends on rational approximation of the function $f$ and is typically negligible. 
The cost of the basis compression procedure is also negligible with respect to the cost of the Lanczos algorithm. In particular, the rational Krylov subspaces used for the compression of the Lanczos basis are built using small projected matrices, so their construction is cheap and does not require the expensive solution of linear systems with the matrix $A$. 
Numerical experiments demonstrate that our algorithm exhibits competitive performance when compared against other low-memory methods for $f(A) b$.

[1] A. A. Casulli and I. Simunec, A low-memory Lanczos method with rational Krylov compression for matrix functions, arXiv:2403.04390, 2024. To appear in SIAM J. Sci. Comput.
\end{ilasabstract}
     \hypertarget{down0282}{}\begin{ilasabstract}
   \talktitle{Linear maps preserving product of involutions}
    
    \textbf{Sushil Singla}, \info{11:30\textrm{--}12:00 @ SC0012 (June 26, Thursday)} \hfill \hyperlink{up0282}{$\Uparrow$}
    
    (in {\color{mstitle}MS35: Preserver Problems, II})
        
        \mtskip
    Let $M_n(\mathbb F)$ the algebra of $n \times n$ matrices over a field  $\mathbb F$.  A matrix $A\in M_n(\mathbb F)$ is said to be involution if $A^2=I$ (the identity matrix in $M_n(\mathbb F)$). Two interesting known facts about the product of involutions are as follows.\\ \\
 (a) A matrix in $  M_n(\mathbb F)$ is similar to its inverse if and only if it can be written as a product of two involutions in $M_n(\mathbb F)$.\\ \\
 (b) An element $X \in M_n(\mathbb F)$ has $\det(X) =\pm 1$ if and only if $X$ can be written as a product of at most four involutions from $M_n(\mathbb F)$. As a consequence, any matrix which is products of involutions can be written as product of at most four involutions.\\ \\
In this talk, we will investigate the bijective linear preservers of matrices in $M_n(\mathbb F)$, which are products of at most two, or three, or four involutions. This is a joint work with Chi-Kwong Li and Tejbir Lohan.

\end{ilasabstract}
     \hypertarget{down0353}{}\begin{ilasabstract}
   \talktitle{nan}
    
    \textbf{K. C. Sivakumar}, \info{17:00\textrm{--}17:30 @ SC0008 (June 26, Thursday)} \hfill \hyperlink{up0353}{$\Uparrow$}
    
    (in {\color{mstitle}MS24: Nonnegative and related families of matrices})
        
        \mtskip
    nan\end{ilasabstract}
     \hypertarget{down0337}{}\begin{ilasabstract}
   \talktitle{nan}
    
    \textbf{Krishnan Sivasubramanian}, \info{14:30\textrm{--}15:00 @ SC2001 (June 26, Thursday)} \hfill \hyperlink{up0337}{$\Uparrow$}
    
    (in {\color{mstitle}MS21: Linear algebra techniques in graph theory})
        
        \mtskip
    nan\end{ilasabstract}
     \hypertarget{down0186}{}\begin{ilasabstract}
   \talktitle{nan}
    
    \textbf{Anthony Man-Cho So}, \info{14:00\textrm{--}14:30 @ SC2006 (June 24, Tuesday)} \hfill \hyperlink{up0186}{$\Uparrow$}
    
    (in {\color{mstitle}MS32: Advances in matrix manifold optimization})
        
        \mtskip
    nan\end{ilasabstract}
     \hypertarget{down0325}{}\begin{ilasabstract}
   \talktitle{Formal orthogonal systems}
    
    \textbf{Wasin So}, \info{14:30\textrm{--}15:00 @ SC1001 (June 26, Thursday)} \hfill \hyperlink{up0325}{$\Uparrow$}
    
    (in {\color{mstitle}MS34: Combinatorics, association scheme, and graphs})
        
        \mtskip
    Dreaming of an effective way to generate orthogonal bases in ${\bf R}^n$, 
we study  formal orthogonal systems which are  collections of nonzero $n \times n$  real matrices 
\[ \{A_0=I_n, A_1, A_2, \ldots,A_{n-1}\}\]
such that $\{A_0x, A_1x,A_2x,\cdots,A_{n-1}x\}$
 is an orthogonal set for all $x$ in ${\bf R}^n$. We prove that 
 formal orthogonal systems of order $n$ exist if and only if $n=1, 2,4,$ or  $8$.
 The proof has an unexpected connection to the famous results of Adolf Hurwitz
  on the product of two sums of $n$ squares.
 
 \vspace{.3in}
 
 Joint work with Ardak Kapbasov and Shaunak Mashalkar.\end{ilasabstract}
     \hypertarget{down0020}{}\begin{ilasabstract}
   \talktitle{Tensor train completion of multiway data observed in a single mode}
    
    \textbf{Shakir Showkat Sofi}, \info{11:30\textrm{--}12:00 @ SC0014 (June 23, Monday)} \hfill \hyperlink{up0020}{$\Uparrow$}
    
    (in {\color{mstitle}MS8: Tensor and quantum information science})
        
        \mtskip
    \begin{bibunit}
        Tensor completion is an extension of matrix completion aimed at recovering a partially observed data tensor by leveraging the observations and the pattern of observation. Completion is more important for tensors than for matrices for several reasons. Higher-order datasets are larger, increasing the likelihood of missing or unreliable entries. Large-scale data often exhibits low-rank properties (intuitively, not every entry is ``equally important,'' unlike in smaller matrices). Many interesting problems can be framed as instances of low-rank tensor completion, including image and video recovery, collaborative filtering, and quantum state tomography \cite{liu2013tc, gross2010quantum}. Low-rank tensor completion is generally solved via convex optimization techniques. Current theories concerning these methods often study probabilistic recovery guarantees under conditions such as random uniform observations and incoherence requirements \cite{candes2009exact, liu2013tc}. However, if an observation pattern has some structure, more efficient algorithms can be developed by leveraging the structure.\par

Algebraic methods exploit the low-rank structure to design algorithms that rely solely on standard numerical linear algebra (NLA) operations. They are fast and are guaranteed to work under reasonable deterministic conditions on the observation pattern. In this line, a specific type of ``fiber-wise'' observation pattern has been discussed, where some of the fibers of a tensor (along a specific mode) are either fully observed or entirely missing, unlike the usual entry-wise observations. This observation is interesting because it appears in many real-life applications and highlights a key difference between matrices and tensors. While missing fibers (rows or columns) in a matrix make completion underdetermined, higher-order tensors can still be completed even if some fibers are entirely missing along one mode. It has been shown that under reasonable conditions, canonical polyadic decomposition (CPD) and multilinear singular value decomposition (MLSVD) of such an incomplete tensor can still be obtained using only standard NLA \cite{mikael2019fibersamp, stijn2023mlsvdfsj}. Note that there is an important difference with the technique of cross or skeleton approximation in the sense that we assume the availability of fibers in one mode only. \par

With the increasing prevalence of big data, the demand for reliable and scalable algorithms has become more pressing. The tensor train (TT) decomposition is stable and can break the curse of dimensionality \cite{oseledets2010tensortrain}. This talk shows how to extend the fiber-wise completion to the TT format. We discuss the deterministic conditions under which the uniqueness of the solution is guaranteed \cite{stijn2023mlsvdfsj, shakir2024ttfw}. Furthermore, we discuss a few interesting applications and briefly highlight the possibility of utilizing this tensor completion framework as a fundamental experimental primitive for efficient quantum state tomography with fewer measurements.

\begin{thebibliography}{7}
\bibitem{liu2013tc} J. Liu, P. Musialski, P. Wonka, and J. Ye. Tensor Completion for Estimating Missing Values in Visual Data. \textit{IEEE Trans. Pattern Anal. Mach. Intell.}, 35:208--220, 2013.

\bibitem{gross2010quantum} D. Gross, Y. K. Liu, S. T. Flammia, S. Becker, and J. Eisert. Quantum State Tomography via Compressed Sensing. \textit{Phys. Rev. Lett.}, 105(15):150401, 2010. APS.

\bibitem{candes2009exact} E. J. Cand{\`e}s and B. Recht. Exact Matrix Completion via Convex Optimization. \textit{Found. Comput. Math.}, 9(6):717--772, 2009. Springer.


\bibitem{mikael2019fibersamp} M. S{\o}rensen and L. De Lathauwer. Fiber Sampling Approach to Canonical Polyadic Decomposition and Application to Tensor Completion. \textit{SIAM J. Matrix Anal. Appl.}, 40:888--917, 2019.

\bibitem{stijn2023mlsvdfsj} M. S{\o}rensen, S. Hendrikx, and L. De Lathauwer. Multilinear Singular Value Decomposition Based Completion with Fibers Observed in a Single Mode. \textit{SIAM J. Matrix Anal. Appl.}, 2025. [Accepted for publication], \url{https://ftp.esat.kuleuven.be/pub/stadius//shendrik/hendrikx2023mlsvdfibersimax.pdf}.

\bibitem{oseledets2010tensortrain} I. Oseledets. Tensor-Train Decomposition. \textit{SIAM J. Sci. Comput.}, 33:2295--2317, 2011.

\bibitem{shakir2024ttfw} S. S. Shakir, S. Hendrikx, and L. De Lathauwer. Tensor Train Completion of Multi-Way Data Observed Along One Mode. In \textit{Proc. 32nd EUSIPCO}, pages 1067--1071, 2024. IEEE.
\end{thebibliography}
        \end{bibunit}
        \end{ilasabstract}
     \hypertarget{down0301}{}\begin{ilasabstract}
   \talktitle{nan}
    
    \textbf{Tomohiro Sogabe}, \info{10:30\textrm{--}11:00 @ SC3001 (June 26, Thursday)} \hfill \hyperlink{up0301}{$\Uparrow$}
    
    (in {\color{mstitle}MS16: Approximations and errors in Krylov-based solvers})
        
        \mtskip
    nan\end{ilasabstract}
     \hypertarget{down0027}{}\begin{ilasabstract}
   \talktitle{Model reduction and matrix compression in dictionary learning applications}
    
    \textbf{Erkki Somersalo}, \info{12:00\textrm{--}12:30 @ SC1003 (June 23, Monday)} \hfill \hyperlink{up0027}{$\Uparrow$}
    
    (in {\color{mstitle}MS26: Utilizing structure to achieve low-complexity algorithms for data science, engineering, and physics})
        
        \mtskip
    Dictionary learning and matching is an attractive way to solve numerically inverse problems in which the forward model is too complex to be used in the inversion process.
Dictionary matching problems lead often to very large underdetermined problems, and dictionary compression is therefore desired. In this talk, we propose a dictionary
compression method that leverages ideas from Bayesian inverse problems with sparsity promoting priors, and takes advantage of the structure of the underlining matrix to design computationally efficient algorithms.
\end{ilasabstract}
     \hypertarget{down0175}{}\begin{ilasabstract}
   \talktitle{GLT-based preconditioning for nonsymmetric Toeplitz systems}
    
    \textbf{Rosita Luisa Sormani}, \info{14:30\textrm{--}15:00 @ SC1003 (June 24, Tuesday)} \hfill \hyperlink{up0175}{$\Uparrow$}
    
    (in {\color{mstitle}MS19: Explicit and hidden asymptotic structures, GLT Analysis, and applications})
        
        \mtskip
    \begin{bibunit}
        Preconditioning for Toeplitz systems has been a prominent area of research for several decades, with many efficient preconditioners available in the real symmetric or Hermitian case. However, the real nonsymmetric setting remains less explored due to the challenges associated with analyzing the eigenvalues and, consequently, the convergence behavior of iterative solvers. To address this, we employ a symmetrization technique that permutes the coefficient matrix into a real symmetric Hankel structure, whose eigenvalue distribution is known. Then, by leveraging Generalized Locally Toeplitz (GLT) theory, we develop a novel preconditioning strategy that involves centrosymmetric preconditioners, such as those derived from the $\tau$ algebra. This approach constitutes a general framework, as it relies solely on the generating function of the Toeplitz matrix, provided that it is defined. We extend all theoretical results to both the multilevel and block settings and demonstrate their effectiveness through numerical experiments on space-fractional diffusion equations, providing several examples to critically evaluate the performance of the proposed preconditioners.

\begin{thebibliography}{2}
\bibitem{1}
C. Garoni, S. Serra-Capizzano, \emph{Generalized locally Toeplitz sequences: theory and applications, vol. II}. Springer, Cham, 2018.
\bibitem{2}
S.Y. Hon, C. Li, R.L. Sormani, R. Krause, S. Serra-Capizzano, \emph{Symbol-based multilevel block $\tau$ preconditioners for multilevel block Toeplitz systems: GLT-based analysis and applications}. SIAM J. Matrix Anal. Appl., to appear
\end{thebibliography}

        \end{bibunit}
        \end{ilasabstract}
     \hypertarget{down0386}{}\begin{ilasabstract}
   \talktitle{nan}
    
    \textbf{David Sossa}, \info{17:30\textrm{--}18:00 @ SC2006 (June 26, Thursday)} \hfill \hyperlink{up0386}{$\Uparrow$}
    
    (in {\color{mstitle}MS28: From matrix theory to Euclidean Jordan algebras, FTvN systems, and beyond})
        
        \mtskip
    nan\end{ilasabstract}
     \hypertarget{down0333}{}\begin{ilasabstract}
   \talktitle{Bridging abstract and numerical linear algebra}
    
    \textbf{Sepideh Stewart}, \info{14:30\textrm{--}15:00 @ SC1005 (June 26, Thursday)} \hfill \hyperlink{up0333}{$\Uparrow$}
    
    (in {\color{mstitle}MS27: Linear algebra education})
        
        \mtskip
    Linear algebra is a key mathematics topic applicable to many other fields. Hence, equipping students with knowing linear algebra well with the ability to tackle problems numerically is an excellent way of preparing them for the future wo\
rkforce.  In this talk, we will discuss the importance of teaching both abstract and numerical linear algebra in our classrooms. We will discuss ways to transform an existing abstract linear algebra into a course that includes numerical \
linear algebra, as well as create a new course. Deciding to balance the amount of theory and computation is a challenge we often face in our lectures.  We will address some challenges of creating such courses and ways to combat them.
We welcome your input and invite you to join us as we continue this work. This is a joint work with Rachel Quinlan and Mike Michailidis.\end{ilasabstract}
     \hypertarget{down0323}{}\begin{ilasabstract}
   \talktitle{nan}
    
    \textbf{Sho Suda}, \info{13:30\textrm{--}14:00 @ SC1001 (June 26, Thursday)} \hfill \hyperlink{up0323}{$\Uparrow$}
    
    (in {\color{mstitle}MS34: Combinatorics, association scheme, and graphs})
        
        \mtskip
    nan\end{ilasabstract}
     \hypertarget{down0326}{}\begin{ilasabstract}
   \talktitle{Power difference sets and cyclotomic matrices}
    
    \textbf{Wei-Liang Sun}, \info{15:00\textrm{--}15:30 @ SC1001 (June 26, Thursday)} \hfill \hyperlink{up0326}{$\Uparrow$}
    
    (in {\color{mstitle}MS34: Combinatorics, association scheme, and graphs})
        
        \mtskip
    A difference set in a finite field is a subset where every nonzero element can be expressed as the difference of two elements from the subset, each occurring a fixed number of times. Every difference set gives rise to a symmetric balanced incomplete block design.

In 1933, R. Paley characterized when a set of squares forms a difference set. Later, S. Chowla extended this result to fourth powers. A difference set formed by $\ell$-th powers is called a power difference set. A natural question is whether all $\ell$-th power difference sets can be characterized. However, so far, researchers have only been able to construct such sets for $\ell = 2, 4, 8$ when $\ell \leq 24$, and it is conjectured that no other power difference sets exist.

Cyclotomic numbers captures the sizes of certain subsets intersections in a finite field. These numbers provide crucial information about the existence of a power difference set. The cyclotomic matrix is a matrix whose entries are cyclotomic numbers. By exploring properties of the cyclotomic matrix, we are able to conclude that if a power difference set exists, and if the size of elements outside the set is a square, then $\ell$ is congruent to $0$ or $2$ modulo $8$. This result holds for a power difference set that forms a finite projective plane. In contrast, our result allows for discussing an infinite number of $\ell$ values, extending the scope of previous finite cases. As far as we know, this may be the first application of cyclotomic matrices to power difference sets.
\end{ilasabstract}
     \hypertarget{down0036}{}\begin{ilasabstract}
   \talktitle{nan}
    
    \textbf{Marco Sutti}, \info{12:00\textrm{--}12:30 @ SC2006 (June 23, Monday)} \hfill \hyperlink{up0036}{$\Uparrow$}
    
    (in {\color{mstitle}MS22: Linear algebra applications in computational geometry})
        
        \mtskip
    nan\end{ilasabstract}
     \hypertarget{down0250}{}\begin{ilasabstract}
   \talktitle{nan}
    
    \textbf{Raymond Sze}, \info{10:30\textrm{--}11:00 @ SC0014 (June 25, Wednesday)} \hfill \hyperlink{up0250}{$\Uparrow$}
    
    (in {\color{mstitle}MS8: Tensor and quantum information science})
        
        \mtskip
    nan\end{ilasabstract}
     \hypertarget{down0339}{}\begin{ilasabstract}
   \talktitle{nan}
    
    \textbf{Daniel B. Szyld}, \info{13:30\textrm{--}14:00 @ SC2006 (June 26, Thursday)} \hfill \hyperlink{up0339}{$\Uparrow$}
    
    (in {\color{mstitle}MS13: Advances in QR factorizations})
        
        \mtskip
    nan\end{ilasabstract}
     \hypertarget{down0061}{}\begin{ilasabstract}
   \talktitle{Expansion and the normalized distance Laplacian matrix}
    
    \textbf{Michael Tait}, \info{14:00\textrm{--}14:30 @ SC1005 (June 23, Monday)} \hfill \hyperlink{up0061}{$\Uparrow$}
    
    (in {\color{mstitle}MS15: Graphs and their eigenvalues: Celebrating the work of Fan Chung Graham})
        
        \mtskip
    The normalized distance Laplacian of a graph $G$ is defined as $\mathcal{D}^\mathcal{L}(G)=T(G)^{-1/2}(T(G)-\mathcal{D}(G))T(G)^{-1/2}$ where $\mathcal{D}(G)$ is the matrix with pairwise distances between vertices and $T(G)$ is the diagonal transmission matrix. We discuss the spectral gap of this matrix and the related distance Cheeger constant. Contrary to the classical case, both of these quantities are bounded away from $0$. We characterize graphs with minimal distance Cheeger constant and we make a conjecture about graphs with minimal spectral gap. 
 
This is joint work with John Byrne, Jacob Johnston, and Carl Schildkraut.\end{ilasabstract}
     \hypertarget{down0178}{}\begin{ilasabstract}
   \talktitle{Extremal problems on graphs with $q=2$}
    
    \textbf{Michael Tait}, \info{14:00\textrm{--}14:30 @ SC1005 (June 24, Tuesday)} \hfill \hyperlink{up0178}{$\Uparrow$}
    
    (in {\color{mstitle}MS17: Graphs and matrices in honor of Leslie Hogben's retirement})
        
        \mtskip
    For a graph $G$, let $q(G)$ denote the minimum number of distinct eigenvalues over all symmetric matrices with the same zero/nonzero pattern as $G$, with the diagonal free (this family of matrices denoted $\mathcal{S}(G)$). A graph has $q(G) = 2$ if and only if there is an orthogonal matrix in $\mathcal{S}(G)$. We discuss extremal problems on graphs with $q(G)=2$. In particular we consider how sparse such graphs or their complements can be.

This is joint work with Wayne Barrett, Shaun Fallat, Vera Furst, Shahla Nasserasr, and Brendan Rooney.

\end{ilasabstract}
     \hypertarget{down0227}{}\begin{ilasabstract}
   \talktitle{On the maximal $A_\alpha$-index of graphs with a prescribed number of edges}
    
    \textbf{Bit-Shun Tam}, \info{17:00\textrm{--}17:30 @ SC2001 (June 24, Tuesday)} \hfill \hyperlink{up0227}{$\Uparrow$}
    
    (in {\color{mstitle}MS25: Enumerative/algebraic combinatorics and matrices})
        
        \mtskip
    \noindent  For any real number $\alpha\in [0,1]$, by the {\it
$A_\alpha$-matrix} of a graph $G$ we mean the matrix
$A_{\alpha}(G)=\alpha D(G)+(1-\alpha)A(G)$, where $A(G)$ and $D(G)$
are the adjacency matrix and the diagonal matrix of vertex degrees
of $G$, respectively. The largest eigenvalue of $A_{\alpha}(G)$ is
called the $A_\alpha$-index of $G$. Chang and Tam (2023) have solved
the problem of determining graphs with maximal $A_{\alpha}$-index
over $\mathcal{G}(n,m)$, the class of graphs with $n$ vertices and
$m$ edges, for $\alpha \in [\frac{1}{2},1)$ and $1\le m\le 2n-3$. In
the same paper, they posed the problem of characterizing graphs in
$\mathcal{G}(n,m)$ that maximize the $A_{\alpha}$-index for $0<
\alpha < \frac{1}{2}$ and $m\le n-1$. In this work, it is noted
that, for any $\alpha\in [0,1)$, the problem of characterizing
graphs with maximal $A_{\alpha}$-index over $\mathcal{G}(n,m)$ with
$m\le n-1$ is equivalent to the problem of characterizing graphs
with maximal $A_{\alpha}$-index over $\mathscr{S}(m)$, the class of
graphs with $m$ edges. In connection with the latter problem, we
pose the following conjecture: Let $m\ge 3$ be a positive integer
and suppose that $m={\binom{d}{2}}+t$ with $0\le t < d$. There exists
a real number $\alpha_0$, $\alpha_0=\frac{1}{2}$ for $m=3$ and
$\alpha_0\in [0,\frac{1}{2})$ for $m\ge 4$, such that for any
$\alpha \in [0,1)$, $C^m_{d+1}$ (replaced by $K_d$, in case $t=0$),
where $C^m_n$ denotes the quasi-complete graph with $n$ vertices and
$m$ edges, or $K_{1,m}$ is the unique connected graph with $m$ edges
that maximize the $A_{\alpha}$-index over $\mathscr{S}(m)$,
depending on whether $\alpha\in [0,\alpha_0)$ or $\alpha\in
(\alpha_0,1)$; when $\alpha = \alpha_0$, there are exactly two
connected graphs that maximize the $A_{\alpha}$-index over
$\mathscr{S}(m)$, namely, $C^m_{d+1}$ (or $K_d$, in case $t=0$) and
$K_{1,m}$. The conjecture is established when $t=0$.

\vspace{2mm} \noindent{\it Keywords}\,: Maximal $A_{\alpha}$-index
problem; Maximal graph; Threshold graph; Neighborhood equivalence
classes; Quasi-complete graphs; Quasi-stars.
 \vspace{2mm}

\noindent{AMS subject classification:} 05C35,\ 05C50\end{ilasabstract}
     \hypertarget{down0255}{}\begin{ilasabstract}
   \talktitle{nan}
    
    \textbf{Tin-Yau Tam}, \info{11:30\textrm{--}12:00 @ SC1001 (June 25, Wednesday)} \hfill \hyperlink{up0255}{$\Uparrow$}
    
    (in {\color{mstitle}MS10: Matrix means and related topics})
        
        \mtskip
    nan\end{ilasabstract}
     \hypertarget{down0161}{}\begin{ilasabstract}
   \talktitle{On homeomorphisms between geometric structure spaces of primitive $C^*$-algebras}
    
    \textbf{Ryotaro Tanaka}, \info{13:30\textrm{--}14:00 @ SC0012 (June 24, Tuesday)} \hfill \hyperlink{up0161}{$\Uparrow$}
    
    (in {\color{mstitle}MS12: Preserver problems, I})
        
        \mtskip
    The notion of geometric structure spaces of Banach spaces was introduced for classifying non-smooth Banach spaces under Birkhoff-James isomorphisms. The geometric structure space $\mathfrak{S}(X)$ of a Banach space $X$ reflects the geometric features of the unit ball $B_X$ of $X$, and is equipped with a closure operator which does not necessarily induce a topology. There is a natural notion of homeomorphisms between (generalized) closure spaces. Banach spaces $X$ and $Y$ are said to have homeomorphic geometric structure spaces, denoted by $X\sim_\mathfrak{S} Y$, if there exists a homeomorphism between $\mathfrak{S} (X)$ and $\mathfrak{S} (Y)$. The classification of Banach spaces with respect to $\sim_\mathfrak{S}$ is strictly coarser (and hence, the results are stronger) than that with respect to the Birkhoff-James equivalence $\sim_{BJ}$.

The purpose of this talk is to present a recent progress on the theory of geometric structure spaces of $C^*$-algebras. In particular, a description of homeomorphisms between the normal parts of geometric structure spaces of $C^*$-algebras acting irreducibly on Hilbert spaces is given.
\end{ilasabstract}
     \hypertarget{down0315}{}\begin{ilasabstract}
   \talktitle{Isometries of Lipschitz free Banach spaces}
    
    \textbf{Tamas Titkos}, \info{13:30\textrm{--}14:00 @ SC0012 (June 26, Thursday)} \hfill \hyperlink{up0315}{$\Uparrow$}
    
    (in {\color{mstitle}MS35: Preserver Problems, II})
        
        \mtskip
    In the first part of the talk, I will focus on isometries of $\mathcal{W}_p(M)$ spaces, mainly in the case when $p=1$ -- the case which is closely connected to the theory of Lipschitz-free spaces. It is known that if $F$ is an isometry of M, then its push-forward $F_{\#}$ is an isometry of $\mathcal{W}_p(M)$. A natural question arises: is this embedding surjective? We know several concrete examples where the answer is yes, but the answer, in general, is no. In the second part of the talk, I will present some new results about isometries of Lipschitz-free spaces. In particular, I will describe surjective linear isometries and linear isometry groups of a large class of Lipschitz-free spaces.
\end{ilasabstract}
     \hypertarget{down0342}{}\begin{ilasabstract}
   \talktitle{nan}
    
    \textbf{Sivan Toledo}, \info{15:00\textrm{--}15:30 @ SC2006 (June 26, Thursday)} \hfill \hyperlink{up0342}{$\Uparrow$}
    
    (in {\color{mstitle}MS13: Advances in QR factorizations})
        
        \mtskip
    nan\end{ilasabstract}
     \hypertarget{down0168}{}\begin{ilasabstract}
   \talktitle{Efficient solution of sequences of parametrized Lyapunov equations with applications}
    
    \textbf{Zoran Tomljanovic}, \info{15:00\textrm{--}15:30 @ SC0014 (June 24, Tuesday)} \hfill \hyperlink{up0168}{$\Uparrow$}
    
    (in {\color{mstitle}MS6: Model reduction})
        
        \mtskip
    The solutions of parametrized Lyapunov equations frequently serve as intermediate steps in a broader procedure aimed at computing $trace(EX)$, where $X$ represents the solution to the Lyapunov equation and $E$ is a given matrix. Our focus is on studying problems where the parameter dependence of the coefficient matrix is encoded as a low-rank modification of a fixed, seed matrix.\\
We propose two novel numerical procedures that fully exploit such a common structure. The first one builds upon the Sherman-Morrison-Woodbury   formula and recycling Krylov techniques, and it is well-suited for small dimensional problems as it uses dense numerical linear algebra tools. The second algorithm can instead address large-scale problems by relying on state-of-the-art projection techniques based on the extended Krylov subspace. We test the new algorithms on several problems arising in studying damped vibrational systems and analyzing output synchronization problems for multi-agent systems. Our results show that the proposed algorithms are superior to state-of-the-art techniques as they can remarkably speed up the computation of accurate solutions.\\
This is joint work with Davide Palitta, Ivica Naki\'{c} and Jens Saak.\end{ilasabstract}
     \hypertarget{down0281}{}\begin{ilasabstract}
   \talktitle{nan}
    
    \textbf{Ming-Cheng Tsai}, \info{11:00\textrm{--}11:30 @ SC0012 (June 26, Thursday)} \hfill \hyperlink{up0281}{$\Uparrow$}
    
    (in {\color{mstitle}MS35: Preserver Problems, II})
        
        \mtskip
    nan\end{ilasabstract}
     \hypertarget{down0370}{}\begin{ilasabstract}
   \talktitle{nan}
    
    \textbf{Shen-Fu Tsai}, \info{17:30\textrm{--}18:00 @ SC1001 (June 26, Thursday)} \hfill \hyperlink{up0370}{$\Uparrow$}
    
    (in {\color{mstitle}MS25: Enumerative/algebraic combinatorics and matrices})
        
        \mtskip
    nan\end{ilasabstract}
     \hypertarget{down0288}{}\begin{ilasabstract}
   \talktitle{nan}
    
    \textbf{Pin-Chieh Tseng}, \info{11:30\textrm{--}12:00 @ SC1001 (June 26, Thursday)} \hfill \hyperlink{up0288}{$\Uparrow$}
    
    (in {\color{mstitle}MS34: Combinatorics, association scheme, and graphs})
        
        \mtskip
    nan\end{ilasabstract}
     \hypertarget{down0075}{}\begin{ilasabstract}
   \talktitle{nan}
    
    \textbf{ShengLi Tzeng}, \info{15:00\textrm{--}15:30 @ SC4011 (June 23, Monday)} \hfill \hyperlink{up0075}{$\Uparrow$}
    
    (in {\color{mstitle}MS20: Manifold learning and statistical applications})
        
        \mtskip
    nan\end{ilasabstract}
     \hypertarget{down0260}{}\begin{ilasabstract}
   \talktitle{nan}
    
    \textbf{Frank Uhlig}, \info{11:00\textrm{--}11:30 @ SC1005 (June 25, Wednesday)} \hfill \hyperlink{up0260}{$\Uparrow$}
    
    (in {\color{mstitle}MS27: Linear algebra education})
        
        \mtskip
    nan\end{ilasabstract}
     \hypertarget{down0142}{}\begin{ilasabstract}
   \talktitle{Matrix sign patterns that allow the strong multiplicity property}
    
    \textbf{Kevin Vander Meulen}, \info{11:00\textrm{--}11:30 @ SC2001 (June 24, Tuesday)} \hfill \hyperlink{up0142}{$\Uparrow$}
    
    (in {\color{mstitle}MS2: Combinatorial matrix theory})
        
        \mtskip
    Various strong multiplicity properties have been developed to aid
with inverse eigenvalue problems. The nonsymmetric strong multiplicity
property (nSMP) has recently been introduced to provide information
about the possible eigenvalue multiplicities of a matrix based on the
sign pattern of the matrix. The nSMP is useful for the problem
of determining the number of distinct eigenvalues allowed by a pattern.
We provide a characterization of the patterns that allow
the nSMP. We also describe classes of patterns of arbitrary order
that require the nSMP. This presentation includes joint work
with Abhilash Saha, Leona Tilis, and Adam Van Tuyl.
\end{ilasabstract}
     \hypertarget{down0392}{}\begin{ilasabstract}
   \talktitle{An inverse eigenvalue problem linked to multiple orthogonal polynomials
}
    
    \textbf{Robbe Vermeiren}, \info{16:30\textrm{--}17:00 @ SC4011 (June 26, Thursday)} \hfill \hyperlink{up0392}{$\Uparrow$}
    
    (in {\color{mstitle}MS23: Advances in Krylov subspace methods and their applications})
        
        \mtskip
    Multiple orthogonal polynomials (MOPs) arise in various applications, including approximation theory, random matrix theory, and numerical integration. To define MOPs, one needs orthogonality conditions with respect to multiple measures. In this talk, we restrict our attention to the case of two measures. These MOPs satisfy recurrence relations, and we focus specifically on the stepline recurrence relation.
We derive an inverse eigenvalue problem: given some initial spectral data, retrieve the recurrence matrix associated with the stepline recurrence relation. Several techniques for solving this inverse problem are proposed, and numerical illustrations are provided to demonstrate their correctness.
\end{ilasabstract}
     \hypertarget{down0388}{}\begin{ilasabstract}
   \talktitle{nan}
    
    \textbf{Bastien Vieublé}, \info{16:30\textrm{--}17:00 @ SC3001 (June 26, Thursday)} \hfill \hyperlink{up0388}{$\Uparrow$}
    
    (in {\color{mstitle}MS16: Approximations and errors in Krylov-based solvers})
        
        \mtskip
    nan\end{ilasabstract}
     \hypertarget{down0126}{}\begin{ilasabstract}
   \talktitle{Isometries and metric properties of quantum Wasserstein distances}
    
    \textbf{Dániel Virosztek}, \info{10:30\textrm{--}11:00 @ SC0012 (June 24, Tuesday)} \hfill \hyperlink{up0126}{$\Uparrow$}
    
    (in {\color{mstitle}MS12: Preserver problems, I})
        
        \mtskip
    Although the theory of classical optimal transport has been playing an important role in mathematical physics (especially in fluid dynamics) and probability since the late 80s, concepts of optimal transportation in quantum mechanics have emerged only very recently. We briefly review the most relevant approaches and discuss a non-quadratic generalization of the quantum mechanical optimal transport problem introduced by De Palma and Trevisan where quantum channels realize the transport. Relying on this general machinery, we introduce p-Wasserstein distances and divergences and study their fundamental geometric properties. Finally, we demonstrate that the quadratic quantum Wasserstein divergences are genuine metrics, and summarize our recent results on the isometries of the qubit state space with respect to Wasserstein distances induced by distinguished transport cost operators.
\end{ilasabstract}
     \hypertarget{down0246}{}\begin{ilasabstract}
   \talktitle{nan}
    
    \textbf{Jani A. Virtanen}, \info{11:30\textrm{--}12:00 @ SC0009 (June 25, Wednesday)} \hfill \hyperlink{up0246}{$\Uparrow$}
    
    (in {\color{mstitle}MS33: Norms of matrices, numerical range, applications of functional analysis to matrix theory})
        
        \mtskip
    nan\end{ilasabstract}
     \hypertarget{down0045}{}\begin{ilasabstract}
   \talktitle{nan}
    
    \textbf{Anna Vishnyakova}, \info{15:00\textrm{--}15:30 @ SC0008 (June 23, Monday)} \hfill \hyperlink{up0045}{$\Uparrow$}
    
    (in {\color{mstitle}MS9: Total positivity})
        
        \mtskip
    nan\end{ilasabstract}
     \hypertarget{down0134}{}\begin{ilasabstract}
   \talktitle{nan}
    
    \textbf{Trung Dung Vuong}, \info{11:30\textrm{--}12:00 @ SC1001 (June 24, Tuesday)} \hfill \hyperlink{up0134}{$\Uparrow$}
    
    (in {\color{mstitle}MS10: Matrix means and related topics})
        
        \mtskip
    nan\end{ilasabstract}
     \hypertarget{down0124}{}\begin{ilasabstract}
   \talktitle{On some multi-variable means of ALM type}
    
    \textbf{Shuhei Wada}, \info{11:00\textrm{--}11:30 @ SC0009 (June 24, Tuesday)} \hfill \hyperlink{up0124}{$\Uparrow$}
    
    (in {\color{mstitle}MS29: Matrix functions and related topics})
        
        \mtskip
    The study of multivariable operator geometric means was pioneered by Ando, Li, and Mathias (ALM). In recent years, research has primarily focused on operator means obtained as solutions to operator equations. Nevertheless, there has been growing interest in operator means of ALM type, which are obtained by generalizing their approach. ALM-type means are particularly notable because the approximating sequences are explicitly given, making it easier to understand their properties. In this talk, we will introduce an example of an ALM-type mean and discuss its properties. Additionally, we will touch upon related results.
\end{ilasabstract}
     \hypertarget{down0041}{}\begin{ilasabstract}
   \talktitle{nan}
    
    \textbf{Shao-Hsuan Wang}, \info{11:30\textrm{--}12:00 @ SC4011 (June 23, Monday)} \hfill \hyperlink{up0041}{$\Uparrow$}
    
    (in {\color{mstitle}MS20: Manifold learning and statistical applications})
        
        \mtskip
    nan\end{ilasabstract}
     \hypertarget{down0297}{}\begin{ilasabstract}
   \talktitle{nan}
    
    \textbf{Wei Wang}, \info{11:30\textrm{--}12:00 @ SC2001 (June 26, Thursday)} \hfill \hyperlink{up0297}{$\Uparrow$}
    
    (in {\color{mstitle}MS2: Combinatorial matrix theory})
        
        \mtskip
    nan\end{ilasabstract}
     \hypertarget{down0316}{}\begin{ilasabstract}
   \talktitle{Linear maps preserving disjoint idempotents
}
    
    \textbf{Ya-Shu Wang}, \info{14:00\textrm{--}14:30 @ SC0012 (June 26, Thursday)} \hfill \hyperlink{up0316}{$\Uparrow$}
    
    (in {\color{mstitle}MS35: Preserver Problems, II})
        
        \mtskip
    Let ${\bf M}_n(F)$ denote the set of all $n \times n$ matrices and ${\bf S}_n(F)$ denote the set of symmetric $n \times n$ matrices over a field $F$, respectively.
In this talk, I will present a characterization of linear maps on ${\bf M}_n(F)$ and ${\bf S}_n(F)$ that send disjoint rank one idempotents to disjoint idempotents. As an application,  I will also characterize linear maps on ${\bf M}_n(F)$ and ${\bf S}_n(F)$ that preserve matrices annihilated by a fixed polynomial under certain assumptions.
\end{ilasabstract}
     \hypertarget{down0058}{}\begin{ilasabstract}
   \talktitle{nan}
    
    \textbf{Aaron Welters}, \info{14:00\textrm{--}14:30 @ SC1003 (June 23, Monday)} \hfill \hyperlink{up0058}{$\Uparrow$}
    
    (in {\color{mstitle}MS26: Utilizing structure to achieve low-complexity algorithms for data science, engineering, and physics})
        
        \mtskip
    nan\end{ilasabstract}
     \hypertarget{down0273}{}\begin{ilasabstract}
   \talktitle{Generalized Smith method for large-scale nonsymmetric algebraic Riccati equations}
    
    \textbf{Peter Chang-Yi Weng}, \info{11:30\textrm{--}12:00 @ SC4011 (June 25, Wednesday)} \hfill \hyperlink{up0273}{$\Uparrow$}
    
    (in {\color{mstitle}MS1: Embracing new opportunities in numerical linear algebra})
        
        \mtskip
    This paper presents an effective algorithm about a computation of the numerical
low-rank solution to a large-scale nonsymmetric algebraic Riccati equation with
a nonsingular $M$-matrix. The method first applies the Newton’s method to compute the nonsymmetric algebraic Riccati equation, followed by a derivation of
generalized Sylvester equations. By extending the Smith method, the proposed
algorithm achieves a quadratic convergence and has the $O(n)$ computational
complexity. A detailed convergence, error analysis, truncation and compression
process, and numerical examples will be provided.\end{ilasabstract}
     \hypertarget{down0130}{}\begin{ilasabstract}
   \talktitle{Reduced-order modeling of mechanical systems via structured barycentric forms}
    
    \textbf{Steffen W. R. Werner}, \info{11:00\textrm{--}11:30 @ SC0014 (June 24, Tuesday)} \hfill \hyperlink{up0130}{$\Uparrow$}
    
    (in {\color{mstitle}MS6: Model reduction})
        
        \mtskip
    \begin{bibunit}
        Data-driven reduced-order modeling is an essential tool in constructing
high-fidelity compact models to approximate physical phenomena when explicit
models, such as state-space formulations with access to internal variables, are
not available yet abundant input/output data are.
When deriving models from theoretical principles, the results
typically contain differential structures that lead to certain system
properties.
A common example are dynamical systems with second-order time derivatives
like
\begin{equation*}
  M \ddot{x}(t) + D \dot{x}(t) + K x(t) = B u(t), \quad
  y(t) = C x(t),
\end{equation*}
arising in the modeling of mechanical or electro-mechanical processes.
In this case, data are often available in the frequency domain, where the
systems' input-to-output behavior is described by rational functions
of the form
\begin{equation*}
  H(s) = C (s^{2} M + s D + K)^{-1} B,
\end{equation*}
rather than differential equations.
Classical frequency domain approaches like the Loewner framework, vector
fitting and AAA are available and can be used to learn unstructured
(first-order) models from data.
However, these models in their original formulation do not reflect the
structure-inherited properties.
The key element in the derivation of frequency domain approaches is the
barycentric form of rational functions.
In this work, we present a structured extension of the barycentric form for the
case of mechanical systems.
This structured barycentric form is given by
\begin{equation*}
  \widehat{H}(s) = \left(\sum_{j = 1}^k\frac{h_{j} w_{j}}{(s - \lambda_{j})
    (s - \sigma_{j})} \right) \left/ \left(1 + \sum_{j = 1}^{k}
    \frac{w_{j}}{(s - \lambda_{j})(s - \sigma_{j})}\right)\right.;
\end{equation*}
see~\cite{WernerSWR_GosGW24}.
Building on this structured rational function representation, we develop new
algorithms for learning reduced-order models of mechanical phenomena in the
frequency domain, while enforcing the mechanical system structure in the model
description.

\begin{thebibliography}{10}
\bibitem{WernerSWR_GosGW24}
  I.~V. Gosea, S.~Gugercin, and S.~W.~R. Werner.
  \newblock Structured barycentric forms for interpolation-based data-driven
    reduced modeling of second-order systems.
  \newblock {\em Adv. Comput. Math.}, 50(2):26, 2024.
  \newblock doi:10.1007/s10444-024-10118-7
\end{thebibliography}

        \end{bibunit}
        \end{ilasabstract}
     \hypertarget{down0363}{}\begin{ilasabstract}
   \talktitle{Efficiently solving nonstandard Riccati equations via indefinite factorizations}
    
    \textbf{Steffen W. R. Werner}, \info{16:00\textrm{--}16:30 @ SC0014 (June 26, Thursday)} \hfill \hyperlink{up0363}{$\Uparrow$}
    
    (in {\color{mstitle}MS5: Advances in matrix equations: Theory, computations, and applications})
        
        \mtskip
    \begin{bibunit}
        The continuous-time symmetric algebraic Riccati equation (CARE) is a special
type of nonlinear matrix-valued equation and an essential component
of many applications, including controller design, model order reduction and
game theory.
While there have been many developments in recent years regarding new solution
methods of CAREs in the setting of large-scale sparse coefficient matrices,
these methods are typically based on semi-definite low-rank factorizations of
their solutions and consider the classical CARE formulation
\begin{equation*}
  A^{\mathsf{T}} X E + E^{\mathsf{T}} X A + C^{\mathsf{T}} Q C -
    E^{\mathsf{T}} X B R B^{\mathsf{T}} X E = 0,
\end{equation*}
with $Q$ and $R$ symmetric positive definite.
In this work, we investigate solution methods for large-scale sparse generalized
CAREs of the form
\begin{equation*}
  A^{\mathsf{T}} X E + E^{\mathsf{T}} X A + C^{\mathsf{T}} Q C -
    (E^{\mathsf{T}} X B + S) R (E^{\mathsf{T}} X B + S)^{\mathsf{T}} = 0,
\end{equation*}
where $Q$ and $R$ can be symmetric positive definite,  negative definite,
or even indefinite.
The solutions and intermediate approximations to such general equations often
do not follow the classical semi-definite structure.
Therefore, we are utilizing low-rank symmetric indefinite $LDL^{\mathsf{T}}$
factorizations of the CARE solution in our algorithms, which enable efficient
computations; see, for example,~\cite{Werner_SaaW24}.

\begin{thebibliography}{10}
\bibitem{Werner_SaaW24}
J.~Saak and S.~W.~R. Werner.
\newblock Using {$LDL^{T}$} factorizations in {N}ewton's method for solving
  general large-scale algebraic {R}iccati equations.
\newblock {\em Electron. Trans. Numer. Anal.}, 62:95--118, 2024.
\newblock doi:10.1553/etna\_vol62s95
\end{thebibliography}

        \end{bibunit}
        \end{ilasabstract}
     \hypertarget{down0317}{}\begin{ilasabstract}
   \talktitle{nan}
    
    \textbf{Ngai-Ching Wong}, \info{14:30\textrm{--}15:00 @ SC0012 (June 26, Thursday)} \hfill \hyperlink{up0317}{$\Uparrow$}
    
    (in {\color{mstitle}MS35: Preserver Problems, II})
        
        \mtskip
    nan\end{ilasabstract}
     \hypertarget{down0233}{}\begin{ilasabstract}
   \talktitle{nan}
    
    \textbf{Chin-Tien Wu}, \info{16:00\textrm{--}16:30 @ SC3001 (June 24, Tuesday)} \hfill \hyperlink{up0233}{$\Uparrow$}
    
    (in {\color{mstitle}MS11: Structured matrix computations and its applications})
        
        \mtskip
    nan\end{ilasabstract}
     \hypertarget{down0011}{}\begin{ilasabstract}
   \talktitle{Linear preservers of sign regularity}
    
    \textbf{Shivangi Yadav}, \info{11:30\textrm{--}12:00 @ SC0008 (June 23, Monday)} \hfill \hyperlink{up0011}{$\Uparrow$}
    
    (in {\color{mstitle}MS9: Total positivity})
        
        \mtskip
    The classification of linear maps that act on a space of bounded linear operators and preserve certain functions, subsets, relations, etc. has a long history, beginning with Frobenius, who characterized in 1897 the  determinant-preserving linear maps on matrix algebras. In this talk, I will present a classification of all surjective linear mappings $\mathcal{L}:\mathbb{R}^{m\times n}\to\mathbb{R}^{m\times n}$ that preserve: (i)~sign regularity and (ii)~sign regularity with a given sign pattern, as well as (iii)~strict versions of these. As a special case of our results, we recover the characterization of linear preservers for the class of square totally positive and totally non-negative matrices (by Berman--Hershkowitz--Johnson in 1985). This is a joint work with Projesh Nath Choudhury.
\end{ilasabstract}
     \hypertarget{down0022}{}\begin{ilasabstract}
   \talktitle{Stability of AN-operators under functional calculus}
    
    \textbf{Takeaki Yamazaki}, \info{11:00\textrm{--}11:30 @ SC1001 (June 23, Monday)} \hfill \hyperlink{up0022}{$\Uparrow$}
    
    (in {\color{mstitle}MS10: Matrix means and related topics})
        
        \mtskip
    This talk is based on [1].
Let $\mathcal{B}(H)$ be the set of all bounded linear operators on a complex Hilbert space. An operator $T\in \mathcal{B}(H)$ satisfies $\mathcal{AN}$-property if and only if for any closed subspace $K$ of $H$, there exists a unit vector $x\in K$ such that $\|T|_{K}\|=\|T|_{K}x\|$. The set of operators satisfying $\mathcal{AN}$-property is not closed. 
Moreover $\mathcal{AN}$-property is not stable under some operations.
In this talk, we shall introduce stability of $\mathcal{AN}$-property under functional calculus on positive definite operators.

This is a joint work with Professor Golla Ramesh, Hiroyuki Osaka and Yoichi Udagawa.

\noindent
[1] G. Ramesh, H. Osaka,
Y. Udagawa and T. Yamazaki, {\it Stability of  $\mathcal{AN}$ -operators under functional calculus},
Anal. Math. {\bf 49} (2023), 825--839.
\end{ilasabstract}
     \hypertarget{down0185}{}\begin{ilasabstract}
   \talktitle{Nested Grassmannians for dimensionality reduction with applications}
    
    \textbf{Chun-Hao Yang}, \info{13:30\textrm{--}14:00 @ SC2006 (June 24, Tuesday)} \hfill \hyperlink{up0185}{$\Uparrow$}
    
    (in {\color{mstitle}MS32: Advances in matrix manifold optimization})
        
        \mtskip
    In the recent past, nested structure of Riemannian manifolds has been studied in the context of dimensionality reduction as an alternative to the popular principal geodesic analysis (PGA) technique, for example, the principal nested spheres. In this paper, we propose a novel framework for constructing a nested sequence of homogeneous Riemannian manifolds. Common examples of homogeneous Riemannian manifolds include the spheres, the Stiefel manifolds, and the Grassmann manifolds. In particular, we focus on applying the proposed framework to the Grassmann manifolds, giving rise to the nested Grassmannians (NG). An important application in which Grassmann manifolds are encountered is planar shape analysis. Specifically, each planar (2D) shape can be represented as a point in the complex projective space which is a complex Grassmann manifold. Some salient features of our framework are: (i) it explicitly exploits the geometry of the homogeneous Riemannain manifolds and (ii) the nested lower-dimensional submanifolds need not be geodesic. With the proposed NG structure, we develop algorithms for the supervised and unsupervised dimensionality reduction problems respectively. The proposed algorithms are compared with PGA via simulation studies and real data experiments and are shown to achieve a higher ratio of expressed variance compared to PGA.
\end{ilasabstract}
     \hypertarget{down0150}{}\begin{ilasabstract}
   \talktitle{Landmark diffusion accelerates alternating diffusion maps for multi-sensor fusion}
    
    \textbf{Sing-Yuan Yeh}, \info{10:30\textrm{--}11:00 @ SC4011 (June 24, Tuesday)} \hfill \hyperlink{up0150}{$\Uparrow$}
    
    (in {\color{mstitle}MS20: Manifold learning and statistical applications})
        
        \mtskip
    Alternating Diffusion (AD) is a commonly applied diffusion-based sensor fusion algorithm. While it has been successfully applied to various problems, its computational burden remains limited. Inspired by the landmark diffusion idea considered in the Robust and Scalable Embedding via Landmark Diffusion (ROSELAND), we propose a variation of AD, called Landmark AD (LAD), which captures the essence of AD while offering superior computational efficiency. Through a series of theoretical analyses, we investigate the sample complexity of LAD within the manifold setup. We then apply LAD to address the automatic sleep stage annotation problem using two electroencephalogram channels, demonstrating its efficacy in practical applications.
\end{ilasabstract}
     \hypertarget{down0362}{}\begin{ilasabstract}
   \talktitle{nan}
    
    \textbf{Sang-Gyun Youn}, \info{17:30\textrm{--}18:00 @ SC0012 (June 26, Thursday)} \hfill \hyperlink{up0362}{$\Uparrow$}
    
    (in {\color{mstitle}MS7: Linear algebra and quantum information science})
        
        \mtskip
    nan\end{ilasabstract}
     \hypertarget{down0224}{}\begin{ilasabstract}
   \talktitle{Relationships between minimum rank problem parameters for cobipartite graphs}
    
    \textbf{Derek Young}, \info{17:30\textrm{--}18:00 @ SC1005 (June 24, Tuesday)} \hfill \hyperlink{up0224}{$\Uparrow$}
    
    (in {\color{mstitle}MS17: Graphs and matrices in honor of Leslie Hogben's retirement})
        
        \mtskip
    The minimum rank problem for zero-nonzero matrix patterns is to determine the
smallest rank of a matrix whose zero entries occur in specified positions.
Similarly, the minimum rank problem for a simple graph is to find the smallest
rank of a symmetric matrix whose off-diagonal nonzero entries occur according
to the edges of a given graph.  In each case, a fundamental combinatorial lower
bound exists; for the former, it is the triangle number of the pattern, while
for the latter it is the zero forcing number of the graph.  For a given
zero-nonzero pattern, there exists an associated cobipartite graph.  In
previous work, the minimum rank of the pattern and the maximum nullity of its
associated cobipartite graph were shown to obey a simple relationship; each is
equal to the number of vertices in the graph minus the other.  We show that the
corresponding bounds (that of the triangle number and the zero forcing number)
obey this same relationship.  This forms a connection between the goal of
understanding when the triangle number is equal to the minimum rank of a
pattern and that of determining when the zero forcing number of a graph is
equal to its maximum nullity.
\end{ilasabstract}
     \hypertarget{down0068}{}\begin{ilasabstract}
   \talktitle{Authalic energy minimization for area-preserving mappings}
    
    \textbf{Mei-Heng Yueh}, \info{14:30\textrm{--}15:00 @ SC2006 (June 23, Monday)} \hfill \hyperlink{up0068}{$\Uparrow$}
    
    (in {\color{mstitle}MS22: Linear algebra applications in computational geometry})
        
        \mtskip
    The authalic energy is a functional specifically designed to measure area distortion in surface mappings, providing a foundation for the efficient computation of area-preserving mappings of simplicial surfaces. Such mappings can serve as parameterizations that define a unified coordinate system, which simplifies various image and geometry processing tasks, such as surface registration and blending. In this talk, I will introduce authalic energy minimization for computing area-preserving simplicial mappings and demonstrate its practical utility in geometry processing.
\end{ilasabstract}
     \hypertarget{down0184}{}\begin{ilasabstract}
   \talktitle{The critical groups of hypercubes and beyond}
    
    \textbf{Chi Ho Yuen}, \info{15:00\textrm{--}15:30 @ SC2001 (June 24, Tuesday)} \hfill \hyperlink{up0184}{$\Uparrow$}
    
    (in {\color{mstitle}MS25: Enumerative/algebraic combinatorics and matrices})
        
        \mtskip
    The critical group of a graph $G$ is the torsion part of the cokernel of its Laplacian. It refines the number of spanning trees of $G$ in an algebraic way and is related to the chip-firing game (or abelian sandpile model) on $G$. In this talk, we survey several results on the critical groups of the hypercube graphs and their generalizations, including my works on Cayley graphs (joint with J. Gao, J. Marx-Kuo, and V. McDonald) and Adinkras (partly joint with K. Iga, C. Klivans, J. Kostiuk, and with K. Hung), which are decorated graphs introduced by physicists to encode special supersymmetry algebras. The emphasis will be on the novel algebraic techniques employed to prove these results.\end{ilasabstract}
     \hypertarget{down0146}{}\begin{ilasabstract}
   \talktitle{nan}
    
    \textbf{Yang Zhang}, \info{11:30\textrm{--}12:00 @ SC2006 (June 24, Tuesday)} \hfill \hyperlink{up0146}{$\Uparrow$}
    
    (in {\color{mstitle}MS3: Matrix inequalities with applications})
        
        \mtskip
    nan\end{ilasabstract}
     \hypertarget{down0387}{}\begin{ilasabstract}
   \talktitle{nan}
    
    \textbf{Ning Zheng}, \info{16:00\textrm{--}16:30 @ SC3001 (June 26, Thursday)} \hfill \hyperlink{up0387}{$\Uparrow$}
    
    (in {\color{mstitle}MS16: Approximations and errors in Krylov-based solvers})
        
        \mtskip
    nan\end{ilasabstract}
     \hypertarget{down0209}{}\begin{ilasabstract}
   \talktitle{nan}
    
    \textbf{Ralf Zimmerman}, \info{16:00\textrm{--}16:30 @ SC0014 (June 24, Tuesday)} \hfill \hyperlink{up0209}{$\Uparrow$}
    
    (in {\color{mstitle}MS6: Model reduction})
        
        \mtskip
    nan\end{ilasabstract}
     \hypertarget{down0251}{}\begin{ilasabstract}
   \talktitle{nan}
    
    \textbf{Jeroen Zuiddam}, \info{11:00\textrm{--}11:30 @ SC0014 (June 25, Wednesday)} \hfill \hyperlink{up0251}{$\Uparrow$}
    
    (in {\color{mstitle}MS8: Tensor and quantum information science})
        
        \mtskip
    nan\end{ilasabstract}
     \hypertarget{down0166}{}\begin{ilasabstract}
   \talktitle{nan}
    
    \textbf{Eric de Sturler}, \info{14:00\textrm{--}14:30 @ SC0014 (June 24, Tuesday)} \hfill \hyperlink{up0166}{$\Uparrow$}
    
    (in {\color{mstitle}MS6: Model reduction})
        
        \mtskip
    nan\end{ilasabstract}
     \hypertarget{down0263}{}\begin{ilasabstract}
   \talktitle{Stochastic matrices with infinitely many stochastic roots}
    
    \textbf{Helena Šmigoc}, \info{11:00\textrm{--}11:30 @ SC2001 (June 25, Wednesday)} \hfill \hyperlink{up0263}{$\Uparrow$}
    
    (in {\color{mstitle}MS2: Combinatorial matrix theory})
        
        \mtskip
    We introduce and study the class of arbitrarily finely divisible stochastic matrices ($\mathrm{AFD}_+$-matrices): stochastic matrices that have a stochastic $c$-th root for infinitely many natural numbers $c$. This notion generalises the class of embeddable stochastic matrices. 
We will explore the connection between the spectral properties of an  $\mathrm{AFD}_+$-matrix $A$ and the spectral properties of a limit point $L$ of its stochastic roots. This connection, which is first formalised in the broader context of complex and real square matrices, poses restrictions on $A$ assuming $L$ is given. For example, if an $\mathrm{AFD}_+$-matrix $A$ has a corresponding irreducible limit point $L$, then $A$ has to be a circulant matrix. We close with a complete characterisation of $\mathrm{AFD}_+$-matrices of rank-two.\end{ilasabstract}
     \hypertarget{down0331}{}\begin{ilasabstract}
   \talktitle{Posing a question}
    
    \textbf{Helena Šmigoc}, \info{13:30\textrm{--}14:00 @ SC1005 (June 26, Thursday)} \hfill \hyperlink{up0331}{$\Uparrow$}
    
    (in {\color{mstitle}MS27: Linear algebra education})
        
        \mtskip
    Linear algebra is included in the curriculum for students pursuing various degrees such as computer science, biology, engineering, and pure mathematics. When students from different fields are in the same classroom, it can be challenging to tailor instruction to meet everyone's needs.  We will explore ways of setting a problem in this context, with the aim to motivate students with diverse interests, and  develop their understanding of the topic.
\end{ilasabstract}
    \newpage

\section{Abstracts of Contributed Talks}
\newpage

\end{document}
